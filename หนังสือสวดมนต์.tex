%!TEX program = xelatex
%!TEX encoding = UTF-8 Unicode 
\documentclass{article}
\usepackage{fontspec}
\defaultfontfeatures{Mapping=tex-text}
\usepackage{xunicode}
\usepackage{xltxtra}
\setmainfont{TH SarabunPSK}
\XeTeXlinebreaklocale 'th_TH' 
\usepackage{amsmath}
\usepackage{amsfonts}
\usepackage{amssymb}
\usepackage{csquotes}
\usepackage{mathtools}
\usepackage{pgfplots}

\usepackage{color}
\definecolor{white}{rgb}{1,1,1}
\definecolor{dkgreen}{rgb}{0,0.6,0}
\definecolor{gray}{rgb}{0.5,0.5,0.5}
\definecolor{mauve}{rgb}{0.58,0,0.82}
\definecolor{cream}{rgb}{1, 0.992, 0.816}
\definecolor{lightyellow}{rgb}{1, 1, 0.875}

\title{หนังสือสวดมนต์แปล}
\date{\today}

\begin{document}
\pagecolor{lightyellow}
\maketitle
\newpage
\tableofcontents

\newpage
\section{คำบูชาพระรัตนตรัย}
โย โส ภะคะวา อะระหัง สัมมาสัมพุทโธ
\begin{description}
พระผู้มีพระภาคเจ้านั้น พระองค์ใด, เป็นพระอรหันต์, 
ดับเพลิงกิเลสเพลิงทุกข์สิ้นเชิง, ตรัสรู้ชอบได้โดยพระองค์เอง
\end{description}
ส๎วากขาโต เยนะ ภะคะวะตา ธัมโม 
\begin{description}
พระธรรม เป็นธรรมอันพระผู้มีพระภาคเจ้า พระองค์ใด, ตรัสไว้ดีแล้ว
\end{description}
สุปะฏิปันโน ยัสสะ ภะคะวะโต สาวะกะสังโฆ
\begin{description}
พระสงฆ์สาวกของพระผู้มีพระภาคเจ้า พระองค์ใด, ปฏิบัติดีแล้ว
\end{description}
ตัมมะยัง ภะคะวันตัง สะธัมมัง สะสังฆัง
อิเมหิ สักกาเรหิ ยะถาระหัง อาโรปิเตหิ อะภิปูชะยามะ
\begin{description}
ข้าพเจ้าทั้งหลาย, ขอบูชาอย่างยิ่งซึ่งพระผู้มีพระภาคเจ้าพระองค์นั้น
,พร้อมทั้งพระธรรมและพระสงฆ์,
ด้วยเครื่องสักการะทั้งหลายเหล่านี้, อันยกขึ้นตามสมควรแล้วอย่างไร
\end{description}
สาธุ โน ภันเต ภะคะวา สุจิระปะรินิพพุโตปิ
\begin{description}
ข้าแต่พระองค์ผู้เจริญ, พระผู้มีพระภาคเจ้าแม้ปรินิพพานนานแล้ว,
ทรงสร้างคุณอันสำเร็จประโยชน์ไว้แก่ข้าพเจ้าทั้งหลาย.
\end{description}
ปัจฉิมาชะนะตานุกัมปะมานะสา
\begin{description}
ทรงมีพระหฤทัยอนุเคราะห์แก่พวกข้าพเจ้า อันเป็นชนรุ่นหลัง
\end{description}
อิเม สักกาเร ทุคคะตะปัณณาการะภูเต ปะฏิคคัณหาตุ
\begin{description}
ขอพระผู้มีพระภาคเจ้าจงรับเครื่องสักการะ อันเป็นบรรณาการของคน
ยากทั้งหลายเหล่านี้
\end{description}
อัมหากัง ทีฆะรัตตัง หิตายะ สุขายะ
\begin{description}
เพื่อประโยชน์และความสุขแก่ข้าพเจ้าทั้งหลาย ตลอดกาลนาน เทอญฯ
\end{description}
อะระหัง สัมมาสัมพุทโธ ภะคะวา
\begin{description}
พระผู้มีพระภาคเจ้า, เป็นพระอรหันต์, ดับเพลิงกิเลสเพลิงทุกข์สิ้นเชิง,
ตรัสรู้ชอบได้โดยพระองค์เอง
\end{description}
พุทธัง ภะคะวันตัง อะภิวาเทมิ
\begin{description}
ข้าพเจ้าอภิวาทพระผู้มีพระภาคเจ้า, ผู้รู้ ผู้ตื่น ผู้เบิกบาน (กราบ)
\end{description}
ส๎วากขาโต ภะคะวะตา ธัมโม 
\begin{description}
พระธรรมเป็นธรรมที่พระผู้มีพระภาคเจ้า, ตรัสไว้ดีแล้ว
\end{description}
ธัมมัง นะมัสสามิ
\begin{description}
ข้าพเจ้านมัสการพระธรรม (กราบ)
\end{description}
สุปะฏิปันโน ภะคะวะโต สาวะกะสังโฆ,
\begin{description}
พระสงฆ์สาวกของพระผู้มีพระภาคเจ้า, ปฏิบัติดีแล้ว
\end{description}
สังฆัง นะมามิ.
\begin{description}
ข้าพเจ้านอบน้อมพระสงฆ์ (กราบ)
\end{description}
\newpage

\end{document}