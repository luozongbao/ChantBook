%!TEX program = xelatex
%!TEX encoding = UTF-8 Unicode 
\documentclass[12pt]{article}
\usepackage{fontspec}
\usepackage[a5paper]{geometry}
\defaultfontfeatures{Mapping=tex-text}
\usepackage{xunicode}
\usepackage{xltxtra}
\setmainfont{TH SarabunPSK}
\XeTeXlinebreaklocale 'th_TH' 
\usepackage{amsmath}
\usepackage{amsfonts}
\usepackage{amssymb}
\usepackage{csquotes}
\usepackage{mathtools}
\usepackage{pgfplots}

\usepackage{color}
\definecolor{white}{rgb}{1,1,1}
\definecolor{dkgreen}{rgb}{0,0.6,0}
\definecolor{gray}{rgb}{0.5,0.5,0.5}
\definecolor{mauve}{rgb}{0.58,0,0.82}
\definecolor{cream}{rgb}{1, 0.992, 0.816}
\definecolor{lightyellow}{rgb}{1, 1, 0.875}



\title{หนังสือสวดมนต์แปล}
\date{\today}
\newfontfamily{\Title}[Scale=3]{TH SarabunPSK}

\begin{document}
\pagecolor{lightyellow}
\maketitle
\newpage
\tableofcontents

\pagebreak

\begin{titlepage}

\vspace*{\fill}
\begin{center}
  \scalebox{3}{\textbf{คำทำวัตรเช้า}}
\end{center}
\vspace{\fill}

\end{titlepage}
\pagebreak


\section{คำบูชาพระรัตนตรัย}
\hrule
\textbf{โย โส ภะคะวา อะระหัง สัมมาสัมพุทโธ}\\
\indent พระผู้มีพระภาคเจ้านั้น พระองค์ใด, เป็นพระอรหันต์, \\
\indent ดับเพลิงกิเลสเพลิงทุกข์สิ้นเชิง, ตรัสรู้ชอบได้โดยพระองค์เอง\\
\textbf{ส๎วากขาโต เยนะ ภะคะวะตา ธัมโม}\\
\indent พระธรรม เป็นธรรมอันพระผู้มีพระภาคเจ้า พระองค์ใด, ตรัสไว้ดีแล้ว\\
\textbf{สุปะฏิปันโน ยัสสะ ภะคะวะโต สาวะกะสังโฆ}\\
\indent พระสงฆ์สาวกของพระผู้มีพระภาคเจ้า พระองค์ใด, ปฏิบัติดีแล้ว\\
\textbf{ตัมมะยัง ภะคะวันตัง สะธัมมัง สะสังฆัง\\
อิเมหิ สักกาเรหิ ยะถาระหัง อาโรปิเตหิ อะภิปูชะยามะ}\\
\indent ข้าพเจ้าทั้งหลาย, ขอบูชาอย่างยิ่งซึ่งพระผู้มีพระภาคเจ้าพระองค์นั้น,\\
\indent พร้อมทั้งพระธรรมและพระสงฆ์,\\
\indent ด้วยเครื่องสักการะทั้งหลายเหล่านี้, อันยกขึ้นตามสมควรแล้วอย่างไร\\
\textbf{สาธุ โน ภันเต ภะคะวา สุจิระปะรินิพพุโตปิ}\\
\indent ข้าแต่พระองค์ผู้เจริญ, พระผู้มีพระภาคเจ้าแม้ปรินิพพานนานแล้ว,\\
\indent ทรงสร้างคุณอันสำเร็จประโยชน์ไว้แก่ข้าพเจ้าทั้งหลาย.\\
\textbf{ปัจฉิมาชะนะตานุกัมปะมานะสา}\\
\indent ทรงมีพระหฤทัยอนุเคราะห์แก่พวกข้าพเจ้า อันเป็นชนรุ่นหลัง\\
\textbf{อิเม สักกาเร ทุคคะตะปัณณาการะภูเต ปะฏิคคัณหาตุ}\\
\indent ขอพระผู้มีพระภาคเจ้าจงรับเครื่องสักการะ อันเป็นบรรณาการของคน\\
\indent ยากทั้งหลายเหล่านี้\\
\textbf{อัมหากัง ทีฆะรัตตัง หิตายะ สุขายะ}\\
\indent เพื่อประโยชน์และความสุขแก่ข้าพเจ้าทั้งหลาย ตลอดกาลนาน เทอญฯ\\
\textbf{อะระหัง สัมมาสัมพุทโธ ภะคะวา}\\
\indent พระผู้มีพระภาคเจ้า, เป็นพระอรหันต์, ดับเพลิงกิเลสเพลิงทุกข์สิ้นเชิง,\\
\indent ตรัสรู้ชอบได้โดยพระองค์เอง\\
\textbf{พุทธัง ภะคะวันตัง อะภิวาเทมิ}\\
\indent ข้าพเจ้าอภิวาทพระผู้มีพระภาคเจ้า, ผู้รู้ ผู้ตื่น ผู้เบิกบาน (กราบ)\\
\textbf{ส๎วากขาโต ภะคะวะตา ธัมโม} \\
\indent พระธรรมเป็นธรรมที่พระผู้มีพระภาคเจ้า, ตรัสไว้ดีแล้ว\\
\textbf{ธัมมัง นะมัสสามิ}\\
\indent ข้าพเจ้านมัสการพระธรรม (กราบ)\\
\textbf{สุปะฏิปันโน ภะคะวะโต สาวะกะสังโฆ,}\\
\indent พระสงฆ์สาวกของพระผู้มีพระภาคเจ้า, ปฏิบัติดีแล้ว\\
\textbf{สังฆัง นะมามิ}\\
\indent ข้าพเจ้านอบน้อมพระสงฆ์ (กราบ)

\pagebreak
\section{ปุพพภาคนมการ}
\hrule
\begin{center}
\textbf{(หันทะ มะยัง พุทธัสสะ ภะคะวะโต ปุพพะภาคะนะมะการัง กะโรมะ เส)}\\
เชิญเถิด เราทั้งหลาย ทำความนอบน้อมอันเป็นส่วนเบื้องต้น แด่พระผู้มีพระภาคเจ้าเถิด
\end{center}
\textbf{นะโม ตัสสะ ภะคะวะโต,}\\
\indent ขอนอบน้อมแด่พระผู้มีพระภาคเจ้าพระองค์นั้น\\
\textbf{อะระหะโต,}\\
\indent ซึ่งเป็นผู้ไกลจากกิเลส\\
\textbf{สัมมาสัมพุทธัสสะ}\\
\indent ตรัสรู้ชอบได้โดยพระองค์เอง.	
\begin{center}
(กล่าว ๓ ครั้ง)
\end{center}

\pagebreak

\section{พุทธาภิถุติ}
\hrule
\begin{center}
\textbf{(หันทะ มะยัง พุทธาภิถุติง กะโรมะ เส)}\\
เชิญเถิด เราทั้งหลาย ทำความชมเชยเฉพาะพระพุทธเจ้าเถิด
\end{center}
\textbf{โย โส ตะถาคะโต\\}
\indent พระตถาคตเจ้านั้น พระองค์ใด\\
\textbf{อะระหัง\\}
\indent เป็นผู้ไกลจากกิเลส\\
\textbf{สัมมาสัมพุทโธ\\}
\indent เป็นผู้ตรัสรู้ชอบได้โดยพระองค์เอง\\
\textbf{วิชชาจะระณะสัมปันโน\\}
\indent เป็นผู้ถึงพร้อมด้วยวิชชาและจรณะ\\
\textbf{สุคะโต\\}
\indent เป็นผู้ไปแล้วด้วยดี\\
\textbf{โลกะวิท\\}
\indent เป็นผู้รู้โลกอย่างแจ่มแจ้ง\\
\textbf{อะนุตตะโร ปุริสะทัมมะสาระถิ\\}
\indent เป็นผู้สามารถฝึกบุรุษที่สมควรฝึกได้อย่างไม่มีใครยิ่งกว่า\\
\textbf{สัตถา เทวะมะนุสสานัง\\}
\indent เป็นครูผู้สอนของเทวดาและมนุษย์ทั้งหลาย\\
\textbf{พุทโธ\\}
\indent เป็นผู้รู้ ผู้ตื่น ผู้เบิกบานด้วยธรรม\\
\textbf{ภะคะวา\\}
\indent เป็นผู้มีความจำเริญ จำแนกธรรมสั่งสอนสัตว์\\
\textbf{โย อิมัง โลกัง สะเทวะกัง สะมาระกัง สะพ๎รัห๎มะกัง, \\
สัสสะมะณะพ๎ราห๎มะณิงปะชัง สะเทวะมะนุสสัง\\
 สะยัง อะภิญญา สัจฉิกัต๎วา ปะเวเทสิ\\}
\indent พระผู้มีพระภาคเจ้าพระองค์ใด, ได้ทรงทำความดับทุกข์ให้แจ้ง \\
\indent ด้วยพระปัญญาอันยิ่งเองแล้ว, ทรงสอนโลกนี้พร้อมทั้งเทวดา \\
\indent มาร พรหม และหมู่สัตว์ พร้อมทั้งสมณพราหมณ์\\
\indent พร้อมทั้งเทวดาและมนุษย์ให้รู้ตาม\\
\textbf{โย ธัมมัง เทเสสิ\\}
\indent พระผู้มีพระภาคเจ้าพระองค์ใด ทรงแสดงธรรมแล้ว\\
\textbf{อาทิกัล๎ยาณัง\\}
\indent ไพเราะในเบื้องต้น\\
\textbf{มัชเฌกัล๎ยาณัง\\}
\indent ไพเราะในท่ามกลาง\\
\textbf{ปะริโยสานะกัล๎ยาณัง\\}
\indent ไพเราะในที่สุด\\
\textbf{สาตถัง สะพ๎ยัญชะนัง เกวะละปะริปุณณัง ปะริสุทธัง พ๎รัห๎มะจะริยัง ปะกาเสสิ\\}
\indent ทรงประกาศพรหมจรรย์ คือแบบแห่งการปฏิบัติอันประเสริฐ บริสุทธิ์\\
\indent บริบูรณ์ สิ้นเชิง,พร้อมทั้งอรรถะ (คำอธิบาย) พร้อมทั้งพยัญชนะ (หัวข้อ)\\
\textbf{ตะมะหัง ภะคะวันตัง อะภิปูชะยามิ\\}
\indent ข้าพเจ้าบูชาอย่างยิ่ง เฉพาะพระผู้มีพระภาคเจ้าพระองค์นั้น\\
\textbf{ตะมะหัง ภะคะวันตัง สิระสา นะมามิ\\}
\indent ข้าพเจ้านอบน้อมพระผู้มีพระภาคเจ้า พระองค์นั้นด้วยเศียรเกล้า\\
\begin{center}
(กราบระลึกพระพุทธคุณ)
\end{center}
\pagebreak
\section{ธัมมาภิถุติ}
\hrule
\begin{center}
\textbf{(หันทะ มะยัง ธัมมาภิถุติง กะโรมะ เส)\\}
เชิญเถิด เราทั้งหลาย ทำความชมเชยเฉพาะพระธรรมเถิด
\end{center}
\textbf{โย โส ส๎วากขาโต ภะคะวะตา ธัมโม\\}
\indent พระธรรมนั้นใด, เป็นสิ่งที่พระผู้มีพระภาคเจ้าได้ตรัสไว้ดีแล้ว\\
\textbf{สันทิฏฐิโก\\}
\indent เป็นสิ่งที่ผู้ศึกษาและปฏิบัติพึงเห็นได้ด้วยตนเอง\\
\textbf{อะกาลิโก\\}
\indent เป็นสิ่งที่ปฏิบัติได้ และให้ผลได้ไม่จำกัดกาล\\
\textbf{เอหิปัสสิโก\\}
\indent เป็นสิ่งที่ควรกล่าวกะผู้อื่นว่า ท่านจงมาดูเถิด\\
\textbf{โอปะนะยิโก\\}
\indent เป็นสิ่งที่ควรน้อมเข้ามาใส่ตัว\\
\textbf{ปัจจัตตัง เวทิตัพโพ วิญญูหิ\\}
\indent เป็นสิ่งที่ผู้รู้ก็รู้ได้เฉพาะตน\\
\textbf{ตะมะหัง ธัมมัง อะภิปูชะยามิ\\}
\indent ข้าพเจ้าบูชาอย่างยิ่ง เฉพาะพระธรรมนั้น\\
\textbf{ตะมะหัง ธัมมัง สิระสา นะมามิ\\}
\indent ข้าพเจ้านอบน้อมพระธรรมนั้น ด้วยเศียรเกล้า
\begin{center}
(กราบระลึกพระธรรมคุณ)
\end{center}
\pagebreak
\section{สังฆาภิถุติ}
\hrule
\begin{center}
\textbf{(หันทะ มะยัง สังฆาภิถุติง กะโรมะ เส)\\}
เชิญเถิด เราทั้งหลาย ทำความชมเชยเฉพาะพระสงฆ์เถิด
\end{center}
\textbf{โย โส สุปะฏิปันโน ภะคะวะโต สาวะกะสังโฆ\\}
\indent สงฆ์สาวกของพระผู้มีพระภาคเจ้านั้นหมู่ใด ปฏิบัติดีแล้ว\\
\textbf{อุชุปะฏิปันโน ภะคะวะโต สาวะกะสังโฆ\\}
\indent สงฆ์สาวกของพระผู้มีพระภาคเจ้าหมู่ใด ปฏิบัติตรงแล้ว\\
\textbf{ญายะปะฏิปันโน ภะคะวะโต สาวะกะสังโฆ\\}
\indent สงฆ์สาวกของพระผู้มีพระภาคเจ้าหมู่ใด, ปฏิบัติเพื่อรู้ธรรมเป็นเครื่องออกจากทุกข์แล้ว\\
\textbf{สามีจิปะฏิปันโน ภะคะวะโต สาวะกะสังโฆ\\}
\indent สงฆ์สาวกของพระผู้มีพระภาคเจ้าหมู่ใด, ปฏิบัติสมควรแล้ว\\
\textbf{ยะทิทัง\\}
\indent ได้แก่บุคคลเหล่านี้คือ\\
\textbf{จัตตาริ ปุริสะยุคานิ อัฏฐะ ปุริสะปุคคะลา\\}
\indent คู่แห่งบุรุษ ๔ คู่, นับเรียงตัวบุรุษได้ ๘ บุรุษ*
\footnote{* สี่คู่คือ โสดาปัตติมรรค โสดาปัตติผล, สกิทาคามิมรรค สกิทาคามิผล,
อนาคามิมรรค อนาคามิผล, อรหัตตมรรค อรหัตตผล.}\\
\textbf{เอสะ ภะคะวะโต สาวะกะสังโฆ\\}
\indent นั่นแหละสงฆ์สาวกของพระผู้มีพระภาคเจ้า\\
\textbf{อาหุเนยโย\\}
\indent เป็นสงฆ์ควรแก่สักการะที่เขานำมาบูชา\\
\textbf{ปาหุเนยโย\\}
\indent เป็นสงฆ์ควรแก่สักการะที่เขาจัดไว้ต้อนรับ\\
\textbf{ทักขิเณยโย\\}
\indent เป็นผู้ควรรับทักษิณาทาน\\
\textbf{อัญชะลิกะระณีโย\\}
\indent เป็นผู้ที่บุคคลทั่วไปควรทำอัญชลี\\
\textbf{อะนุตตะรัง ปุญญักเขตตัง โลกัสสะ\\}
\indent เป็นเนื้อนาบุญของโลก, ไม่มีนาบุญอื่นยิ่งกว่า\\
\textbf{ตะมะหัง สังฆัง อะภิปูชะยามิ\\}
\indent ข้าพเจ้าบูชาอย่างยิ่ง เฉพาะพระสงฆ์หมู่นั้น\\
\textbf{ตะมะหัง สังฆัง สิระสา นะมามิ\\}
\indent ข้าพเจ้านอบน้อมพระสงฆ์หมู่นั้น ด้วยเศียรเกล้า\\
\begin{center}
(กราบระลึกพระสังฆคุณ)
\end{center}
\pagebreak

\section{รตนัตตยัปปณามคาถา}
\hrule
\begin{center}
\textbf{(หันทะ มะยัง ระตะนัตตะยัปปะณามะคาถาโย เจวะสังเวคะวัตถุปะริกิตตะนะปาฐัญจะ ภะณามะ เส)\\}
เชิญเถิด เราทั้งหลาย กล่าวคำนอบน้อมพระรัตนตรัยและบาลีที่กำหนดวัตถุเครื่องแสดงความสังเวชเถิด
\end{center}
\textbf{พุทโธ สุสุทโธ กะรุณามะหัณณะโว\\}
\indent พระพุทธเจ้าผู้บริสุทธิ์ มีพระกรุณาดุจห้วงมหรรณพ\\
\textbf{โยจจันตะสุทธัพพะระญาณะโลจะโน\\}
\indent พระองค์ใด มีตาคือญาณอันประเสริฐหมดจดถึงที่สุด\\
\textbf{โลกัสสะ ปาปูปะกิเลสะฆาตะโก\\}
\indent เป็นผู้ฆ่าเสียซึ่งบาปและอุปกิเลสของโลก\\
\textbf{วันทามิ พุทธัง อะหะมาทะเรนะ ตัง\\}
\indent ข้าพเจ้าไหว้พระพุทธเจ้าพระองค์นั้น โดยใจเคารพเอื้อเฟื้อ\\
\textbf{ธัมโม ปะทีโป วิยะ ตัสสะ สัตถุโน\\}
\indent พระธรรมของพระศาสดา สว่างรุ่งเรืองเปรียบดวงประทีป\\
\textbf{โย มัคคะปากามะตะเภทะภินนะโก\\}
\indent จำแนกประเภท คือ มรรค ผล นิพพาน, ส่วนใด\\
\textbf{โลกุตตะโร โย จะ ตะทัตถะทีปะโน\\}
\indent ซึ่งเป็นตัวโลกุตตระ, และส่วนใดที่ชี้แนวแห่งโลกุตตระนั้น\\
\textbf{วันทามิ ธัมมัง อะหะมาทะเรนะ ตัง\\}
\indent ข้าพเจ้าไหว้พระธรรมนั้น โดยใจเคารพเอื้อเฟื้อ\\
\textbf{สังโฆ สุเขตตาภ๎ยะติเขตตะสัญญิโต\\}
\indent พระสงฆ์เป็นนาบุญอันยิ่งใหญ่กว่านาบุญอันดีทั้งหลาย\\
\textbf{โย ทิฏฐะสันโต สุคะตานุโพธะโก\\}
\indent เป็นผู้เห็นพระนิพพาน, ตรัสรู้ตามพระสุคต, หมู่ใด\\
\textbf{โลลัปปะหี\footnote{อ่านว่า ฮี} โน อะริโย สุเมธะโส\\}
\indent เป็นผู้ละกิเลสเครื่องโลเลเป็นพระอริยเจ้า มีปัญญาดี\\
\textbf{วันทามิ สังฆัง อะหะมาทะเรนะ ตัง\\}
\indent ข้าพเจ้าไหว้พระสงฆ์หมู่นั้นโดยใจเคารพเอื้อเฟื้อ\\
\textbf{อิจเจวะเมกันตะภิปูชะเนยยะกัง, วัตถุตตะยัง วันทะยะตาภิสังขะตัง,\\
ปุญญัง มะยา ยัง มะมะ สัพพุปัททะวา,มา โหนตุ เว ตัสสะ ปะภาวะสิทธิยา\\}
\indent บุญใดที่ข้าพเจ้าผู้ไหว้อยู่ซึ่งวัตถุสาม, คือพระรัตนตรัยอันควรบูชายิ่งโดยส่วนเดียว,\\
\indent ได้กระทำแล้วเป็นอย่างยิ่งเช่นนี้, ขออุปัทวะ(ความชั่ว) ทั้งหลาย,\\
\indent จงอย่ามีแก่ข้าพเจ้าเลย, ด้วยอำนาจความสำเร็จอันเกิดจากบุญนั้น\\
\pagebreak
\section{สังเวคปริกิตตนปาฐะ}
\hrule
\textbf{อิธะ ตะถาคะโต โลเก อุปปันโน\\}
\indent พระตถาคตเจ้าเกิดขึ้นแล้วในโลกนี้\\
\textbf{อะระหัง สัมมาสัมพุทโธ\\}
\indent เป็นผู้ไกลจากกิเลส ตรัสรู้ชอบได้โดยพระองค์เอง\\
\textbf{ธัมโม จะ เทสิโต นิยยานิโก\\}
\indent และพระธรรมที่ทรงแสดงเป็นธรรมเครื่องออกจากทุกข์\\
\textbf{อุปะสะมิโก ปะรินิพพานิโก\\}
\indent เป็นเครื่องสงบกิเลส, เป็นไปเพื่อปรินิพพาน\\
\textbf{สัมโพธะคามี สุคะตัปปะเวทิโต\\}
\indent เป็นไปเพื่อความรู้พร้อม, เป็นธรรมที่พระสุคตประกาศ\\
\textbf{มะยันตัง ธัมมัง สุต๎วา เอวัง ชานามะ\\}
\indent พวกเราเมื่อได้ฟังธรรมนั้นแล้ว, จึงได้รู้อย่างนี้ว่า\\
\textbf{ชาติปิ ทุกขา\\}
\indent แม้ความเกิดก็เป็นทุกข์\\
\textbf{ชะราปิ ทุกขา\\}
\indent แม้ความแก่ก็เป็นทุกข์\\
\textbf{มะระณัมปิ ทุกขัง\\}
\indent แม้ความตายก็เป็นทุกข์\\
\textbf{โสกะปะริเทวะทุกขะโทมะนัสสุปายาสาปิ ทุกขา\\}
\indent แม้ความโศก ความร่ำไรรำพัน ความไม่สบายกาย\\
\indent ความไม่สบายใจ ความคับแค้นใจ ก็เป็นทุกข์\\
\textbf{อัปปิเยหิ สัมปะโยโค ทุกโข\\}
\indent ความประสบกับสิ่งไม่เป็นที่รักที่พอใจ ก็เป็นทุกข์\\
\textbf{ปิเยหิ วิปปะโยโค ทุกโข\\}
\indent ความพลัดพรากจากสิ่งเป็นที่รักที่พอใจ ก็เป็นทุกข์\\
\textbf{ยัมปิจฉัง นะ ละภะติ ตัมปิ ทุกขัง\\}
\indent มีความปรารถนาสิ่งใดไม่ได้สิ่งนั้นนั่นก็เป็นทุกข์\\
\textbf{สังขิตเตนะ ปัญจุปาทานักขันธา ทุกขา\\}
\indent ว่าโดยย่อ อุปาทานขันธ์ทั้ง ๕ เป็นตัวทุกข์\\
\textbf{เสยยะถีทัง\\}
\indent ได้แก่สิ่งเหล่านี้คือ\\
\textbf{รูปูปาทานักขันโธ\\}
\indent ขันธ์ อันเป็นที่ตั้งแห่งความยึดมั่น คือรูป\\
\textbf{เวทะนูปาทานักขันโธ\\}
\indent ขันธ์ อันเป็นที่ตั้งแห่งความยึดมั่น คือเวทนา\\
\textbf{สัญญูปาทานักขันโธ\\}
\indent ขันธ์ อันเป็นที่ตั้งแห่งความยึดมั่น คือสัญญา\\
\textbf{สังขารูปาทานักขันโธ\\}
\indent ขันธ์ อันเป็นที่ตั้งแห่งความยึดมั่น คือสังขาร\\
\textbf{วิญญาณูปาทานักขันโธ\\}
\indent ขันธ์ อันเป็นที่ตั้งแห่งความยึดมั่น คือวิญญาณ\\
\textbf{เยสัง ปะริญญายะ\\}
\indent เพื่อให้สาวกกำหนดรอบรู้อุปาทานขันธ์เหล่านี้เอง\\
\textbf{ธะระมาโน โส ภะคะวา\\}
\indent จึงพระผู้มีพระภาคเจ้านั้น เมื่อยังทรงพระชนม์อยู่\\
\textbf{เอวัง พะหุลัง สาวะเก วิเนติ\\}
\indent ย่อมทรงแนะนำสาวกทั้งหลาย, เช่นนี้เป็นส่วนมาก\\
\textbf{เอวัง ภาคา จะ ปะนัสสะ ภะคะวะโต สาวะเกสุ อะนุสาสะนี พะหุลา ปะวัตตะติ\\}
\indent อนึ่งคำสั่งสอนของพระผู้มีพระภาคเจ้านั้นย่อมเป็นไปในสาวกทั้งหลาย,\\
\indent ส่วนมากมีส่วนคือการจำแนกอย่างนี้ว่า\\
\textbf{รูปัง อะนิจจัง\\}
\indent รูปไม่เที่ยง\\
\textbf{เวทะนา อะนิจจา\\}
\indent เวทนาไม่เที่ยง\\
\textbf{สัญญา อะนิจจา\\}
\indent สัญญาไม่เที่ยง\\
\textbf{สังขารา อะนิจจา\\}
\indent สังขารไม่เที่ยง\\
\textbf{วิญญาณัง อะนิจจัง\\}
\indent วิญญาณไม่เที่ยง\\
\textbf{รูปัง อะนัตตา\\}
\indent รูปไม่ใช่ตัวตน\\
\textbf{เวทะนา อะนัตตา\\}
\indent เวทนาไม่ใช่ตัวตน\\
\textbf{สัญญา อะนัตตา\\}
\indent สัญญาไม่ใช่ตัวตน\\
\textbf{สังขารา อะนัตตา\\}
\indent สังขารไม่ใช่ตัวตน\\
\textbf{วิญญาณัง อะนัตตา\\}
\indent วิญญาณไม่ใช่ตัวตน\\
\textbf{สัพเพ สังขารา อะนิจจา\\}
\indent สังขารทั้งหลายทั้งปวง ไม่เที่ยง\\
\textbf{สัพเพ ธัมมา อะนัตตาติ\\}
\indent ธรรมทั้งหลายทั้งปวง ไม่ใช่ตัวตน, ดังนี้\\
\textbf{เต (ตา)  มะยัง โอติณณามหะ\\}
\indent พวกเราทั้งหลายเป็นผู้ถูกครอบงำแล้ว\\
\textbf{ชาติยา\\}
\indent โดยความเกิด\\
\textbf{ชะรามะระเณนะ\\}
\indent โดยความแก่และความตาย\\
\textbf{โสเกหิ ปะริเทเวหิ ทุกเขหิ โทมะนัสเสหิ อุปายาเสหิ\\}
\indent โดยความโศก ความร่ำไรรำพัน ความไม่สบายกาย\\
\indent ความไม่สบายใจ ความคับแค้นใจทั้งหลาย\\
\textbf{ทุกโขติณณา\\}
\indent เป็นผู้ถูกความทุกข์หยั่งเอาแล้ว\\
\textbf{ทุกขะปะเรตา\\}
\indent เป็นผู้มีความทุกข์เป็นเบื้องหน้าแล้ว\\
\textbf{อัปเปวะนามิมัสสะ เกวะลัสสะ ทุกขักขันธัสสะ อันตะกิริยา ปัญญาเยถาติ\\}
\indent ทำไฉนการทำที่สุดแห่งกองทุกข์ทั้งสิ้นนี้, จะพึงปรากฏชัดแก่เราได้\\
\begin{center}\textbf{สำหรับ พระภิกษุ - สามเณรสวด}\end{center}
\textbf{จิระปะรินิพพุตัมปิ ตัง ภะคะวันตัง อุททิสสะ อะระหันตัง สัมมาสัมพุทธัง\\}
\indent เราทั้งหลาย อุทิศเฉพาะพระผู้มีพระภาคเจ้า, ผู้ไกลจากกิเลส\\
\indent ตรัสรู้ชอบได้โดยพระองค์เอง, แม้ปรินิพพานนานแล้ว, พระองค์นั้น\\
\textbf{สัทธา อะคารัส๎มา อะนะคาริยัง ปัพพะชิตา\\}
\indent เป็นผู้มีศรัทธา ออกบวชจากเรือน ไม่เกี่ยวข้องด้วยเรือนแล้ว\\
\textbf{ตัส๎มิง ภะคะวะติ พ๎รห๎มะจะริยัง จะรามะ\\}
\indent ประพฤติอยู่ซึ่งพรหมจรรย์ ในพระผู้มีพระภาคเจ้า พระองค์นั้น\\
\textbf{ภิกขูนัง สิกขาสาชีวะสะมาปันนา\\}
\indent ถึงพร้อมด้วยสิกขาและธรรมเป็นเครื่องเลี้ยงชีวิตของภิกษุทั้งหลาย\\
\textbf{ตัง โน พ๎รห๎มะจะริยัง อิมัสสะ เกวะลัสสะ ทุกขักขันธัสสะ อันตะ กิริยายะ สังวัตตะตุ\\}
\indent ขอให้พรหมจรรย์ของเราทั้งหลายนั้น, จงเป็นไปเพื่อการทำที่สุดแห่งกอง ทุกข์ทั้งสิ้นนี้ เทอญ\\
\begin{center}\textbf{สำหรับอุบาสก, อุบาสิกา}\end{center}
\textbf{จิระปะรินิพพุตัมปิ ตัง ภะคะวันตัง สะระณังคะตา\\}
\indent เราทั้งหลาย ผู้ถึงแล้วซึ่งพระผู้มีพระภาคเจ้า,\\
\indent แม้ปรินิพพานนานแล้วพระองค์นั้น เป็นสรณะ\\
\textbf{ธัมมัญจะ สังฆัญจะ}\\
\indent ถึงพระธรรมด้วย ถึงพระสงฆ์ด้วย\\
\textbf{ตัสสะ ภะคะวะโต สาสะนัง, ยะถาสะติ, ยะถาพะลัง\\
มะนะสิกะโรมะ อะนุปะฏิปัชชามะ}\\
\indent จักทำในใจอยู่ ปฏิบัติตามอยู่ ซึ่งคำสั่งสอนของพระผู้มีพระภาคเจ้านั้น\\
\indent ตามสติกำลัง\\
\textbf{สา สา โน ปะฏิปัตติ}\\
\indent ขอให้ความปฏิบัตินั้นๆ ของเราทั้งหลาย\\
\textbf{อิมัสสะ เกวะลัสสะ ทุกขักขันธัสสะ อันตะกิริยายะ สังวัตตะตุ}\\
\indent จงเป็นไปเพื่อการทำที่สุดแห่งกองทุกข์ทั้งสิ้นนี้ เทอญ\\
\begin{center}
(จบคำทำวัตรเช้า)
\end{center}
\pagebreak
\section{คำแผ่เมตตา}
\hrule
\begin{center}(หันทะ มะยัง เมตตาผะระณัง กะโรมะ เส)\end{center}
\textbf{อะหัง สุขิโต โหมิ\\}
\indent ขอให้ข้าพเจ้าจงเป็นผู้ถึงสุข\\
\textbf{นิททุกโข โหมิ\\}
\indent จงเป็นผู้ไร้ทุกข์\\
\textbf{อะเวโร โหมิ\\}
\indent จงเป็นผู้ไม่มีเวร\\
\textbf{อัพฺยาปัชโฌ โหมิ \\}
\indent จงเป็นผู้ไม่เบียดเบียนซึ่งกันและกัน\\
\textbf{อะนีโฆ โหมิ\\}
\indent จงเป็นผู้ไม่มีทุกข์\\
\textbf{สุขี อัตตานัง ปะริหะรามิ\\}
\indent จงรักษาตนอยู่เป็นสุขเถิด\\
\textbf{สัพเพ สัตตา สุขิตา โหนตุ\\}
\indent ขอสัตว์ทั้งหลายทั้งปวงจงเป็นผู้ถึงความสุข\\
\textbf{สัพเพ สัตตา อะเวรา โหนตุ\\}
\indent ขอสัตว์ทั้งหลายทั้งปวงจงเป็นผู้ไม่มีเวร\\
\textbf{สัพเพ สัตตา อัพฺยาปัชฌา โหนตุ\\}
\indent ขอสัตว์ทั้งหลายทั้งปวงจงอย่าได้เบียดเบียนซึ่งกันและกัน\\
\textbf{สัพเพ สัตตา อะนีฆา โหนตุ\\}
\indent ขอสัตว์ทั้งหลายทั้งปวงจงเป็นผู้ไม่มีทุกข์\\
\textbf{สัพเพ สัตตา สุขี อัตตานัง ปะริหะรันตุ\\}
\indent ขอสัตว์ทั้งหลายทั้งปวงจงรักษาตนอยู่เป็นสุขเถิด\\
\textbf{สัพเพ สัตตา สัพพะทุกขา ปะมุญจันตุ\\}
\indent ขอสัตว์ทั้งหลายทั้งปวงจงพ้นจากทุกข์ทั้งมวล\\
\textbf{สัพเพ สัตตา ลัทธะสัมปัตติโต มา วิคัจฉันตุ\\}
\indent ขอสัตว์ทั้งหลายทั้งปวงจงอย่าได้พรากจากสมบัติอันตนได้แล้ว\\
\textbf{สัพเพ สัตตา กัมมัสสะกา กัมมะทายาทา\\
กัมมะโยนิ กัมมะพันธุ กัมมะปะฏิสะระณา\\}
\indent สัตว์ทั้งหลายทั้งปวงมีกรรมเป็นของของตน, มีกรรมเป็นผู้ให้ผล,\\
\indent มีกรรมเป็นแดนเกิด, มีกรรมเป็นผู้ติดตาม,มีกรรมเป็นที่พึ่งอาศ้ย\\
\textbf{ยัง กัมมัง กะรัสสันติ, กัลฺยาณัง วา ปาปะกัง วา, ตัสสะ ทายาทา ภาวิสสันติ\\}
\indent จักทำกรรมอันใดไว้, เป็นบุญหรือเป็นบาป, จักต้องเป็นผู้ได้รับผลกรรมนั้นๆ สืบไป

\pagebreak
\section{ท๎วัตติงสาการปาฐะ}
\hrule
\begin{center}
\textbf{(หันทะ มะยัง ท๎วัตติงสาการะปาฐัง ภะณามะ เส)\\}
เชิญเถิด เราทั้งหลาย จงกล่าวคาถาแสดงอาการ ๓๒ ในร่างกายเถิด
\end{center}
\textbf{อะยัง โข เม กาโย\\}
\indent กายของเรานี้แล\\
\textbf{อุทธัง ปาทะตะลา\\}
\indent เบื้องบนแต่พื้นเท้าขึ้นมา\\
\textbf{อะโธ เกสะมัตถะกา\\}
\indent เบื้องต่ำแต่ปลายผมลงไป\\
\textbf{ตะจะปะริยันโต\\}
\indent มีหนังหุ้มอยู่เป็นที่สุดรอบ\\
\textbf{ปูโรนานัปปะการัสสะ อะสุจิโน\\}
\indent เต็มไปด้วยของไม่สะอาด มีประการต่างๆ\\
\textbf{อัตถิ อิมัสมิง กาเย \\}
\indent มีอยู่ในกายนี้\\
\textbf{เกสา \\}
\indent คือผมทั้งหลาย\\
\textbf{โลมา} \\
\indent คือขนทั้งหลาย\\
\textbf{นะขา} \\
\indent คือเล็บทั้งหลาย\\
\textbf{ทันตา} \\
\indent คือฟันทั้งหลาย\\
\textbf{ตะโจ} \\
\indent หนัง\\
\textbf{มังสัง} \\
\indent เนื้อ\\
\textbf{นะหารู} \\
\indent เอ็นทั้งหลาย\\
\textbf{อัฏฐิ} \\
\indent กระดูกทั้งหลาย\\
\textbf{อัฏฐิมิญชัง} \\
\indent เยื่อในกระดูก\\
\textbf{วักกัง} \\
\indent ไต\\
\textbf{หะทะยัง} \\
\indent หัวใจ\\
\textbf{ยะกะนัง} \\
\indent ตับ\\
\textbf{กิโลมะกัง} \\
\indent พังผืด\\
\textbf{ปิหะกัง} \\
\indent ม้าม\\
\textbf{ปัปผาสัง} \\
\indent ปอด\\
\textbf{อันตัง} \\
\indent ไส้ใหญ่\\
\textbf{อันตะคุณัง} \\
\indent สายรัดไส้\\
\textbf{อุทะริยัง} \\
\indent อาหารใหม่\\
\textbf{กะรีสัง} \\
\indent อาหารเก่า\\
\textbf{ปิตตัง} \\
\indent น้ำดี\\
\textbf{เสมหัง} \\
\indent น้ำเสลด\\
\textbf{ปุพโพ} \\
\indent น้ำเหลือง\\
\textbf{โลหิตัง} \\
\indent น้ำเลือด\\
\textbf{เสโท} \\
\indent น้ำเหงื่อ\\
\textbf{เมโท} \\
\indent น้ำมันข้น\\
\textbf{อัสสุ} \\
\indent น้ำตา\\
\textbf{วะสา} \\
\indent น้ำมันเหลว\\
\textbf{เขโฬ} \\
\indent น้ำลาย\\
\textbf{สิงฆาณิกา} \\
\indent น้ำมูก\\
\textbf{ละสิกา} \\
\indent น้ำมันไขข้อ\\
\textbf{มุตตัง} \\
\indent น้ำมูตร\\
\textbf{มัตถะเกมัตถะลุงคัง} \\
\indent เยื่อในสมอง\\
\textbf{เอวะมะยังเม กาโย} \\
\indent กายของเรานี้อย่างนี้\\
\textbf{อุทธังปาทะตะลา} \\
\indent เบื้องบนแต่พื้นเท้าขึ้นมา\\
\textbf{อะโธเกสะมัตถะกา}\\
\indent เบื้องต่ำแต่ปลายผมลงไป\\
\textbf{ตะจะปะริยันโต} \\
\indent มีหนังหุ้มอยู่เป็นที่สุดรอบ\\
\textbf{ปูโรนานัปปะการัสสะ อะสุจิโน}\\
\indent เต็มไปด้วยของไม่สะอาด มีประการต่างๆ อย่างนี้แลฯ\\

\pagebreak
\section{อภิณหปัจจเวกขณปาฐะ}
\hrule
\begin{center}
\textbf{(หันทะ มะยัง อะภิณหะปัจจะเวกขะณะปาฐัง ภะณามะ เส)}\\
เชิญเถิดเราทั้งหลาย มาสวดอภิณหปัจจเวกขณะปาฐะกันเถิด
\end{center}
\textbf{ชะราธัมโมมหิ ชะรัง อะนะตีโต (อะนะตีตา) \\}
\indent เรามีความแก่เป็นธรรมดาจะล่วงพ้นความแก่ไปไม่ได้\\
\textbf{พ๎ยาธิธัมโมมหิ พ๎ยาธิง อะนะตีโต (อะนะตีตา) } \\
\indent เรามีความเจ็บไข้เป็นธรรมดา จะล่วงพ้นความเจ็บไข้ไปไม่ได้\\
\textbf{มะระณะธัมโมมหิ มะระณัง อะนะตีโต (อะนะตีตา)} \\
\indent เรามีความตายเป็นธรรมดาจะล่วงพ้นความตายไปไม่ได้\\
\textbf{สัพเพหิ เม ปิเยหิ มะนาเปหิ นานาภาโว วินาภาโว\\}
\indent เราจะละเว้นเป็นต่างๆ, คือว่าจะต้องพลัดพรากจากของรักของเจริญใจ ทั้งหลายทั้งปวง\\
\textbf{กัมมัสสะโกมหิ กัมมะทายาโท กัมมะโยนิ\\
กัมมะพันธุ กัมมะปะฏิสะระโณ (ณา)  \\}
\indent เราเป็นผู้มีกรรมเป็นของๆตน, มีกรรมเป็นผู้ให้ผล,\\
\indent มีกรรมเป็นแดนเกิด, มีกรรมเป็นผู้ติดตาม, มีกรรมเป็นที่พึ่งอาศัย\\
\textbf{ยัง กัมมัง กะริสสามิ, กัลยาณัง วา ปาปะกัง วา,\\
ตัสสะ ทายาโท (ทา)  ภะวิสสาม\\}
\indent เราทำกรรมอันใดไว้, เป็นบุญหรือเป็นบาป,\\
\indent เราจะเป็นทายาท, คือว่าเราจะต้องได้รับผลของกรรมนั้นๆ สืบไป\\
\textbf{เอวัง อัมเหหิ อะภิณหัง ปัจจะเวกขิตัพพัง\\}
\indent เราทั้งหลายควรพิจารณาอย่างนี้ทุกวันๆ เถิด

\pagebreak
\section{บทพิจารณาสังขาร }
\hrule
\begin{center}
\textbf{ (หันทะ มะยัง ธัมมะสังเวคะปัจจะเวกขะณะปาฐัง ภะณามะ เส)\\}
เชิญเถิด เราทั้งหลาย จงกล่าวคาถาพิจารณาธรรมสังเวชเถิด
\end{center}
\textbf{สัพเพ สังขารา อะนิจจา\\}
\indent สังขารคือร่างกายจิตใจ, แลรูปธรรมนามธรรมทั้งหมดทั้งสิ้น,\\
\indent มันไม่เที่ยง, เกิดขึ้นแล้วดับไปมีแล้วหายไป\\
\textbf{สัพเพ สังขารา ทุกขา\\}
\indent สังขารคือร่างกายจิตใจ, แลรูปธรรมนามธรรมทั้งหมดทั้งสิ้น,\\
\indent มันเป็นทุกข์ทนยาก, เพราะเกิดขึ้นแล้วแก่เจ็บตายไป\\
\textbf{สัพเพ ธัมมา อะนัตตา\\}
\indent สิ่งทั้งหลายทั้งปวง, ทั้งที่เป็นสังขารแลมิใช่สังขารทั้งหมดทั้งสิ้น,\\
\indent ไม่ใช่ตัวไม่ใช่ตน, ไม่ควรถือว่าเราว่าของเราว่าตัวว่าตนของเรา\\
\textbf{อะธุวัง ชีวิตัง\\}
\indent ชีวิตเป็นของไม่ยั่งยืน\\
\textbf{ธุวัง มะระณัง\\}
\indent ความตายเป็นของยั่งยืน\\
\textbf{อะวัสสัง มะยา มะริตัพพัง\\}
\indent อันเราจะพึงตายเป็นแท้\\
\textbf{มะระณะปะริโยสานัง เม ชีวิตัง\\}
\indent ชีวิตของเรามีความตายเป็นที่สุดรอบ\\
\textbf{ชีวิตัง เม อะนิยะตัง\\}
\indent ชีวิตของเราเป็นของไม่เที่ยง\\
\textbf{มะระณัง เม นิยะตัง\\}
\indent ความตายของเราเป็นของเที่ยง\\
\textbf{วะตะ\\}
\indent ควรที่จะสังเวช\\
\textbf{อะยัง กาโย อะจิรัง\\}
\indent ร่างกายนี้มิได้ตั้งอยู่นาน\\
\textbf{อะเปตะวิญญาโณ\\}
\indent ครั้นปราศจากวิญญาณ\\
\textbf{ฉุฑโฑ\\}
\indent อันเขาทิ้งเสียแล้ว\\
\textbf{อธิเสสสะติ\\}
\indent จักนอนทับ\\
\textbf{ปะฐะวิง\\}
\indent ซึ่งแผ่นดิน\\
\textbf{กะลิงคะรัง อิวะ\\}
\indent ประดุจดังว่าท่อนไม้และท่อนฟืน\\
\textbf{นิรัตถัง\\}
\indent หาประโยชน์มิได้\\
\textbf{อะนิจจา วะตะ สังขารา\\}
\indent สังขารทั้งหลายไม่เที่ยงหนอ\\
\textbf{อุปปาทะวะยะธัมมิโน\\}
\indent มีความเกิดขึ้นแล้วมีความเสื่อมไปเป็นธรรมดา\\
\textbf{อุปปัชชิตฺวา นิรุชฌันติ\\}
\indent ครั้นเกิดขึ้นแล้วย่อมดับไป\\
\textbf{เตสัง วูปะสะโม สุโข\\}
\indent ความเข้าไปสงบระงับสังขารทั้งหลาย, เป็นสุขอย่างยิ่ง, ดังนี้\\

\pagebreak
\section{ตังขณิกปัจจเวกขณปาฐะ}
\hrule
\begin{center}
\textbf{(นำ) หันทะ มะยัง ตังขณิกกะปัจจะเวกขณะปาฐัง ภะณามะ เส}
\end{center}
\textbf{ปะฏิสังขา โยนิโส จีวะรัง ปะฏิเสวามิ\\}
\indent เราย่อมพิจาราโดยแยบคายแล้วนุ่งห่มจีวร\\
\textbf{ยาวะเทวะสีตัสสะปฏิฆาตายะ\\}
\indent เพียงเพื่อบำบัดความหนาว\\
\textbf{อุณหัสสะ ปฏิฆาตายะ\\}
\indent เพื่อบำบัดความร้อน\\
\textbf{ฑังสะมะกะสะวาตาตะปะสิริงสะปะสัมผัสสานังปฏิฆาตายะ\\}
\indent เพื่อบำบัดสัมผัสอันเกิดจากเหลือบ ยุง ลม แดดและสัตว์เลื้อยคลานทั้งหลาย\\
\textbf{ยาวะเทวะ หิริโกปินนะปะฏิจฉาทะนัตถัง\\}
\indent และเพียงเพื่อปกปิดอวัยวะอันให้เกิดจากความละอาย\\
\textbf{ปฏิสังขา โยนิโส ปิณฑะปาตัง ปะฏิเสวามิ\\}
\indent เราย่อมพิจารณาโดยแยบคายแล้วฉันบิณฑบาต\\
\textbf{เนวะ ทะวายะ\\}
\indent ไม่ให้เป็นไปเพื่อความเพลิดเพลินสนุกสนาน\\
\textbf{นะ มะทายะ\\}
\indent ไม่ให้เป็นไปเพื่อความเมามันเกิดกำลังพลังทางกาย\\
\textbf{นะ มัณฑะนายะ\\}
\indent ไม่ให้เป็นไปเพื่อประดับ\\
\textbf{นะ วิภูสะนายะ\\}
\indent ไม่ให้เป็นไปพื่อตกแต่ง\\
\textbf{ยะวะเทวะ อิมัสสะ กายัสสะ ฐิติยา\\}
\indent แต่ให้เป็นไปเพียงเพื่อความตั้งอยู่ได้แห่งกายนี้\\
\textbf{ยาปะนายะ\\}
\indent เพื่อความเป็นไปได้ของอัตภาพ\\
\textbf{วิหิงสุปะระติยา\\}
\indent เพื่อความสิ้นไปแห่ความลำบากทางกาย\\
\textbf{พรัหมะจะริยานุคคะหายะ\\}
\indent เพื่ออนุเคราะห์แก่การประพฤติพรหมจรรย์\\
\textbf{อิติ ปุราณัญจะ เวทะนัง ปะฏิหังขามิ\\}
\indent ด้วยการทำอย่างนี้ เราย่อมระงับเสียได้ซึ่งทุกขเวทนาเก่า คือความหิว\\
\textbf{นะวัญจะเวทะนัง นะ อุปปาเทสสามิ\\}
\indent และไม่ทำทุกขเวทนาใหม่ให้เกิดขึ้น\\
\textbf{ยาตฺรา จะ เม ภะวิสสะติ อะนะวัชชะตา จะ ผาสุวิหาโร จาติ\\}
\indent อนึ่งความเป็นไปโดยสะดวกแห่งอัตภาพนี้ด้วย ความเป็นผู้หาโทษมิได้ด้วย\\ 
\indent และความเป็นอยู่โดยผาสุขด้วย จักมีแก่เรา ดังนี้\\
\textbf{ปะฏิสังขา โยนิโส เสนาสะนังปะฏิเสวามิ\\}
\indent เราย่อมพิจารณาโดยแยบคายแล้วใช้สอยเสนาสนะ
\textbf{ยาวะเทวะสีตัสสะ ปะฏิฆาตายะ}\\
\indent เพียงเพื่อบำบัดความหนาว\\
\textbf{อุณหัสสะปะฏิฆาตายะ}\\
\indent เพื่อบำบัดความร้อน\\
\textbf{ฑังสะมะกะสะวาตาตะปะสิริงสะปะสัมผัสสานัง ปะฏิฆาตายะ}\\
\indent เพื่อบำบัดสัมผัสอันเกิดจาก เหลือบยุงลมแดดและสัตว์เลื้อยคลานทั้งหลาย\\
\textbf{ยะวะเทวะ อุตุปะริสสะยะวิโนทะนัง ปะฏิสัลลานารามัตถัง}\\
\indent เพียงเพื่อบรรเทาอันตรายอันจะพึงมีจากดินฟ้าอากาศ\\ 
\indent และเพื่อความเป็นผู้ยินดีอยู่ได้ในที่หลีกเร้นสำหรับภาวนา\\
\textbf{ปะฏิสังขา โยนิโส คิลานะปัจจะยะเภสัชชปะริกขารัง ปะฏิเสวามิ}\\
\indent เราย่อมพิจาณาโยแยบคายแล้วบริโภคเภสัชบริชารอันเกื้อกูลแก่คนไข้\\
\textbf{ยะวะเวทวะ อุปปันนานัง เวยยาพาธิกานัง เวทะนานัง ปะฏิฆาตายะ}\\
\indent เพียงเพื่อบำบัดทุกขเวทนาอันบังเกิดขึ้นแล้ว มีอาพาธต่าง ๆ เป็นมูล\\
\textbf{อัพฺยาปัชณะปะระมะตายาติ}\\
\indent เพื่อความเป็นผู้ไม่มีโรคเบียดเบียนเป็นอย่างยิ่ง ดังนี้\\

\pagebreak
\section{สัพพปัตติทานคาถา}
\hrule
\begin{center}
\textbf{(หันทะ มะยัง สัพพะปัตติทานะคาถาโย ภะณามะ เส)}\\
เชิญเถิด เราทั้งหลาย จงกล่าวคำแผ่ส่วนบุญให้แก่สรรพสัตว์ทั้งหลายเถิด
\end{center}
\textbf{ปุญญัสสิทานิ กะตัสสะ ยานัญญานิ กะตานิ เม,\\
เตสัญจะ ภาคิโน โหนตุ สัตตานันตาปปะมาณะกา}\\
\indent สัตว์ทั้งหลาย ไม่มีที่สุด ไม่มีประมาณ, จงมีส่วนแห่งบุญที่ข้าพเจ้าได้ทำในบัดนี้,\\
\indent และแห่งบุญอื่นที่ได้ทำไว้ก่อนแล้ว \\
\textbf{เย ปิยา คุณะวันตา จะ มัยหัง มาตาปิตาทะโย,\\
ทิฏฐา เม จาป๎ยะทิฏฐา วา อัญเญ มัชฌัตตะเวริโน}\\
\indent คือจะเป็นสัตว์เหล่าใด, ซึ่งเป็นที่รักใคร่และมีบุญคุณ,\\
\indent เช่นมารดาบิดาของข้าพเจ้าเป็นต้น ก็ดี ที่ข้าพเจ้าเห็นแล้วหรือไม่ได้เห็น \\
\indent ก็ดี, สัตว์เหล่าอื่นที่เป็นกลางๆหรือเป็นคู่เวรกัน ก็ดี\\
\textbf{สัตตา ติฏฐันติ โลกัส๎มิง เตภุมมา จะตุโยนิกา,\\
ปัญเจกะจะตุโวการา สังสะรันตา ภะวาภะเว}\\
\indent สัตว์ทั้งหลายตั้งอยู่ในโลก, อยู่ในภูมิทั้งสาม, อยู่ในกำเนิดทั้งสี่,\\
\indent มีขันธ์ห้าขันธ์ มีขันธ์ขันธ์เดียว มีขันธ์สี่ขันธ์, กำลังท่องเที่ยวอยู่ในภพ\\
\indent น้อยภพใหญ่ ก็ดี\\
\textbf{ญาตัง เย ปัตติทานัมเม อะนุโมทันตุ เต สะยัง,\\
เย จิมัง นัปปะชานันติ เทวา เตสัง นิเวทะยุง,}\\ 
\indent \m สัตว์เหล่าใด รู้ส่วนบุญที่ข้าพเจ้าแผ่ให้แล้ว, สัตว์เหล่านั้นจงอนุโมทนาเองเถิด,\\
\indent ส่วนสัตว์เหล่าใดยังไม่รู้ส่วนบุญนี้, ขอเทวดาทั้งหลาย, จงบอกสัตว์เหล่านั้นให้รู้\\
\indent มะยา ทินนานะ ปุญญานัง อะนุโมทะนะเหตุนา,\\
\textbf{สัพเพ สัตตา สะทา โหนตุ อะเวรา สุขะชีวิโน,\\
เขมัปปะทัญจะ ปัปโปนตุ เตสาสา สิชฌะตัง สุภา}\\
\indent เพราะเหตุที่ได้อนุโมทนาส่วนบุญที่ข้าพเจ้าแผ่ให้แล้ว,\\
\indent สัตว์ทั้งหลายทั้งปวง, จงเป็นผู้ไม่มีเวร, อยู่เป็นสุขทุกเมื่อ, จงถึงบทอัน\\
\indent เกษม, กล่าวคือพระนิพพาน, ความปรารถนาที่ดีงามของสัตว์เหล่านั้นจงสำเร็จ เถิด\\

\pagebreak
\vspace*{\fill}
\begin{center}
  \scalebox{3}{\textbf{คำทำวัตรเย็น}}
\end{center}
\vspace{\fill}

\pagebreak

\section{คำบูชาพระรัตนตรัย}
\hrule
\textbf{โย โส ภะคะวา อะระหัง สัมมาสัมพุทโธ,}\\
\indent พระผู้มีพระภาคเจ้านั้น พระองค์ใด, เป็นพระอรหันต์,\\
\indent ดับเพลิงกิเลสเพลิงทุกข์สิ้นเชิง, ตรัสรู้ชอบได้โดยพระองค์เอง\\
\textbf{ส๎วากขาโต เยนะ ภะคะวะตา ธัมโม, }\\
\indent พระธรรม เป็นธรรมอันพระผู้มีพระภาคเจ้า พระองค์ใด, ตรัสไว้ดีแล้ว\\
\textbf{สุปะฏิปันโน ยัสสะ ภะคะวะโต สาวะกะสังโฆ,}\\
\indent พระสงฆ์สาวกของพระผู้มีพระภาคเจ้า พระองค์ใด, ปฏิบัติดีแล้ว\\
\textbf{ตัมมะยัง ภะคะวันตัง สะธัมมัง สะสังฆัง,\\
อิเมหิ สักกาเรหิ ยะถาระหัง อาโรปิเตหิ อะภิปูชะยามะ,}\\
\indent ข้าพเจ้าทั้งหลาย, ขอบูชาอย่างยิ่งซึ่งพระผู้มีพระภาคเจ้าพระองค์นั้น,\\
\indent พร้อมทั้งพระธรรมและพระสงฆ์,\\
\indent ด้วยเครื่องสักการะทั้งหลายเหล่านี้, อันยกขึ้นตามสมควรแล้วอย่างไร\\
\textbf{สาธุ โน ภันเต ภะคะวา สุจิระปะรินิพพุโตปิ,}\\
\indent ข้าแต่พระองค์ผู้เจริญ, พระผู้มีพระภาคเจ้าแม้ปรินิพพานนานแล้ว,\\
\indent ทรงสร้างคุณอันสำเร็จประโยชน์ไว้แก่ข้าพเจ้าทั้งหลาย.\\
\textbf{ปัจฉิมาชะนะตานุกัมปะมานะสา,}\\
\indent ทรงมีพระหฤทัยอนุเคราะห์แก่ข้าพเจ้า อันเป็นชนรุ่นหลัง\\
\textbf{อิเม สักกาเร ทุคคะตะปัณณาการะภูเต ปะฏิคคัณหาตุ,}\\
\indent ขอพระผู้มีพระภาคเจ้าจงรับเครื่องสักการะ อันเป็นบรรณาการของคนยากทั้งหลายเหล่านี้\\
\textbf{อัมหากัง ทีฆะรัตตัง หิตายะ สุขายะ,}\\
\indent เพื่อประโยชน์และความสุขแก่พวกข้าพเจ้าทั้งหลาย ตลอดกาลนาน เทอญฯ\\
\textbf{อะระหัง สัมมาสัมพุทโธ ภะคะวา,}\\
\indent พระผู้มีพระภาคเจ้า, เป็นพระอรหันต์, ดับเพลิงกิเลสเพลิงทุกข์สิ้นเชิง,\\
\indent ตรัสรู้ชอบได้โดยพระองค์เอง\\
\textbf{พุทธัง ภะคะวันตัง อะภิวาเทมิ,}\\
\indent ข้าพเจ้าอภิวาทพระผู้มีพระภาคเจ้า, ผู้รู้ ผู้ตื่น ผู้เบิกบาน (กราบ)\\
\textbf{ส๎วากขาโต ภะคะวะตา ธัมโม}\\
\indent พระธรรมเป็นธรรมที่พระผู้มีพระภาคเจ้า, ตรัสไว้ดีแล้ว\\
\textbf{ธัมมัง นะมัสสามิ}\\
\indent ข้าพเจ้านมัสการพระธรรม (กราบ)\\
\textbf{สุปะฏิปันโน ภะคะวะโต สาวะกะสังโฆ,}\\
\indent พระสงฆ์สาวกของพระผู้มีพระภาคเจ้า, ปฏิบัติดีแล้ว\\
\textbf{สังฆัง นะมามิ.}\\
\indent ข้าพเจ้านอบน้อมพระสงฆ์ (กราบ)\\

\pagebreak

\section{ปุพพภาคนมการ}
\hrule
\begin{center}
\textbf{(หันทะ มะยัง พุทธัสสะ ภะคะวะโต ปุพพะภาคะนะมะการัง กะโร มะ เส)}\\
เชิญเถิด เราทั้งหลาย ทำความนอบน้อมอันเป็นส่วนเบื้องต้น แด่พระผู้มีพระภาคเจ้าเถิด
\end{center}
\textbf{นะโม ตัสสะ ภะคะวะโต,}\\
\indent ขอนอบน้อมแด่พระผู้มีพระภาคเจ้าพระองค์นั้น\\
\textbf{อะระหะโต,}\\
\indent ซึ่งเป็นผู้ไกลจากกิเลส\\
\textbf{สัมมาสัมพุทธัสสะ.}\\
\indent ตรัสรู้ชอบได้โดยพระองค์เอง\\.
\begin{center}
(กล่าว ๓ ครั้ง)
\end{center}

\pagebreak

\section{พุทธานุสสติ}
\hrule
\begin{center}
\textbf{(หันทะ มะยัง พุทธานุสสะตินะยัง กะโรมะ เส)}\\
เชิญเถิด เราทั้งหลาย ทำความตามระลึกถึงพระพุทธเจ้าเถิด
\end{center}
\textbf{ตัง โข ปะนะ ภะคะวันตัง เอวัง กัล๎ยาโณ กิตติสัทโท อัพภุคคะโต,}\\
\indent ก็กิตติศัพท์อันงามของพระผู้มีพระภาคเจ้านั้น, ได้ฟุ้งไปแล้วอย่างนี้ว่า:-\\
\textbf{อิติปิ โส ภะคะวา,}\\
\indent เพราะเหตุอย่างนี้ๆ, พระผู้มีพระภาคเจ้านั้น\\
\textbf{อะระหัง,}\\
\indent เป็นผู้ไกลจากกิเลส\\
\textbf{สัมมาสัมพุทโธ,}\\
\indent เป็นผู้ตรัสรู้ชอบได้โดยพระองค์เอง\\
\textbf{วิชชาจะระณะสัมปันโน,}\\
\indent เป็นผู้ถึงพร้อมด้วยวิชชาและจรณะ\\
\textbf{สุคะโต,}\\
\indent เป็นผู้ไปแล้วด้วยดี\\
\textbf{โลกะวิทู,}\\
\indent เป็นผู้รู้โลกอย่างแจ่มแจ้ง\\
\textbf{อะนุตตะโร ปุริสะทัมมะสาระถิ,}\\
\indent เป็นผู้สามารถฝึกบุรุษที่สมควรฝึกได้อย่างไม่มีใครยิ่งกว่า\\
\textbf{สัตถา เทวะมะนุสสานัง,}\\
\indent เป็นครูผู้สอนของเทวดาและมนุษย์ทั้งหลาย\\
\textbf{พุทโธ,}\\
\indent เป็นผู้รู้ ผู้ตื่น ผู้เบิกบานด้วยธรรม\\
\textbf{ภะคะวา ติ.}\\
\indent เป็นผู้มีความจำเริญจำแนกธรรมสั่งสอนสัตว์, ดังนี้\\

\pagebreak

\section{พุทธาภิคีติ}
\hrule
\begin{center}
\textbf{(หันทะ มะยัง พุทธาภิคีติง กะโรมะ เส)}\\
เชิญเถิด เราทั้งหลาย ทำความขับคาถา พรรณนาเฉพาะพระพุทธเจ้าเถิด
\end{center}
\textbf{พุทธ๎ะวาระหันตะวะระตาทิคุณาภิยุตโต,}\\
\indent พระพุทธเจ้าประกอบด้วยคุณ, มีความประเสริฐแห่งอรหันตคุณเป็นต้น\\
\textbf{สุทธาภิญาณะกะรุณาหิ สะมาคะตัตโต,}\\
\indent มีพระองค์อันประกอบด้วยพระญาณ, และพระกรุณาอันบริสุทธิ์\\
\textbf{โพเธสิ โย สุชะนะตัง กะมะลังวะ สูโร,}\\
\indent พระองค์ใดทรงกระทำชนที่ดีให้เบิกบาน, ดุจอาทิตย์ทำบัวให้บาน\\
\textbf{วันทามะหัง ตะมะระณัง สิระสา ชิเนนทัง,}\\
\indent ข้าพเจ้าไหว้พระชินสีห์ผู้ไม่มีกิเลสพระองค์นั้นด้วยเศียรเกล้า\\
\textbf{พุทโธ โย สัพพะปาณีนัง สะระณัง เขมะมุตตะมัง,}\\
\indent พระพุทธเจ้าพระองค์ใดเป็นสรณะอันเกษมสูงสุดของสัตว์ทั้งหลาย\\
\textbf{ปะฐะมานุสสะติฏฐานัง วันทามิ ตัง สิเรนะหัง,}\\
\indent ข้าพเจ้าไหว้พระพุทธเจ้าพระองค์นั้นอันเป็นที่ตั้งแห่งความระลึกองค์ที่หนึ่ง ด้วยเศียรเกล้า\\
\textbf{พุทธัสสาหัส๎มิ ทาโสวะ (ทาสีวะ) พุทโธ เม สามิกิสสะโร,}\\
\indent ข้าพเจ้าเป็นทาสของพระพุทธเจ้า, พระพุทธเจ้าเป็นนายมีอิสระเหนือ ข้าพเจ้า,\\
\textbf{พุทโธ ทุกขัสสะ ฆาตา จะ วิธาตา จะ หิตัสสะ เม,}\\
\indent พระพุทธเจ้าเป็นเครื่องกำจัดทุกข์, และทรงไว้ซึ่งประโยชน์แก่ข้าพเจ้า\\
\textbf{พุทธัสสาหัง นิยยาเทมิ สะรีรัญชีวิตัญจิทัง,}\\
\indent ข้าพเจ้ามอบกายถวายชีวิตนี้แด่พระพุทธเจ้า\\
\textbf{วันทันโตหัง (วันทันตีหัง) จะริสสามิ, พุทธัสเสวะ สุโพธิตัง,}\\
\indent ข้าพเจ้าผู้ไหว้อยู่จักประพฤติตาม, ซึ่งความตรัสรู้ดีของพระพุทธเจ้า\\
\textbf{นัตถิ เม สะระณัง อัญญัง, พุทโธ เม สะระณัง วะรัง,}\\
\indent สรณะอื่นของข้าพเจ้าไม่มี, พระพุทธเจ้าเป็นสรณะอันประเสริฐของข้าพเจ้า\\
\textbf{เอเตนะ สัจจะวัชเชนะ วัฑเฒยยัง สัตถุสาสะเน,}\\
\indent ด้วยการกล่าวคำสัตย์นี้, ข้าพเจ้าพึงเจริญในพระศาสนาของพระศาสดา\\
\textbf{พุทธัง เม วันทะมาเนนะ (วันทะมานายะ), ยัง ปุญญัง ปะสุตัง อิธะ,}\\
\indent ข้าพเจ้าผู้ไหว้อยู่ซึ่งพระพุทธเจ้า, ได้ขวนขวายบุญใดในบัดนี้\\
\textbf{สัพเพปิ อันตะรายา เม, มาเหสุง ตัสสะ เตชะสา.}\\
\indent อันตรายทั้งปวงอย่าได้มีแก่ข้าพเจ้าด้วยเดชแห่งบุญนั้น\\
\begin{center}
(หมอบกราบ)
\end{center}
\textbf{กาเยนะ วาจายะ วะ เจตะสา วา,}\\
\indent ด้วยกายก็ดี ด้วยวาจาก็ดี ด้วยใจก็ดี\\
\textbf{พุทเธ กุกัมมัง ปะกะตัง มะยา ยัง,}\\
\indent กรรมน่าติเตียนอันใด ที่ข้าพเจ้ากระทำแล้ว ในพระพุทธเจ้า\\
\textbf{พุทโธ ปะฏิคคัณ๎หะตุ อัจจะยันตัง,}\\
\indent ขอพระพุทธเจ้าจงงดซึ่งโทษล่วงเกินอันนั้น\\
\textbf{กาลันตะเร สังวะริตุง วะ พุทเธ.}\\
\indent เพื่อการสำรวมระวัง ในพระพุทธเจ้า ในกาลต่อไป\\
\begin{center}
\emph{บทขอให้งดโทษนี้ มิได้เป็นการล้างบาป, เป็นเพียงการเปิดเผยตัวเอง;
และคำว่าโทษในที่นี้มิได้หมายถึงกรรม : หมายถึงโทษเพียงเล็กน้อยซึ่งเป็น “ส่วนตัว” ระหว่างกัน ที่พึงอโหสิกันได้.
การขอขมาชนิดนี้สำเร็จผลได้ ในเมื่อผู้ขอตั้งใจทำจริงๆ, และเป็นเพียงศีลธรรม และสิ่งที่ควรประพฤติ.}
\end{center}
\pagebreak

\section{ธัมมานุสสติ}
\hrule
\begin{center}
\textbf{(หันทะ มะยัง ธัมมานุสสะตินะยัง กะโรมะ เส)}\\
เชิญเถิด เราทั้งหลาย ทำความตามระลึกถึงพระธรรมเถิด
\end{center}
\textbf{ส๎วากขาโต ภะคะวะตา ธัมโม,}\\
\indent พระธรรมเป็นสิ่งที่พระผู้มีพระภาคเจ้า ได้ตรัสไว้ดีแล้ว\\
\textbf{สันทิฏฐิโก,}\\
\indent เป็นสิ่งที่ผู้ศึกษาและปฏิบัติพึงเห็นได้ด้วยตนเอง\\
\textbf{อะกาลิโก,}\\
\indent เป็นสิ่งที่ปฏิบัติได้ และให้ผลได้ ไม่จำกัดกาล\\
\textbf{เอหิปัสสิโก,}\\
\indent เป็นสิ่งที่ควรกล่าวกะผู้อื่นว่า ท่านจงมาดูเถิด\\
\textbf{โอปะนะยิโก,}\\
\indent เป็นสิ่งที่ควรน้อมเข้ามาใส่ตัว\\
\textbf{ปัจจัตตัง เวทิตัพโพ วิญญูหี  ติ.}\\
\indent เป็นสิ่งที่ผู้รู้ก็รู้ได้เฉพาะตน, ดังนี้.

\pagebreak
\section{ธัมมาภิคีติ}
\hrule
\begin{center}
\textbf{(หันทะ มะยัง ธัมมาภิคีติง กะโรมะ เส)}\\
เชิญเถิด เราทั้งหลาย ทำความขับคาถา พรรณนาเฉพาะพระธรรมเถิด
\end{center}
\textbf{ส๎วากขาตะตาทิคุณะโยคะวะเสนะ เสยโย,}\\
\indent พระธรรมเป็นสิ่งที่ประเสริฐเพราะประกอบด้วยคุณ,\\ 
\indent คือความที่พระผู้มีพระภาคเจ้าตรัสไว้ดีแล้วเป็นต้น\\
\textbf{โย มัคคะปากะปะริยัตติวิโมกขะเภโท,}\\
\indent เป็นธรรมอันจำแนกเป็นมรรคผลปริยัติและนิพพาน\\
\textbf{ธัมโม กุโลกะปะตะนา ตะทะธาริธารี,}\\
\indent เป็นธรรมทรงไว้ซึ่งผู้ทรงธรรม จากการตกไปสู่โลกที่ชั่ว\\
\textbf{วันทามะหัง ตะมะหะรัง วะระธัมมะเมตัง,}\\
\indent ข้าพเจ้าไหว้พระธรรมอันประเสริฐนั้น อันเป็นเครื่องขจัดเสียซึ่งความมืด\\
\textbf{ธัมโม โย สัพพะปาณีนัง สะระณัง เขมะมุตตะมัง,}\\
\indent พระธรรมใดเป็นสรณะอันเกษมสูงสุดของสัตว์ทั้งหลาย\\
\textbf{ทุติยานุสสะติฏฐานัง วันทามิ ตัง สิเรนะหัง,}\\
\indent ข้าพเจ้าไหว้พระธรรมนั้นอันเป็นที่ตั้งแห่งความระลึกองค์ที่สองด้วยเศียรเกล้า\\
\textbf{ธัมมัสสาหัส๎มิ ทาโสวะ (ทาสีวะ), ธัมโม เม สามิกิสสะโร,}\\
\indent ข้าพเจ้าเป็นทาสของพระธรรม, พระธรรมเป็นนายมีอิสระเหนือข้าพเจ้า\\
\textbf{ธัมโม ทุกขัสสะ ฆาตา จะ วิธาตา จะ หิตัสสะ เม,}\\
\indent พระธรรมเป็นเครื่องกำจัดทุกข์, และทรงไว้ซึ่งประโยชน์แก่ข้าพเจ้า\\
\textbf{ธัมมัสสาหัง นิยยาเทมิ สะรีรัญชีวิตัญจิทัง,}\\
\indent ข้าพเจ้ามอบกายถวายชีวิตนี้แด่พระธรรม\\
\textbf{วันทันโตหัง (วันทันตีหัง) \footnote{คำในวงเล็บสำหรับผู้หญิงว่า} จะริสสามิ, ธัมมัสเสวะ สุธัมมะตัง,}\\
\indent ข้าพเจ้าผู้ไหว้อยู่จักประพฤติตาม, ซึ่งความเป็นธรรมดีของพระธรรม\\
\textbf{นัตถิ เม สะระณัง อัญญัง, ธัมโม เม สะระณัง วะรัง,}\\
\indent สรณะอื่นของข้าพเจ้าไม่มี, พระธรรมเป็นสรณะอันประเสริฐของข้าพเจ้า\\
\textbf{เอเตนะ สัจจะวัชเชนะ, วัฑเฒยยัง สัตถุสาสะเน,}\\
\indent ด้วยการกล่าวคำสัตย์นี้, ข้าพเจ้าพึงเจริญในพระศาสนาของพระศาสดา\\
\textbf{ธัมมัง เม วันทะมาเนนะ (วันทะมานายะ), ยัง ปุญญัง ปะสุตัง อิธะ,}\\
\indent ข้าพเจ้าผู้ไหว้อยู่ซึ่งพระธรรม, ได้ขวนขวายบุญใดในบัดนี้\\
\textbf{สัพเพปิ อันตะรายา เม, มาเหสุง ตัสสะ เตชะสา.}\\
\indent อันตรายทั้งปวงอย่าได้มีแก่ข้าพเจ้าด้วยเดชแห่งบุญนั้น.
\begin{center}
(หมอบกราบ)
\end{center}
\textbf{กาเยนะ วาจายะ วะ เจตะสา วา,}\\
\indent ด้วยกายก็ดี ด้วยวาจาก็ดี ด้วยใจก็ดี\\
\textbf{ธัมเม กุกัมมัง ปะกะตัง มะยา ยัง,}\\
\indent กรรมน่าติเตียนอันใดที่ข้าพเจ้ากระทำแล้วในพระธรรม\\
\textbf{ธัมโม ปะฏิคคัณ๎หะตุ อัจจะยันตัง,}\\
\indent ขอพระธรรมจงงดซึ่งโทษล่วงเกินอันนั้น\\
\textbf{กาลันตะเร สังวะริตุง วะ ธัมเม.}\\
\indent เพื่อการสำรวมระวังในพระธรรมในกาลต่อไป\\

\pagebreak
\section{สังฆานุสสติ}
\hrule
\begin{center}
\textbf{(หันทะ มะยัง สังฆานุสสะตินะยัง กะโรมะ เส)}\\
เชิญเถิด เราทั้งหลาย ทำความตามระลึกถึงพระสงฆ์เถิด
\end{center}
\textbf{สุปะฏิปันโน ภะคะวะโต สาวะกะสังโฆ,}\\
\indent สงฆ์สาวกของพระผู้มีพระภาคเจ้า หมู่ใด, ปฏิบัติดีแล้ว\\
\textbf{อุชุปะฏิปันโน ภะคะวะโต สาวะกะสังโฆ,}\\
\indent สงฆ์สาวกของพระผู้มีพระภาคเจ้า หมู่ใด, ปฏิบัติตรงแล้ว\\
\textbf{ญายะปะฏิปันโน ภะคะวะโต สาวะกะสังโฆ,}\\
\indent สงฆ์สาวกของพระผู้มีพระภาคเจ้า หมู่ใด, ปฏิบัติเพื่อรู้ธรรมเป็นเครื่องออกจากทุกข์แล้ว\\
\textbf{สามีจิปะฏิปันโน ภะคะวะโต สาวะกะสังโฆ,}\\
\indent สงฆ์สาวกของพระผู้มีพระภาคเจ้าหมู่ใด, ปฏิบัติสมควรแล้ว\\
\textbf{ยะทิทัง, ได้แก่บุคคลเหล่านี้คือ :-\\
\indent จัตตาริ ปุริสะยุคานิ อัฏฐะ ปุริสะปุคคะลา,}\\
\indent คู่แห่งบุรุษ ๔ คู่, นับเรียงตัวบุรุษได้ ๘ บุรุษ \\
\textbf{เอสะ ภะคะวะโต สาวะกะสังโฆ,}\\
\indent นั่นแหละ สงฆ์สาวกของพระผู้มีพระภาคเจ้า\\
\textbf{อาหุเนยโย,}\\
\indent เป็นสงฆ์ควรแก่สักการะ ที่เขานำมาบูชา\\
\textbf{ปาหุเนยโย,}\\
\indent เป็นสงฆ์ควรแก่สักการะที่เขาจัดไว้ต้อนรับ\\
\textbf{ทักขิเณยโย,}\\
\indent เป็นผู้ควรรับทักษิณาทาน\\
\textbf{อัญชะลิกะระณีโย,}\\
\indent เป็นผู้ที่บุคคลทั่วไปควรทำอัญชลี\\
\textbf{อะนุตตะรัง ปุญญักเขตตัง โลกัสสา ติ.}\\
\indent เป็นเนื้อนาบุญของโลก, ไม่มีนาบุญอื่นยิ่งกว่า ดังนี้\\

\pagebreak
\section{สังฆาภิคีติ}
\hrule
\begin{center}
\textbf{(หันทะ มะยัง สังฆาภิคีติง กะโรมะ เส)}\\
เชิญเถิด เราทั้งหลาย ทำความขับคาถา พรรณนาเฉพาะพระสงฆ์เถิด
\end{center}
\textbf{สัทธัมมะโช สุปะฏิปัตติคุณาทิยุตโต,}\\
\indent พระสงฆ์ที่เกิดโดยพระสัทธรรม, ประกอบด้วยคุณมีความปฏิบัติดีเป็นต้น\\
\textbf{โยฏฐัพพิโธ อะริยะปุคคะละสังฆะเสฏโฐ,}\\
\indent เป็นหมู่แห่งพระอริยบุคคลอันประเสริฐแปดจำพวก\\
\textbf{สีลาทิธัมมะปะวะราสะยะกายะจิตโต,}\\
\indent มีกายและจิตอันอาศัยธรรมมีศีลเป็นต้นอันบวร\\
\textbf{วันทามะหัง ตะมะริยานะ คะณัง สุสุทธัง,}\\
\indent ข้าพเจ้าไหว้หมู่แห่งพระอริยเจ้าเหล่านั้น อันบริสุทธิ์ด้วยดี\\
\textbf{สังโฆ โย สัพพะปาณีนัง สะระณัง เขมะมุตตะมัง,}\\
\indent พระสงฆ์หมู่ใดเป็นสรณะอันเกษมสูงสุดของสัตว์ทั้งหลาย\\
\textbf{ตะติยานุสสะติฏฐานัง วันทามิ ตัง สิเรนะหัง,}\\
\indent ข้าพเจ้าไหว้พระสงฆ์หมู่นั้น อันเป็นที่ตั้งแห่งความระลึกองค์ที่สาม ด้วยเศียรเกล้า\\
\textbf{สังฆัสสาหัส๎มิ ทาโสวะ (ทาสีวะ), สังโฆ เม สามิกิสสะโร,}\\
\indent ข้าพเจ้าเป็นทาสของพระสงฆ์, พระสงฆ์เป็นนายมีอิสระเหนือข้าพเจ้า\\
\textbf{สังโฆ ทุกขัสสะ ฆาตา จะ วิธาตา จะ หิตัสสะ เม,}\\
\indent พระสงฆ์เป็นเครื่องกำจัดทุกข์, และทรงไว้ซึ่งประโยชน์แก่ข้าพเจ้า\\
\textbf{สังฆัสสาหัง นิยยาเทมิ สะรีรัญชีวิตัญจิทัง,}\\
\indent ข้าพเจ้ามอบกายถวายชีวิตนี้แด่พระสงฆ์\\
\textbf{วันทันโตหัง (วันทันตีหัง) สังฆัสโสปะฏิปันนะตัง,}\\
\indent ข้าพเจ้าผู้ไหว้อยู่จักประพฤติตาม, ซึ่งความปฏิบัติดีของพระสงฆ์\\
\textbf{นัตถิ เม สะระณัง อัญญัง, สังโฆ เม สะระณัง วะรัง,}\\
\indent สรณะอื่นของข้าพเจ้าไม่มี, พระสงฆ์เป็นสรณะอันประเสริฐของข้าพเจ้า\\
\textbf{เอเตนะ สัจจะวัชเชนะ, วัฑเฒยยัง สัตถุ สาสะเน,}\\
\indent ด้วยการกล่าวคำสัตย์นี้, ข้าพเจ้าพึงเจริญในพระศาสนาของพระศาสดา\\
\textbf{สังฆัง เม วันทะมาเนนะ (วันทะมานายะ), ยัง ปุญญัง ปะสุตัง อิธะ,}\\
\indent ข้าพเจ้าผู้ไหว้อยู่ซึ่งพระสงฆ์, ได้ขวนขวายบุญใด ในบัดนี้\\
\textbf{สัพเพปิ อันตะรายา เม, มาเหสุง ตัสสะ เตชะสา.}\\
\indent อันตรายทั้งปวงอย่าได้มีแก่ข้าพเจ้าด้วยเดชแห่งบุญนั้น.\\
\begin{center}
(หมอบกราบ)
\end{center}
\textbf{กาเยนะ วาจายะ วะ เจตะสา วา,}\\
\indent ด้วยกายก็ดี ด้วยวาจาก็ดี ด้วยใจก็ดี\\
\textbf{สังเฆ กุกัมมัง ปะกะตัง มะยา ยัง,}\\
\indent กรรมน่าติเตียนอันใดที่ข้าพเจ้าได้กระทำแล้วในพระสงฆ์\\
\textbf{สังโฆ ปะฏิคคัณ๎หะตุ อัจจะยันตัง,}\\
\indent ขอพระสงฆ์ จงงดซึ่งโทษล่วงเกินอันนั้น\\
\textbf{กาลันตะเร สังวะริตุง วะ สังเฆ.}\\
\indent เพื่อการสำรวมระวัง ในพระสงฆ์ในกาลต่อไป\\
\begin{center}
(จบคำทำวัตรเย็น)
\end{center}



\pagebreak
\section{สรณคมนปาฐะ}
\hrule
\begin{center}
\textbf{(หันทะ มะยัง ติสะระณะคะมะนะปาฐัง ภะณามะ เส)}\\
เชิญเถิด เราทั้งหลาย จงกล่าวคาถาเพื่อระลึกถึงพระรัตนตรัยเถิด
\end{center}
\textbf{พุทธัง สะระณัง คัจฉามิ}\\
\indent ข้าพเจ้าถือเอาพระพุทธเจ้าเป็นสรณะ\\
\textbf{ธัมมัง สะระณัง คัจฉามิ}\\
\indent ข้าพเจ้าถือเอาพระธรรมเป็นสรณะ\\
\textbf{สังฆัง สะระณัง คัจฉามิ}\\
\indent ข้าพเจ้าถือเอาพระสงฆ์เป็นสรณะ\\
\textbf{ทุติยัมปิ พุทธัง สะระณัง คัจฉามิ}\\
\indent แม้ครั้งที่สอง ข้าพเจ้าถือเอาพระพุทธเจ้าเป็นสรณะ\\
\textbf{ทุติยัมปิ ธัมมัง สะระณัง คัจฉามิ}\\
\indent แม้ครั้งที่สอง ข้าพเจ้าถือเอาพระธรรมเป็นสรณะ\\
\textbf{ทุติยัมปิ สังฆัง สะระณัง คัจฉามิ}\\
\indent แม้ครั้งที่สอง ข้าพเจ้าถือเอาพระสงฆ์เป็นสรณะ\\
\textbf{ตะติยัมปิ พุทธัง สะระณัง คัจฉามิ}\\
\indent แม้ครั้งที่สาม ข้าพเจ้าถือเอาพระพุทธเจ้าเป็นสรณะ\\
\textbf{ตะติยัมปิ ธัมมัง สะระณัง คัจฉามิ}\\
\indent แม้ครั้งที่สาม ข้าพเจ้าถือเอาพระธรรม เป็นสรณะ\\
\textbf{ตะติยัมปิ สังฆัง สะระณัง คัจฉามิ}\\
\indent แม้ครั้งที่สาม ข้าพเจ้าถือเอาพระสงฆ์ เป็นสรณะ\\

\pagebreak
\section{เขมาเขมสรณทีปิกคาถา}
\hrule
\begin{center}
\textbf{(หันทะ มะยัง เขมาเขมะสะระณะทีปิกะคาถาโย ภะณามะ เส)}\\
เชิญเถิด เราทั้งหลาย จงกล่าวคาถาแสดงสรณะอันเกษมและไม่เกษมเถิด
\end{center}
\textbf{พะหุง เว สะระณัง ยันติ ปัพพะตานิ วะนานิ จะ, \\
อารามะรุกขะเจต๎ยานิ มะนุสสา ภะยะตัชชิตา}\\
\indent มนุษย์เป็นอันมาก, เมื่อเกิดมีภัยคุกคามแล้ว ก็ถือเอาภูเขาบ้าง, \\
\indent ป่าไม้บ้าง, อารามและรุกขเจดีย์บ้างเป็นสรณะ\\
\textbf{เนตัง โข สะระณัง เขมัง, เนตัง สะระณะมุตตะมัง,\\
เนตัง สะระณะมาคัมมะ สัพพะ ทุกขา ปะมุจจะติ}\\
\indent นั่น มิใช่สรณะอันเกษมเลย, นั่นมิใช่สรณะอันสูงสุด,\\
\indent เขาอาศัยสรณะนั่นแล้ว, ย่อมไม่พ้นจากทุกข์ทั้งปวงได้\\
\textbf{โย จะ พุทธัญจะ ธัมมัญจะ สังฆัญจะ สะระณัง คะโต,\\
จัตตาริ อะริยะสัจจานิ สัมมัปปัญญายะ ปัสสะติ}\\
\indent ส่วนผู้ใดถือเอาพระพุทธ พระธรรม พระสงฆ์ เป็นสรณะแล้ว,\\
\indent เห็นอริยสัจคือความจริงอันประเสริฐสี่ด้วยปัญญาอันชอบ\\
\textbf{ทุกขัง ทุกขะสะมุปปาทัง ทุกขัสสะ จะ อะติกกะมัง,\\
อะริยัญจัฏฐังคิกัง มัคคัง ทุกขูปะสะมะคามินัง}\\
\indent คือเห็นความทุกข,  เหตุให้เกิดทุกข์, ความก้าวล่วงพ้นทุกข์เสียได้,\\
\indent และหนทางมีองค์แปดอันประเสริฐเครื่องถึงความระงับทุกข์\\
\textbf{เอตัง โข สะระณัง เขมัง เอตัง สะระณะมุตตะมัง,\\
เอตัง สะระณะมาคัมมะ สัพพะทุกขา ปะมุจจะติ}\\
\indent นั่นแหละเป็นสรณะอันเกษม, นั่นเป็นสรณะอันสูงสุด,\\
\indent เขาอาศัยสรณะนั่นแล้ว, ย่อมพ้นจากทุกข์ทั้งปวงได้\\

\pagebreak
\section{อริยธนคาถา}
\hrule
\begin{center}
\textbf{(หันทะ มะยัง อะริยะธะนะคาถาโย ภะณามะ เส)}\\
\indent เชิญเถิด เราทั้งหลาย จงกล่าวคาถาสรรเสริญพระอริยเจ้าเถิด.
\end{center}
\textbf{ยัสสะ สัทธา ตะถาคะเต อะจะลา สุปะติฏฐิตา}\\
\indent ศรัทธาในพระตถาคตของผู้ใดตั้งมั่นอย่างดีไม่หวั่นไหว\\
\textbf{สีลัญจะ ยัสสะ กัล๎ยาณัง อะริยะกันตัง ปะสังสิตัง}\\
\indent และศีลของผู้ใดงดงาม, เป็นที่สรรเสริญที่พอใจของพระอริยเจ้า\\
\textbf{สังเฆ ปะสาโท ยัสสัตถิ อุชุภูตัญจะ ทัสสะนัง}\\
\indent ความเลื่อมใสของผู้ใดมีในพระสงฆ์, และความเห็นของผู้ใดตรง\\
\textbf{อะทะลิทโทติ ตัง อาหุ อะโมฆันตัสสะ ชีวิตัง}\\
\indent บัณฑิตกล่าวเรียกเขาผู้นั้นว่าคนไม่จน, ชีวิตของเขาไม่เป็นหมัน\\
\textbf{ตัส๎มา สัทธัญจะ สีลัญจะ ปะสาทัง ธัมมะทัสสะนัง,\\
อะนุยุญเชถะ เมธาวี สะรัง พุทธานะ สาสะนัง}\\
\indent เพราะฉะนั้น, เมื่อระลึกได้ถึงคำสั่งสอนของพระพุทธเจ้าอยู่,\\
\indent ผู้มีปัญญาควรสร้างศรัทธาศีล, ความเลื่อมใส, และความเห็นธรรมให้เนืองๆ\\

\pagebreak
\section{ติลักขณาทิคาถา}
\hrule
\begin{center}
\textbf{(หันทะ มะยัง ติลักขะณาคาถาโย ภะณามะ เส)}\\
เชิญเถิด เราทั้งหลาย จงกล่าวคาถาแสดงพระไตรลักษณ์เป็นเบื้องต้นเถิด
\end{center}
\textbf{สัพเพ สังขารา อะนิจจาติ ยะทา ปัญญายะ ปัสสะติ}\\
\indent เมื่อใดบุคคลเห็นด้วยปัญญาว่า, สังขารทั้งปวงไม่เที่ยง\\
\textbf{อะถะ นิพพินทะติ ทุกเข เอสะ มัคโค วิสุทธิยา}\\
\indent เมื่อนั้น ย่อมเหนื่อยหน่ายในสิ่งที่เป็นทุกข์ที่ตนหลง,\\
\indent นั่นแหละเป็นทางแห่งพระนิพพานอันเป็นธรรมหมดจด\\
\textbf{สัพเพ สังขารา ทุกขาติ ยะทา ปัญญายะ ปัสสะติ}\\
\indent เมื่อใดบุคคล เห็นด้วยปัญญาว่า, สังขารทั้งปวงเป็นทุกข์\\
\textbf{อะถะ นิพพินทะติ ทุกเข เอสะ มัคโค วิสุทธิยา}\\
\indent เมื่อนั้น ย่อมเหนื่อยหน่ายในสิ่งที่เป็นทุกข์ที่ตนหลง,\\
\indent นั่นแหละเป็นทางแห่งพระนิพพานอันเป็นธรรมหมดจด\\
\textbf{สัพเพ ธัมมา อะนัตตาติ ยะทา ปัญญายะ ปัสสะติ}\\
\indent เมื่อใดบุคคลเห็นด้วยปัญญาว่า, ธรรมทั้งปวงเป็นอนัตตา\\
\textbf{อะถะ นิพพินทะติ ทุกเข เอสะ มัคโค วิสุทธิยา}\\
\indent เมื่อนั้น ย่อมเหนื่อยหน่ายในสิ่งที่เป็นทุกข์ ที่ตนหลง,\\
\indent นั่นแหละเป็นทางแห่งพระนิพพานอันเป็นธรรมหมดจด\\
\textbf{อัปปะกา เต มะนุสเสสุ เย ชะนา ปาระคามิโน}\\
\indent ในหมู่มนุษย์ทั้งหลาย, ผู้ที่ถึงฝั่งแห่งพระนิพพานมีน้อยนัก\\
\textbf{อะถายัง อิตะรา ปะชา ตีระเมวานุธาวะติ}\\
\indent หมู่มนุษย์นอกนั้นย่อมวิ่งเลาะอยู่ตามฝั่งในนี่เอง\\
\textbf{เย จะ โข สัมมะทักขาเต ธัมเม ธัมมานุวัตติโน}\\
\indent ก็ชนเหล่าใดประพฤติสมควรแก่ธรรมในธรรมที่ตรัสรู้ไว้ชอบแล้ว\\
\textbf{เต ชะนา ปาระเมสสันติ มัจจุเธยยัง สุทุตตะรัง}\\
\indent ชนเหล่าใดจักถึงฝั่งแห่งพระนิพพาน, ข้ามพ้นบ่วงแห่งมัจจุราชที่ข้ามได้ยากนัก\\
\textbf{กัณหัง ธัมมัง วิปปะหายะ สุกกัง ภาเวถะ ปัณฑิโต}\\
\indent จงเป็นบัณฑิตละธรรมดำเสีย, แล้วเจริญธรรมขาว\\
\textbf{โอกา อะโนกะมาคัมมะ วิเวเก ยัตถะ ทูระมัง,\\
ตัต๎ราภิระติมิจเฉยยะ หิต๎วา กาเม อะกิญจะโน}\\
\indent จงมาถึงที่ไม่มีน้ำ, จากที่มีน้ำ, จงละกามเสีย, เป็นผู้ไม่มีความกังวล,\\
\indent จงยินดีเฉพาะต่อพระนิพพาน, อันเป็นที่สงัดซึ่งสัตว์ยินดีได้โดยยาก\\

\pagebreak
\section{ภารสุตตคาถา}
\hrule
\begin{center}
\textbf{(หันทะ มะยัง ภาระสุตตะคาถาโย ภะณามะ เส)}\\
เชิญเถิด เราทั้งหลาย จงกล่าวคาถาแสดงภารสูตรเถิด
\end{center}
\textbf{ภารา หะเว ปัญจักขันธา }\\
\indent ขันธ์ทั้งห้า เป็นของหนักเน้อ\\
\textbf{ภาระหาโร จะ ปุคคะโล }\\
\indent บุคคลแหละเป็นผู้แบกของหนักพาไป\\
\textbf{ภาราทานัง ทุกขัง โลเก }\\
\indent การแบกถือของหนักเป็นความทุกข์ในโลก\\
\textbf{ภาระนิกเขปะนัง สุขัง }\\
\indent การสลัดของหนักทิ้งลงเสียเป็นความสุข\\
\textbf{นิกขิปิต๎วา คะรุง ภารัง }\\
\indent พระอริยเจ้าสลัดทิ้งของหนักลงเสียแล้ว\\
\textbf{อัญญัง ภารัง อะนาทิยะ }\\
\indent ทั้งไม่หยิบฉวยเอาของหนักอันอื่นขึ้นมาอีก\\
\textbf{สะมูลัง ตัณหัง อัพพุย๎หะ }\\
\indent เป็นผู้ถอนตัณหาขึ้นได้กระทั่งราก\\
\textbf{นิจฉาโต ปะรินิพพุโต }\\
\indent เป็นผู้หมดสิ่งปรารถนาดับสนิทไม่มีส่วนเหลือ

\pagebreak
\section{ภัทเทกรัตตคาถา}
\hrule
\begin{center}
\textbf{(หันทะ มะยัง ภัทเทกะรัตตะคาถาโย ภะณามะ เส)}\\
เชิญเถิด เราทั้งหลาย จงกล่าวคาถาแสดงผู้มีราตรีเดียวเจริญเถิด
\end{center}
\textbf{อะตีตัง นาน๎วาคะเมยยะ นัปปะฏิกังเข อะนาคะตัง}\\
\indent บุคคลไม่ควรตามคิดถึงสิ่งที่ล่วงไปแล้วด้วยอาลัย, และไม่พึงพะวงถึงสิ่งที่ยังไม่มาถึง\\
\textbf{ยะทะตีตัมปะหีนันตัง อัปปัตตัญจะ อะนาคะตัง}\\
\indent สิ่งเป็นอดีตก็ละไปแล้ว, สิ่งเป็นอนาคตก็ยังไม่มา\\
\textbf{ปัจจุปปันนัญจะ โย ธัมมัง ตัตถะ ตัตถะ วิปัสสะติ,\\
อะสังหิรัง อะสังกุปปัง ตัง วิทธา มะนุพ๎รูหะเย}\\
\indent ผู้ใดเห็นธรรมอันเกิดขึ้นเฉพาะหน้าที่นั้นๆ อย่างแจ่มแจ้ง,\\
\indent ไม่ง่อนแง่นคลอนแคลน, เขาควรพอกพูนอาการเช่นนั้นไว้\\
\textbf{อัชเชวะ กิจจะมาตัปปัง โก ชัญญา มะระณัง สุเว}\\
\indent ความเพียรเป็นกิจที่ต้องทำวันนี้, ใครจะรู้ความตายแม้พรุ่งนี้\\
\textbf{นะ หิ โน สังคะรันเตนะ มะหาเสเนนะ มัจจุนา}\\
\indent เพราะการผัดเพี้ยนต่อมัจจุราชซึ่งมีเสนามากย่อมไม่มีสำหรับเรา\\
\textbf{เอวัง วิหาริมาตาปิง อะโหรัตตะมะตันทิตัง,\\
ตัง เว ภัทเทกะรัตโตติ สันโต อาจิกขะเต มุนิ}\\
\indent มุนีผู้สงบย่อมกล่าวเรียกผู้มีความเพียรอยู่เช่นนั้น,\\
\indent ไม่เกียจคร้านทั้งกลางวันกลางคืนว่า, "ผู้เป็นอยู่แม้เพียงราตรีเดียว ก็น่าชม"\\

\pagebreak
\section{ธัมมคารวาทิคาถา}
\hrule
\begin{center}
\textbf{(หันทะ มะยัง ธัมมะคาระวาทิคาถาโย ภะณามะ เส)}\\
เชิญเถิด เราทั้งหลาย จงกล่าวคาถาแสดงความเคารพพระธรรมเถิด
\end{center}
\textbf{เย จะ อะตีตา สัมพุทธา เย จะ พุทธา อะนาคะตา,\\
โย เจตะระหิ สัมพุทโธ พะหุนนัง โสกะนาสะโน}\\
\indent พระพุทธเจ้าบรรดาที่ล่วงไปแล้วด้วย, ที่ยังไม่มาตรัสรู้ด้วย,\\
\indent และพระพุทธเจ้าผู้ขจัดโศกของมหาชนในกาลบัดนี้ด้วย\\
\textbf{สัพเพ สัทธัมมะคะรุโน วิหะริงสุ วิหาติ จะ,\\
อะถาปิ วิหะริสสันติ เอสา พุทธานะ ธัมมะตา}\\
\indent พระพุทธเจ้าทั้งปวงนั้นทุกพระองค์เคารพพระธรรม, ได้เป็นมาแล้วด้วย,\\
\indent กำลังเป็นอยู่ด้วย, และจักเป็นด้วย, เพราะธรรมดาของพระพุทธเจ้า ทั้งหลายเป็นเช่นนั้นเอง\\
\textbf{ตัส๎มา หิ อัตตะกาเมนะ มะหัตตะมะภิกังขะตา,\\
สัทธัมโม คะรุกาตัพโพ สะรัง พุทธานะ สาสะนัง}\\
\indent เพราะฉะนั้น, บุคคลผู้รักตน, หวังอยู่เฉพาะคุณเบื้องสูง,\\
\indent เมื่อระลึกได้ถึงคำสั่งสอนของพระพุทธเจ้าอยู่, จงทำความเคารพพระธรรม\\
\textbf{นะ หิ ธัมโม อะธัมโม จะ อุโภ สะมะวิปากิโน}\\
\indent ธรรมและอธรรมจะมีผลเหมือนกันทั้งสองอย่างหามิได้\\
\textbf{อะธัมโม นิระยัง เนติ ธัมโม ปาเปติ สุคะติง}\\
\indent อธรรมย่อมนำไปนรก, ธรรมย่อมนำให้ถึงสุคติ\\
\textbf{ธัมโม หะเว รักขะติ ธัมมะจาริง}\\
\indent ธรรมแหละย่อมรักษาผู้ประพฤติธรรมเป็นนิตย์\\
\textbf{ธัมโม สุจิณโณ สุขะมาวะหาติ}\\
\indent ธรรมที่ประพฤติดีแล้วย่อมนำสุขมาให้ตน\\
\textbf{เอสานิสังโส ธัมเม สุจิณเณ}\\
\indent นี่เป็นอานิสงส์ในธรรมที่ตนประพฤติดีแล้ว\\

\pagebreak
\section{โอวาทปาฏิโมกขคาถา}
\hrule
\begin{center}
\textbf{(หันทะ มะยัง โอวาทะปาติโมกขะคาถาโย ภะณามะ เส)}\\
เชิญเถิด เราทั้งหลาย จงกล่าวคาถาแสดงพระโอวาทปาติโมกข์เถิด
\end{center}
\textbf{สัพพะปาปัสสะ อะกะระณัง}\\
\indent การไม่ทำบาปทั้งปวง\\
\textbf{กุสะลัสสูปะสัมปะทา}\\
\indent การทำกุศลให้ถึงพร้อม\\
\textbf{สะจิตตะปะริโยทะปะนัง}\\
\indent การชำระจิตของตนให้ขาวรอบ\\
\textbf{เอตัง พุทธานะ สาสะนัง}\\
\indent ธรรม ๓ อย่างนี้เป็นคำสั่งสอนของพระพุทธเจ้าทั้งหลาย\\
\textbf{ขันตี ปะระมัง ตะโป ตีติกขา}\\
\indent ขันติ คือความอดกลั้นเป็นธรรมเครื่องเผากิเลสอย่างยิ่ง\\
\textbf{นิพพานัง ปะระมัง วะทันติ พุทธา}\\
\indent ผู้รู้ทั้งหลายกล่าวพระนิพพานว่าเป็นธรรมอันยิ่ง\\
\textbf{นะ หิ ปัพพะชิโต ปะรูปะฆาตี}\\
\indent ผู้กำจัดสัตว์อื่นอยู่ไม่ชื่อว่าเป็นบรรพชิตเลย\\
\textbf{สะมะโณ โหติ ปะรัง วิเหฐะยันโต}\\
\indent ผู้ทำสัตว์อื่นให้ลำบากอยู่ไม่ชื่อว่าเป็นสมณะเลย\\
\textbf{อะนูปะวาโท อะนูปะฆาโต}\\
\indent การไม่พูดร้าย, การไม่ทำร้าย\\
\textbf{ปาติโมกเข จะ สังวะโร}\\
\indent การสำรวมในปาติโมกข์\\
\textbf{มัตตัญญุตา จะ ภัตตัส๎มิง}\\
\indent ความเป็นผู้รู้ประมาณในการบริโภค\\
\textbf{ปันตัญจะ สะยะนาสะนัง}\\
\indent การนอนการนั่งในที่อันสงัด\\
\textbf{อะธิจิตเต จะ อาโยโค}\\
\indent ความหมั่นประกอบในการทำจิตให้ยิ่ง\\
\textbf{เอตัง พุทธานะ สาสะนัง}\\
\indent ธรรม ๖ อย่างนี้เป็นคำสั่งสอนของพระพุทธเจ้าทั้งหลาย\\

\pagebreak
\section{ปฐมพุทธภาสิตคาถา}
\hrule
\begin{center}
\textbf{(หันทะ มะยัง ปะฐะมะพุทธะภาสิตะคาถาโย ภะณามะ เส)}\\
เชิญเถิด เราทั้งหลาย จงกล่าวคาถาพุทธภาษิตครั้งแรกของพระพุทธเจ้าเถิด
\end{center}
\textbf{อะเนกะชาติสังสารัง สันธาวิสสัง อะนิพพิสัง}\\
\indent เมื่อเรายังไม่พบญาณ, ได้แล่นท่องเที่ยวไปในสงสารอันเป็นอเนกชาติ\\
\textbf{คะหะการัง คะเวสันโต ทุกขา ชาติ ปุนัปปุนัง,}\\
\indent แสวงหาอยู่ซึ่งนายช่างปลูกเรือน,\\
\indent คือตัณหาผู้สร้างภพ, การเกิดทุกคราวเป็นทุกข์ร่ำไป\\
\textbf{คะหะการะกะ ทิฏโฐสิ ปุนะ เคหัง นะ กาหะสิ}\\
\indent นี่แน่ะ นายช่างปลูกเรือน, เรารู้จักเจ้าเสียแล้ว, เจ้าจะทำเรือนให้เราไม่ได้อีกต่อไป\\
\textbf{สัพพา เต ผาสุกา ภัคคา คะหะกูฏัง วิสังขะตัง,}\\
\indent โครงเรือนทั้งหมดของเจ้าเราหักเสียแล้ว, ยอดเรือนเราก็รื้อเสียแล้ว\\
\textbf{วิสังขาระคะตัง จิตตัง ตัณหานัง ขะยะมัชฌะคา}\\
\indent จิตของเราถึงแล้วซึ่งสภาพที่อะไรปรุงแต่งไม่ได้อีกต่อไป,\\
\indent มันได้ถึงแล้วซึ่งความสิ้นไปแห่งตัณหา (คือ ถึงนิพพาน)\\

\pagebreak
\section{ปัจฉิมพุทโธวาทปาฐะ}
\hrule
\begin{center}
\textbf{(หันทะ มะยัง ปัจฉิมะพุทโธวาทะปาฐัง ภะณามะ เส)}\\
เชิญเถิด เราทั้งหลาย จงกล่าวคำแสดงพระโอวาทครั้งสุดท้ายของพระพุทธเจ้าเถิด
\end{center}
\textbf{หันทะทานิ ภิกขะเว อามันตะยามิ โว}\\
\indent ดูก่อน ภิกษุทั้งหลาย, บัดนี้เราขอเตือนท่านทั้งหลายว่า\\
\textbf{วะยะธัมมา สังขารา}\\
\indent สังขารทั้งหลายมีความเสื่อมไปเป็นธรรมดา\\
\textbf{อัปปะมาเทนะ สัมปาเทถะ}\\
\indent ท่านทั้งหลายจงทำความไม่ประมาทให้ถึงพร้อมเถิด\\
\textbf{อะยัง ตะถาคะตัสสะ ปัจฉิมา วาจา}\\
\indent นี้เป็นพระวาจามีในครั้งสุดท้ายของพระตถาคตเจ้า

\pagebreak
\section{อุททิสสนาธิฏฐานคาถา}
\hrule
\begin{center}
\textbf{(หันทะ มะยัง อุททิสสะนาธิฏฐานะคาถาโย ภะณามะ เส)}\\
เชิญเถิด เราทั้งหลาย จงสวดคาถาอุทิศและอธิษฐานเถิด
\end{center}
\textbf{อิมินา ปุญญะกัมเมนะ}\\
\indent ด้วยบุญนี้อุทิศให้\\
\textbf{อุปัชฌายา คุณุตตะรา}\\
\indent อุปัชฌาย์ผู้เลิศคุณ\\
\textbf{อาจะริยูปะการา จะ}\\
\indent แลอาจารย์ผู้เกื้อหนุน\\
\textbf{มาตา ปิตา จะ ญาตะกา}\\
\indent ทั้งพ่อแม่แลปวงญาติ\\
\textbf{สุริโย จันทิมา ราชา}\\
\indent สูรย์จันทร์และราชา\\
\textbf{คุณะวันตา นะราปิ จะ}\\
\indent ผู้ทรงคุณหรือสูงชาติ\\
\textbf{พ๎รัห๎มะมารา จะ อินทา จะ}\\
\indent พรหมมารและอินทราช\\
\textbf{โลกะปาลา จะ เทวะตา}\\
\indent ทั้งทวยเทพและโลกบาล\\
\textbf{ยะโม มิตตา มะนุสสา จะ}\\
\indent ยมราชมนุษย์มิตร\\
\textbf{มัชฌัตตา เวริกาปิ จะ}\\
\indent ผู้เป็นกลางผู้จองผลาญ\\
\textbf{สัพเพ สัตตา สุขี โหนตุ}\\
\indent ขอให้เป็นสุขศานติ์ทุกทั่วหน้าอย่าทุกข์ทน\\
\textbf{ปุญญานิ ปะกะตานิ เม}\\
\indent บุญผองที่ข้าทำจงช่วยอำนวยศุภผล\\
\textbf{สุขัง จะ ติวิธัง เทนตุ}\\
\indent ให้สุขสามอย่างล้น\\
\textbf{ขิปปัง ปาเปถะ โว มะตัง}\\
\indent ให้ลุถึงนิพพานพลัน\\
\textbf{อิมินา ปุญญะกัมเมนะ}\\
\indent ด้วยบุญนี้ที่เราทำ\\
\textbf{อิมินา อุททิเสนะ จะ}\\
\indent แลอุทิศให้ปวงสัตว์\\
\textbf{ขิปปาหัง สุละเภ เจวะ}\\
\indent เราพลันได้ซึ่งการตัด\\
\textbf{ตัณหุปาทานะเฉทะนัง}\\
\indent ตัวตัณหาอุปาทาน\\
\textbf{เย สันตาเน หินา ธัมมา}\\
\indent สิ่งชั่วในดวงใจ\\
\textbf{ยาวะ นิพพานะโต มะมัง}\\
\indent กว่าเราจะถึงนิพพาน\\
\textbf{นัสสันตุ สัพพะทา เยวะ}\\
\indent มลายสิ้นจากสันดาน\\
\textbf{ยัตถะ ชาโต ภะเว ภะเว}\\
\indent ทุกๆ ภพที่เราเกิด\\
\textbf{อุชุจิตตัง สะติปัญญา}\\
\indent มีจิตตรงและสติทั้งปัญญาอันประเสริฐ\\
\textbf{สัลเลโข วิริยัมหินา}\\
\indent พร้อมทั้งความเพียรเลิศเป็นเครื่องขูดกิเลสหาย\\
\textbf{มารา ละภันตุ โนกาสัง}\\
\indent โอกาสอย่าพึงมีแก่หมู่มารสิ้นทั้งหลาย\\
\textbf{กาตุญจะ วิริเยสุ เม}\\
\indent เป็นช่องประทุษร้ายทำลายล้างความเพียรจม\\
\textbf{พุทธาธิปะวะโร นาโถ}\\
\indent พระพุทธผู้บวรนาถ\\
\textbf{ธัมโม นาโถ วะรุตตะโม}\\
\indent พระธรรมที่พึ่งอุดม\\
\textbf{นาโถ ปัจเจกะพุทโธ จะ}\\
\indent พระปัจเจกะพุทธสม\\
\textbf{สังโฆ นาโถตตะโร มะมัง}\\
\indent ทบพระสงฆ์ที่พึ่งผยอง\\
\textbf{เตโสตตะมานุภาเวนะ}\\
\indent ด้วยอานุภาพนั้น\\
\textbf{มาโรกาสัง ละภันตุ มา}\\
\indent ขอหมู่มาร อย่าได้ช่อง\\
\textbf{ทะสะปุญญานุภาเวนะ}\\
\indent ด้วยเดชบุญทั้งสิบป้อง\\
\textbf{มาโรกาสัง ละภันตุ มา}\\
\indent อย่าเปิดโอกาสแก่มาร เทอญ\\

\pagebreak
\vspace*{\fill}
\begin{center}
  \scalebox{3}{\textbf{บทสวดมนต์เสริม}}
\end{center}
\vspace{\fill}
\pagebreak

\section{อานาปานสติสูตร}
\hrule
\begin{center}
\textbf{(อานาปานะสะติ ภิกขะเว ภาวิตา พะหุลีกะตา)}\\
ดูก่อนภิกษุทั้งหลาย อานาปานสติอันบุคคลเจริญทำให้มากแล้ว
\end{center}
\textbf{มะหัปผะลา โหติ มะหานิสังสา}\\
\indent ย่อมมีผลใหญ่ มีอานิสงส์ใหญ่\\
\textbf{อานาปานะสะติ ภิกขะเว ภาวิตา พะหุลีกะตา}\\
\indent ดูก่อนภิกษุทั้งหลาย อานาปานสติอันบุคคลเจริญทำให้มากแล้ว\\
\textbf{จัตตาโร สะติปัฏฐาเน ปะริปูเรนติ}\\
\indent ย่อมทำให้สติปัฏฐานทั้งสี่ให้บริบูรณ์\\
\textbf{จัตตาโร สะติปัฏฐานา ภาวิตา พะหุลีกะตา}\\
\indent สติปัฏฐานทั้งสี่อันบุคคลเจริญทำให้มากแล้ว\\
\textbf{โพชฌังเค ปะริปูเรนติ}\\
\indent ย่อมทำโพชฌงค์ทั้งเจ็ดให้บริบูรณ์\\
\textbf{สัตตะ โพชฌังคา ภาวิตา พะหุลีกะตา}\\
\indent โพชฌงค์ทั้งเจ็ดอันบุคคลเจริญทำให้มากแล้ว\\
\textbf{วิชชา วิมุตติง ปะริปูเรนติ}\\
\indent ย่อมทำวิชชาและวิมุติให้บริบูรณ์\\
\textbf{กะถัง ภาวิตา จะ ภิกขะเว อานาปานสะติ, กะถัง พะหุลีกะตา}\\
\indent ดูก่อนภิกษุทั้งหลาย ก็อานาปานสติอันบุคคลเจริญทำให้มากแล้วอย่างไรเล่า\\
\textbf{มะหัปผะลา โหติ มะหานิสังสา}\\
\indent จึงมีผลใหญ่ มีอานิสงส์ใหญ่\\
\textbf{อิธะ ภิกขะเว ภิกขุ}\\
\indent ดูก่อนภิกษุทั้งหลาย ภิกษุในธรรมวินัยนี้\\
\textbf{อะรัญญะคะโต วา}\\
\indent ไปแล้วสู่ป่าก็ตาม\\
\textbf{รุกขะมูละคะโต วา}\\
\indent ไปแล้วสู่โคนไม้ก็ตาม\\
\textbf{สุญญาคาระคะโต วา}\\
\indent ไปแล้วสู่เรือนว่างก็ตาม\\
\textbf{นิสีทะติ ปัลลังกัง อาภุชิตวา}\\
\indent นั่งคู้ขาเข้ามาโดยรอบแล้ว\\
\textbf{อุชุง กายัง ปะณิธายะ, ปะริมุขัง สะติง อุปัฏฐะเปตวา}\\
\indent ตั้งกายตรง ดำรงสติมั่น\\
\textbf{โส สะโต วะ อัสสะสะติ, สะโต ปัสสะสะติ}\\
\indent ภิกษุนั้น เป็นผู้มีสติอยู่นั้นเทียว หายใจออก มีสติอยู่หายใจเข้า\\
\\
\textbf{(1) ทีฆัง วา อัสสะสันโต, ทีฆัง อัสสะสามีติ ปะชานาติ}\\
\indent ภิกษุนั้น เมื่อหายใจออกยาว ก็รู้สึกตัวทั่วถึงว่า เราหายใจออกยาว ดังนี้\\
\textbf{ทีฆัง วา ปัสสะสันโต, ทีฆัง ปัสสะสามีติ ปะชานาติ}\\
\indent เมื่อหายใจเข้ายาว ก็รู้สึกตัวทั่วถึงว่า เราหายใจเข้ายาว ดังนี้\\
\\
\textbf{(2) รัสสัง วา อัสสะสันโต, รัสสัง อัสสะสามีติ ปะชานาติ}\\
\indent ภิกษุนั้น เมื่อหายใจออกสั้น ก็รู้สึกตัวทั่วถึงว่า เราหายใจออกสั้น ดังนี้\\
\textbf{รัสสัง วา ปัสสะสันโต, รัสสัง ปัสสะสามีติ ปะชานาติ}\\
\indent เมื่อหายใจเข้าสั้น ก็รู้สึกตัวทั่วถึงว่า เราหายใจเข้าสั้น ดังนี้\\
\\
\textbf{(3) สัพพะกายะปะฏิสังเวที อัสสะสิสสามีติ สิกขะติ}\\
\indent ภิกษุนั้น ย่อมทำในบทศึกษาว่า เราเป็นผู้รู้พร้อมเฉพาะซึ่งกายทั้งปวง จักหายใจออก ดังนี้\\
\textbf{สัพพะกายะปฏิสังเวที ปัสสะสิสสามีติ สิกขะติฯ}\\
\indent ย่อมทำในบทศึกษาว่า เราเป็นผู้รู้พร้อมเฉพาะซึ่งกายทั้งปวง จักหายใจเข้า ดังนี้\\
\\
\textbf{(4) ปัสสัมภะยัง กายะสังขารัง อัสสะสิสสามีติ สิกขะติ}\\
\indent ภิกษุนั้น ย่อมทำในบทศึกษาว่า เราเป็นผู้ทำกายสังขารให้รำงับอยู่ จักหายใจออก ดังนี้\\
\textbf{ปัสสัมภะยัง กายะสังขารัง ปัสสะสิสสามีติ สิกขะติฯ}\\
\indent ย่อมทำในบทศึกษาว่า เราเป็นผู้ทำกายสังขารให้รำงับอยู่จักหายใจเข้า ดังนี้
\begin{center}
(จบหมวดกาย)
\end{center}
\textbf{(5) ปีติปะฏิสังเวที อัสสะสิสสามีติ สิกขะติ}\\
\indent ภิกษุนั้น ย่อมทำในบทศึกษาว่า เราเป็นผู้รู้พร้อมเฉพาะซึ่งปีติจักหายใจออก ดังนี้\\
\textbf{ปีติปะฏิสังเวที ปัสสะสิสสามีติ สิกขะติฯ}\\
\indent ย่อมทำในบทศึกษาว่า เราเป็นผู้รู้พร้อมเฉพาะซึ่งปีติ จักหายใจเข้า ดังนี้\\
\\
\textbf{(6) สุขะปะฏิสังเวที อัสสะสิสสามีติ สิกขะติ}\\
\indent ภิกษุนั้น ย่อมทำในบทศึกษาว่า เราเป็นผู้รู้พร้อมเฉพาะซึ่งสุขจักหายใจออก ดังนี้\\
\textbf{สุขะปะฏิสังเวที ปัสสะสิสสามีติ สิกขะติฯ}\\
\indent ย่อมทำในบทศึกษาว่า เราเป็นผู้รู้พร้อมเฉพาะซึ่งสุข จักหายใจเข้า ดังนี้\\
\\
\textbf{(7) จิตตะสังขาระปะฏิสังเวที อัสสะสิสสามีติ สิกขะติ}\\
\indent ภิกษุนั้น ย่อมทำในบทศึกษาว่า เราเป็นผู้รู้พร้อมเฉพาะซึ่งจิตตสังขารจักหายใจออก ดังนี้\\
\textbf{จิตตะสังขาระปะฏิสังเวที ปัสสะสิสสามีติ สิกขะติฯ}\\
\indent ย่อมทำในบทศึกษาว่า เราเป็นผู้รู้พร้อมเฉพาะซึ่งจิตตสังขารจักหายใจเข้า ดังนี้\\
\\
\textbf{(8) ปัสสัมภะยัง จิตตะสังขารัง อัสสะสิสสามีติ สิกขะติ}\\
\indent ภิกษุนั้น ย่อมทำในบทศึกษาว่า เราเป็นผู้ทำจิตตสังขารให้รำงับอยู่จักหายใจออก ดังนี้\\
\textbf{ปัสสัมภะยัง จิตตะสังขารัง ปัสสะสิสสามีติ สิกขะติฯ}\\
\indent ย่อมทำในบทศึกษาว่า เราเป็นผู้ทำจิตตสังขารให้รำงับอยู่ จักหายใจเข้า ดังนี้
\begin{center}
(จบหมวดเวทนา)
\end{center}
\textbf{(9) จิตตะปะฏิสังเวที อัสสะสิสสามีติ สิกขะติ}\\
\indent ภิกษุนั้น ย่อมทำในบทศึกษาว่า เราเป็นผู้รู้พร้อมเฉพาะซึ่งจิตจักหายใจออก ดังนี้\\
\textbf{จิตตะปะฏิสังเวที ปัสสะสิสสามีติ สิกขะติฯ}\\
\indent ย่อมทำในบทศึกษาว่า เราเป็นผู้รู้พร้อมเฉพาะซึ่งจิต จักหายใจเข้า ดังนี้\\
\\
\textbf{(10) อะภิปปะโมทะยัง จิตตัง อัสสะสิสสามีติ สิกขะติฯ}\\
\indent ภิกษุนั้น ย่อมทำในบทศึกษาว่า เราเป็นผู้ทำจิตให้ปราโมทย์ยิ่งอยู่จักหายใจออก ดังนี้\\
\textbf{อะภิปปะโมทะยัง จิตตัง ปัสสะสิสสามีติ สิกขะติฯ}\\
\indent ย่อมทำในบทศึกษาว่า เราเป็นผู้ทำจิตให้ปราโมทย์ยิ่งอยู่ จักหายใจเข้า ดังนี้\\
\\
\textbf{(11) สะมาทะหัง จิตตัง อัสสะสิสสามีติ สิกขะติ}\\
\indent ภิกษุนั้น ย่อมทำในบทศึกษาว่า เราเป็นผู้ทำจิตให้ตั้งมั่นอยู่จักหายใจออก ดังนี้\\
\textbf{สะมาทะหัง จิตตัง ปัสสะสิสสามีติ สิกขะติฯ}\\
\indent ย่อมทำในบทศึกษาว่า เราเป็นผู้ทำจิตให้ตั้งมั่นอยู่ จักหายใจเข้า ดังนี้\\
\\
\textbf{(12) วิโมจะยัง จิตตัง อัสสะสิสสามีติ สิกขะติ}\\
\indent ภิกษุนั้น ย่อมทำในบทศึกษาว่า เราเป็นผู้ทำจิตให้ปล่อยอยู่จักหายใจออก ดังนี้\\
\textbf{วิโมจะยัง จิตตัง ปัสสะสิสสามีติ สิกขะติฯ}\\
\indent ย่อมทำในบทศึกษาว่า เราเป็นผู้ทำจิตให้ปล่อยอยู่ จักหายใจเข้า ดังนี้
\begin{center}
(จบหมวดจิต)
\end{center}
\textbf{(13) อะนิจจานุปัสสี อัสสะสิสสามีติ สิกขะติ}\\
\indent ภิกษุนั้น ย่อมทำในบทศึกษาว่า เราเป็นผู้ตามเห็น\\
\indent ซึ่งความไม่เที่ยงอยู่เป็นประจำ จักหายใจออก ดังนี้\\
\textbf{อะนิจจานุปัสสี ปัสสะสิสสามีติ สิกขะติฯ}\\
\indent ย่อมทำในบทศึกษาว่า เราเป็นผู้ตามเห็น\\
\indent ซึ่งความไม่เที่ยงอยู่เป็นประจำ จักหายใจเข้า ดังนี้\\
\\
\textbf{(14) วิราคานุปัสสี อัสสะสิสสามีติ สิกขะติ}\\
\indent ภิกษุนั้น ย่อมทำในบทศึกษาว่า เราเป็นผู้ตามเห็น\\
\indent ซึ่งความจางคลายอยู่เป็นประจำ จักหายใจออก ดังนี้\\
\textbf{วิราคานุปัสสี ปัสสะสิสสามีติ สิกขะติฯ}\\
\indent ย่อมทำในบทศึกษาว่า เราเป็นผู้ตามเห็นซึ่งความจางคลาย\\
\indent อยู่เป็นประจำ จักหายใจเข้า ดังนี้\\
\\
\textbf{(15) นิโรธานุปัสสี อัสสะสิสสามีติ สิกขะติ}\\
\indent ภิกษุนั้น ย่อมทำในบทศึกษาว่า เราเป็นผู้ตามเห็น\\
\indent ซึ่งความดับไม่เหลืออยู่เป็นประจำ จักหายใจออก ดังนี้\\
\textbf{นิโรธานุปัสสี ปัสสะสิสสามีติ สิกขะติฯ}\\
\indent ย่อมทำในบทศึกษาว่า เราเป็นผู้ตามเห็นซึ่ง\\
\indent ความดับไม่เหลืออยู่เป็นประจำ จักหายใจเข้า ดังนี้\\
\\
\textbf{(16) ปะฏินิสสัคคานุปัสสี อัสสะสิสสามีติ สิกขะติ}\\
\indent ภิกษุนั้น ย่อมทำในบทศึกษาว่า เราเป็นผู้ตามเห็น\\
\indent ซึ่งความสลัดคืนอยู่เป็นประจำ จักหายใจออก ดังนี้\\
\textbf{ปะฏินิสสัคคานุปัสสี ปัสสะสิสสามีติ สิกขะติฯ}\\
\indent ย่อมทำในบทศึกษาว่า เราเป็นผู้ตามเห็น\\
\indent ซึ่งความสลัดคืนอยู่เป็นประจำ จักหายใจเข้า ดังนี้
\begin{center}
(จบหมวดธรรม)
\end{center}
\textbf{เอวัง ภาวิตา โข ภิกขะเว อานาปานสติ, เอวัง พะหุลีกะตา}\\
\indent ดูก่อนภิกษุทั้งหลาย อานาปานสติอันบุคคลเจริญแล้ว ทำให้มากแล้ว อย่างนี้แล\\
\textbf{มะหัปผะลา โหติ มะหานิสังสา}\\
\indent ย่อมมีผลใหญ่ มีอานิสงส์ใหญ่\\

\pagebreak
\section{อริยมรรคมีองค์แปด}
\hrule
\begin{center}
\textbf{(นำ) หันทะ มะยัง อะริยัฏฐังคิกะมัคคะปาฐัง ภะณามะเส}\\
\end{center}
\textbf{อะยะเมวะ อะริโย อัฏฐังคิโก มัคโค}\\
หนทางนี้แล เป็นหนทางอันประเสริฐ ซึ่งประกอบด้วยองแปด\\
\textbf{เสยยะถีทัง}\\
\indent ได้แก่สิ่งเหล่านี้คือ\\
\textbf{สัมมาทิฏฐิ}\\
\indent ความเห็นชอบ\\
\textbf{สัมมาสังกัปโป}\\
\indent ความดำริชอบ\\
\textbf{สัมมาวาจา}\\
\indent ความพูดจาชอบ\\
\textbf{สัมมากัมมันโต}\\
\indent การทำการงานชอบ\\
\textbf{สัมมาอาชีโว}\\
\indent การเลี้ยงชีวิตชอบ\\
\textbf{สัมมาวายาโม}\\
\indent ความพยายามชอบ\\
\textbf{สัมมาสติ}\\
\indent ความระลึกชอบ\\
\textbf{สัมมาสมาธิ}\\
\indent ความตั้งใจมั่นชอบ\\
\\
\textbf{(๑) กะตะมา จะ ภิกขะเว สัมมาทิฏฐิ}\\
\indent ดูก่อนภิกษุทั้งหลาย ความเห็นชอบเป็นอย่างไรเล่า\\
\textbf{ยังโข ภิกขะเว ทุกเขญาณัง}\\
\indent ดูก่อนภิกษุทั้งหลาย ความรู้อันใด เป็นความรู้ในทุกข์\\
\\
\textbf{ทุกขะสะมุทะเย ญาณัง}\\
\indent เป็นความรู้ในเหตุให้เกิดทุกข์\\
\textbf{ทุกขะนิโรเธ ญาณัง}\\
\indent เป็นความรู้ในความดับแห่งทุกข์\\
\textbf{ทุกขะนิโรธะคามินิยา ปะฏิปะทายะ ญาณัง}\\
\indent เป็นความรู้ในทางดำเนินให้ถึงความดับแห่งทุกข์\\
\textbf{อะยัง วุจจะติ ภิขเว สัมมาทิฏฐิ}\\
\indent ดูก่อนภิกษุทั้งหลาย อันนี้เรากล่าวว่า ความเห็นชอบ\\
\\
\textbf{(๒) กะตะโม จะ  ภิกขเว สัมมาสังกัปโป}\\
\indent ดูก่อน ภิกษุทั้งหลาย ความดำริชอบเป็นอย่างไรเล่า\\
\textbf{เนกขัมมะสังกัปโป}\\
\indent ความดำริในการออกจากกาม\\
\textbf{อัพฺยาปาทะสังกัปโป}\\
\indent ความดำริในการไม่มุ่งร้าย\\
\textbf{อะวิหิงสาสังกัปโป}\\
\indent ความดำริในการไม่เบียดเบียน\\
\textbf{อะยัง วุจจะติ ภิกขะเว สัมมาสังกัปโป}\\
\indent ดูก่อนภิกษุทั้งหลาย อันนี้เรากล่าวว่า ความดำริชอบ\\
\\
\textbf{(๓) กะตะมา จะภิกขะเว สัมมาวาจา}\\
\indent ดูก่อนภิกษุทั้งหลาย การพูดจาชอบเป็นอย่างไรเล่า\\
\textbf{มุสาวาทา เวระมะณี}\\
\indent เจตนาเป็นเครื่องเว้นจากการพูดไม่จริง\\
\textbf{ปิสุณายะ วาจายะ เวระมะณี}\\
\indent เจตนาเป็นเครื่องเว้นจากการพูดส่อเสียด\\
\textbf{ผะรุสายะ วาจายะ เวระมะณี}\\
\indent เจตนาเป็นเครื่องเว้นจากการพูดหยาบ\\
\\
\textbf{สัมผัปปะลาปา เวระมะณี}\\
\indent เจตนาเป็นเครื่องเว้นจากการพูดเพ้อเจ้อ\\
\textbf{อะยัง วุจจะติ ภิกขะเว สัมมาวาจา}\\
\indent ดูก่อนภิกษุทั้งหลาย อันนี้เรากล่าวว่า การพูดจาชอบ\\
\\
\textbf{(๔) กะตะโม จะ ภิกขะเว สัมมากัมมันโต}\\
\indent ดูก่อนภิกษุทั้งหลายการทำการงานชอบเป็นอย่างไรเล่า\\
\textbf{ปาณาติปาตา เวระมะณี}\\
\indent เจตนาเป็นเครื่องงดเว้นจากการฆ่า\\
\textbf{อะทินนาทานา เวระมะณี}\\
\indent เจตนาเป็นเครื่องงดเว้นจากการถือเอาสิ่งของที่ผู้อื่นเขาไม่ได้ให้\\
\textbf{กาเมสุมิจฉาจารา เวระมะณี}\\
\indent เจตนาเป็นเครื่องงดเว้นจากการประพฤติผิดในกามทั้งหลาย\\
\textbf{อะยัง วุจจะติ ภิกขะเว สัมมากัมมันโต}\\
\indent ดูก่อนภิกษุทั้งหลาย อันนี้เรากล่าวว่า การทำการงานชอบ\\
\\
\textbf{(๕) กะตะโม จะ ภิกขะเว สัมมาอาชีโว}\\
\indent ดูก่อนภิกษุทั้งหลาย การเลี้ยงชีวิตชอบเป็นอย่างไรเล่า\\
\textbf{อิธะ ภิกขะเว อะริยะสาวะโก}\\
\indent ดูก่อนภิกษุทั้งหลาย สาวกของพระอริยเจ้าในธรรมวินัยนี้\\
\textbf{มิจฉาอาชีวัง ปะหายะ}\\
\indent เว้นจากการเลี้ยงชีวิตที่ผิดเสีย\\
\textbf{สัมมาอาชีเวนะ ชีวิกัง กัปเปติ}\\
\indent ย่อมสำเร็จความเป็นอยู่ด้วยการเลี้ยงชีวิตที่ชอบ\\
\textbf{อยัง วุจจะติ ภิกขะเว สัมมาอาชีโว}\\
\indent ดูก่อนภิกษุทั้งหลาย อันนี้เรากล่าวว่า การเลี้ยงชีวิตชอบ\\
\\
\textbf{(๖) กะตะโม จะ ภิกขะเว สัมมาวายาโม}\\
\indent ดูก่อนภิกษุทั้งหลาย ความพากเพียรชอบเป็นอย่างไรเล่า\\
\textbf{อิธะ ภิกขะเว ภิกขุ}\\
\indent ดูก่อนภิกษุทั้งหลาย ภิกษุในธรรมวินัยนี้\\
\textbf{อะนุปปันนานัง ปาปะกานัง  อะกุสะลานัง ธัมมานัง อะนุปปาทายะ 
ฉันทัง ชะเนติ วายะมะติ วิริยัง อาระภะติ จิตตัง ปัคคัณหาติ ปะทะหะติ}\\
\indent ย่อมทำความพอใจให้เกิดขึ้น ย่อมพยายาม ปรารภความเพียร \\
\indent ประคองตั้งจิตไว้ เพื่อจะยังอกุศลธรรมอันเป็นบาปที่ยังไม่เกิดไม่ให้เกิดขึ้น\\
\textbf{อุปปันนานัง ปาปะกานัง อะกุสะลานัง ธัมมานัง ปะหานายะ 
ฉันทัง ชะเนติ วายะมะติ วิริยัง อาระภะติ จิตตัง ปัคคัณหาติ ปะทะหะติ}\\
\indent ย่อมทำความพอใจให้เกิดขึ้น ย่อมพยายาม ปรารภความเพียร \\
\indent ประคองตั้งจิตไว้ เพื่อจะละอกุศลธรรมอันเป็นบาปที่เกิดขึ้นแล้ว\\
\textbf{อะนุปปันนานัง กุสะลานัง ธัมมานัง อุปปาทายะ 
ฉันทัง ชะเนติ วายะมะติ วิริยัง อาระภะติ จิตตัง ปัคคัณหาติ ปะทะหะติ}\\
\indent ย่อมทำความพอใจให้เกิดขึ้น ย่อพยายาม ปรารภความเพียร \\
\indent ประคองตั้งจิตไว้ เพื่อจะยังกุศลกรรมที่ยังไม่เกิดให้เกิดขึ้น\\
\textbf{อุปปันนานัง กุสะลานัง ธัมมานัง ฐิติยา อะสัมโมสายะ 
ภิยะโยภาวายะ เวปุลายะ ภาวะนายะ ปาริปูริยา 
ฉันทัง ชะเนติ วายะมะติ วิริยัง อาระภะติ จิตตัง ปัคคัณหาติ ปะทะหะติ }\\
\indent ย่อมทำความพอใจให้เกิดขึ้น ย่อมพยายาม ปรารภความเพียร \\
\indent ประคองตั้งจิตไว้ เพื่อความตั้งอยู่ ความไม่เลอะเลือน ความงอกงามยิ่งขึ้น \\
\indent ความไพบูลย์ ความเจริญ ความเต็มรอบ แห่งกุศลธรรมที่เกิดขึ้นแล้ว\\
\textbf{อะยัง วุจจติ ภิกขะเว สัมมาวายาโม}\\
\indent ดูก่อนภิกษุทั้งหลาย อันนี้เรากล่าวว่า ความพากเพียรชอบ\\
\\
\textbf{(๗) กะตะมา จะ ภิกขะเว สัมมาสะติ}\\
\indent ดูก่อนภิกษุทั้งหลาย, ความระลึกชอบเป็นอย่างไรเล่า\\
\textbf{อิธะ ภิกขะเว ภิกขุ}\\
\indent ดูก่อนภิกษุทั้งหลาย ภิกษุในธรรมวินัยนี้\\
\textbf{กาเย กายานุปัสสี วิหะระติ}\\
\indent ย่อมเป็นผู้พิจารณาเห็นกายในกายอยู่เป็นประจำ\\
\textbf{อาตาปี สัมปะชาโน สะติมา วิเนยยะ โลเก อภิชฌา โทมะนัสสัง}\\
\indent มีความเพียรเครื่องเผากิเลส มีสัมปชัญญะมีสติ \\
\indent ถอนความพอใจ และความไม่พอใจในโลกออกเสียได้\\
\textbf{เวทนาสุ เวทนานุปสฺสี วิหะระติ}\\
\indent ย่อมเป็นผู้พิจารณาเห็นเวทนาในเวทนาทั้งหลายอยู่เป็นประจำ\\
\textbf{อาตาปี สัมปะชาโน สะติมา วิเนยยะ โลเก อะภิชฌา โทมะนัสสัง}\\
\indent มีความเพียรเครื่องเผากิเลส มีสัมปะชัญญะมีสติ \\
\indent ถอนความพอใจและความไม่พอใจในโลกออกเสียได้\\
\textbf{จิตเต จิตตานุปัสสี วิหะระติ}\\
\indent ย่อมเป็นผู้พิจารณาเห็นจิตในจิตอยู่เป็นประจำ\\
\textbf{อาตาปี สัมปะชาโน สะติมา วิเนยยะ โลเก อะภิชฌา โทมะนัสสัง}\\
\indent มีความเพียรเครื่องเผากิเลส มีสัมปะชัญญะมีสติ \\
\indent ถอนความพอใจและความไม่พอใจในโลกออกเสียได้\\
\textbf{ธัมเมสุ ธัมมานุปัสสี วิหะระติ}\\
\indent ย่อมเป็นผู้พิจารณาเห็นธรรมในธรรมทั้งหลายอยู่เป็นประจำ\\
\textbf{อาตาปี สัมปะชาโน สะติมา วิเนยฺยะ โลเก อะภิชฌา โทมะนัสสัง}\\
\indent มีความเพียรเครื่องเผากิเลส มีสัมปะชัญญะมีสติ \\
\indent ถอนความพอใจและความไม่พอใจในโลกออกเสียได้\\
\textbf{อะยัง วุจจะติ ภิกขะเว สัมมาสะติ}\\
\indent ดูก่อนภิกษุทั้งหลาย อันนี้เรากล่าวว่า ความระลึกชอบ\\
\\
\textbf{(๘) กะตะโม จะ ภิกขะเว สัมมาสะมาธิ}\\
\indent ดูก่อนภิกษุ ทั้งหลายความตั้งใจมั่นชอบเป็นอย่างไรเล่า\\
\textbf{อิธะ ภิกขะเว ภิกขุ}\\
\indent ดูก่อนภิกษุทั้งหลาย ภิกษุในธรรมวินัยนี้\\
\textbf{วิวิจเจวะ กาเมหิ}\\
\indent สงัดแล้วจากกามทั้งหลาย\\
\textbf{วิวิจ จะ อะกุสะเลหิ ธัมเมหิ}\\
\indent สงัดแล้วจากธรรมที่เป็นอกุศลทั้งหลาย\\
\textbf{สะวิตักกัง สะวิจารัง วิเวกะชัง ปีติสุขัง ปะฐะมัง ฌานัง อุปะสัมปัชชะ วิหะระติ}\\
\indent เข้าถึงปฐมฌาณ ประกอบด้วยวิตกวิจาร มีปีติ และสุขอันเกิดจากวิเวกแล้วแลอยู่\\
\textbf{วิตักกะ วิจารานัง วูปะสะมา}\\
\indent เพราะความที่วิตกวิจารทั้งสองระงับลง\\
\textbf{อัชฌัตตัง สัมปะสาทะนัง เจตะโส เอโกทิภาวัง 
\indent อะวิตักกัง อะวิจารัง สะมาธิชัง ปีติสุขัง ทุติยัง ฌาณัง อุปฺสัมปัชชะ วิหะระติ}\\
\indent เข้าถึงทุติยะฌาณ เป็นเครื่องผ่องใส่แห่งใจในภายใจให้สามาธิเป้นธรรมอันเอกผุดมีขึ้น \\
\indent ไม่มีวิตก ไม่มีวิจาร มีแต่ปีติและสุขอันเกิดจากสมาธิแล้วแลอยู่\\
\textbf{ปีติยา จะ วิราคา}\\
\indent อนึ่ง เพราะความจางคลายไปแห่งปีติ\\
\textbf{อุเปกขะโก จะ วิหะระติ สะโต จะ สัมปะชาโน}\\
\indent ย่อมเป็นผู้อยู่อุเบกขา มีสติและสัมปชัญญะ\\
\textbf{สุขัญจะ กาเยนะ ปะฏิสังเวเทติ}\\
\indent และย่อมเสวยสุขด้วยนามกาย\\
\textbf{ยันตัง อะริยา อาจิกขันติ อุเปกขะโก สะติมา สุขะ วิหารี ติ}\\
\indent ชนิดที่พระอริยเจ้าทั้งหลาย ย่อมกล่าวสรรเสริญผู้นั้นว่า "เป็นผู้อยู่อุเบกขา\\
\indent มีสติอยู่เป็นปกติสุข" ดังนี้\\
\textbf{ตะติยัง ฌาณัง อุปสัมปัชชะ วิหะระติ}\\
\indent เข้าถึงตติยฌาณแล้วแลอยู่\\
\textbf{สุขัสสะ จะ ปะหานา}\\
\indent เพราะละสุขเสียได้\\
\textbf{ทุกขัสสะ จะ ปะหานา}\\
\indent เพราะละทุกข์เสียได้\\
\textbf{ปุพเพวะ โสมะนัสสะโทมะนัสสานัง อัตถังคะมา}\\
\indent เพราะความดับไปแห่งโสมนัส และโทมนัสทั้งสองในกาลก่อน\\
\textbf{อะทุกขะมะสุขัง อุเปกขาสะติปาริสุทธิง จะตุตถังฌานัง อุปะสัมปัชชะ วิหะระติ}\\
\indent เข้าถึง จตุตฌาณ ไม่มีทุกข์ไม่มีสุข มีแต่ความเป็นสติที่เป็นธรรมชาติบริสุทธิ์เพราะอุเบกขา แล้วแลอยู่\\
\textbf{อยัง วุจะติ ภิกขะเว สัมมาสะมาธิ}\\
\indent ดูก่อนภิกษุทั้งหลาย อันนี้เรากล่าวว่า ความตั้งใจมั่นชอบ\\

\pagebreak
\section{ธัมมจักกัปปวัตนสุตตปาฐะ}
\hrule
\textbf{เทวเม ภิกขะเว อันตา}\\
\indent ดูก่อนภิกษุทั้งหลาย   ที่สุดแห่งการกระทำทั้งสองอย่างเหล่านี้มีอยู่\\
\textbf{ปัพพะชิเตนะ นะ เสวิตัพพา}\\
\indent เป็นสิ่งที่บรรพชิตไม่ควรข้องแวะเลย\\
\textbf{โย จายัง กาเมสุ กามะสุขัลลิกานุโยโค}\\
\indent นี้คือการประกอบตนพัวพันอยู่ด้วยคววามใคร่ในกามทั้งหลาย\\
\textbf{หีโน} (อ่านว่า ฮีโน)\\
\indent เป็นของต่ำทราม\\
\textbf{คัมโม}\\
\indent เป็นของชาวบ้าน\\
\textbf{โปถุชชุนิโก}\\
\indent เป็นของคนชั้นปุถุชน\\
\textbf{อะนะริโย}\\
\indent ไม่ใช่ข้อปฏิบัติของพระอริยเจ้า\\
\textbf{อะนัตถะสัญหิโต}\\
\indent ไม่ประกอบด้วยประโยชน์เลย นี้อย่างหนึ่ง\\
\textbf{โย จายัง อัตตะกิละมาถานุโยโค}\\
\indent อีกอย่างหนึ่งคือการประกอบการทรมานตนให้ลำบาก\\
\textbf{ทุกโข}\\
\indent เป็นสิ่งนำมาซึ่งทุกข์\\
\textbf{อะนะริโย}\\
\indent ไม่ใช่ข้อปฏิบัติของพระอริยเจ้า\\
\textbf{อะนัตถะสัญหิโต}\\
\indent ไม่ประกอบด้วยประโยชน์เลย\\
\textbf{เอเต เต ภิกขุเข อุโภ อันเต อนุปะคัมมะ มะฌิมา ปะฏิปะทา}\\
\indent ดูก่อนภิกษุ ข้อปฏิบัติเป็นทางสายกลางไม่เข้าไปหาที่สุดแห่งการกระทำทั้งสองนั้นมีอยู่\\
\textbf{ตะถาคะเตนะ อภิสัมพุทธา}\\
\indent เป็นข้อปฏิบัติที่ตถาคตได้ตรัสรู้เฉพาะแล้ว\\
\textbf{จักขุกรณี}\\
\indent เป็นเครื่องกระทำให้เกิดจักษุ\\
\textbf{ญาณกะระณี}\\
\indent เป็นเครื่องกระทำให้เกิดปัญญา\\
\textbf{อุปะสะมายะ}\\
\indent เพื่อความสงบ\\
\textbf{อะภิญญายะ}\\
\indent เพื่อความรู้ยิ่ง\\
\textbf{สัมโพธายะ}\\
\indent เพื่อความรู้พร้อม\\
\textbf{นิพพานายะ สังวัตตะติ}\\
\indent เป็นไปเพื่อนิพพาน\\
\textbf{กะตะมา จะ สา ภิกขะเว มัชฌิมา ปะฏิปะทา}\\
\indent ดูก่อภิกษุทั้งหลาย ข้อปฏิบัติเป็นทางสายกลางนั้นเป็นอย่างไรเล่า\\
\textbf{อะยะเมวะ อะริโย อัฏฐังคิโก มัคโค}\\
\indent ข้อปฏิบัติอันเป็นทางสายกลางนั้นคือข้อปฏิบัติเป็นหนทางอันประเสริฐ\\
\indent ประกอบด้วยองค์แปดประการนี้เอง\\
\textbf{เสยยะถีทัง}\\
\indent ได้แก่สิ่งเหล่านี้คือ\\
\textbf{สัมมาทิฏฐิ}\\
\indent ความเห็นชอบ\\
\textbf{สัมมาสังกัปโป}\\
\indent ความดำริชอบ\\
\textbf{สัมมาวาจา}\\
\indent การพูดจาชอบ\\
\textbf{สัมมากัมมันโต}\\
\indent การทำการงานชอบ\\
\textbf{สัมมาอาชีโว}\\
\indent การเลียงชีวิตชอบ\\
\textbf{สัมมาวายาโม}\\
\indent การพยายายามชอบ\\
\textbf{สัมมาสติ}\\
\indent ความระลึกชอบ\\
\textbf{สัมมาสมาธิ}\\
\indent ความตั้งใจมั่นชอบ\\
\textbf{อะยัง โข สา ภิกขะเว มัชฌิมา ปะฏิปะทา}\\
\indent ดูก่อนภิกษุทั้งหลายนี้แลคือข้อปฏิบัติอันเป็นทางสายกลาง\\

\pagebreak
\section{ปฏิจจสมุปปาทธัมมะ}
\hrule
\begin{center}
\textbf{(นำ) หันทะ มายัง ปะฏิจจะสะมุปปาทะ ธัมเมสุ อิทัปปัจจะยะตาทิธัมมะปาฐัง ภาณามะเส}
\end{center}
\textbf{กะตะโม จะ ภิกขะเว ปะฏิจจะสะมุปปาโท}\\
\indent ดูก่อนภิกษุทั้งหลาย ก็ปฏิจจสมุปบาท เป็นอย่างไรเล่า\\
\\
\textbf{(๑) ชาติปัจจะยา ภิกขะเว ชะรามะระณัง}\\
\indent ดูก่อนภิกษุทั้งหลาย เพราะชาติเป็นปัจจัย ชรามรณะย่อมมี\\
\textbf{อุปปาทา วา ภิกขะเว ตะถาคะตานัง\\
อะนุปปาทา วา ตะถาคะตานัง}\\
\indent ดูก่อนภิกษุทั้งหลาย\\
\indent เพราะเหตุที่พระตถาคตทั้งหลาย จะบังเกิดขึ้นก็ตาม จะไม่บังเกิดขึ้นก็ตาม\\
\textbf{ฐิตา วะ สา ธาตุ}\\
\indent ธรรมธาตุนั้น ย่อมตั้งอยู่แล้ว นั่นเทียว\\
\textbf{ธัมมัฏฐิตะตา}\\
\indent คือความตั้งอยู่แห่งธรรมดา\\
\textbf{ธัมมะนิยามะตา}\\
\indent คือความเป็นกฎตายตัวแห่งธรรมดา\\
\textbf{อิทัปปัจจะยะตา}\\
\indent คือความที่เมื่อสิ่งนี้ สิ่งนี้ เป็นปัจจัย สิ่งนี้ ๆ จึงเกิดขึ้น\\
\textbf{ตัง ตะถาคะโต อะภิสัมพุชฌะติ อะภิสะเมติ}\\
\indent ตถาคตย่อมรู้พร้อมเฉพาะ ย่อมถึงพร้อมเฉพาะ ซึ่งธรรมธาตุนั้น\\
\textbf{อะภิสัมพุชฌิตฺวา อะภิสะเมตฺวา}\\
\indent ครั้นรู้พร้อมเฉพาะแล้ว ถึงพร้อมเฉพาะแล้ว\\
\textbf{อาจิกขะติ เทเสติ}\\
\indent ย่อมบอก ย่อมแสดง\\
\textbf{ปัญญะเปติ ปัฏฐะเปติ}\\
\indent ย่อมบัญญัติ ย่อมตั้งขึ้นไว้\\
\textbf{วิวะระติ วิภะชะติ}\\
\indent ย่อมเปิดเผย ย่อมจำแนกแจกแจง\\
\textbf{อุตตานีกะโรติ}\\
\indent ย่อมทำให้เป็นเหมือนการหงายของที่คว่ำ\\
\textbf{ปัสสะถาติ จาหะ, ชาติปัจจะยา ภิกขะเว ชะรา มะระณัง}\\
\indent และได้กล่าวแล้วในบัดนี้ว่า \\
\indent ดูก่อนภิกษุทั้งหลาย ท่านทั้งหลายจงมาดู เพราะชาติเป็นปัจจัย ชรา มรณะย่อมมี\\
\textbf{อิติ โข ภิกขเว}\\
\indent ดูก่อนภิกษุทั้งหลาย เพราะเหตุดังนี้แล\\
\textbf{ยาตัตฺระ ตะถะตา}\\
\indent ธรรมธาตุใด ในกรณีนั้น อันเป็นตถตา คือความเป็นอย่างนั้น\\
\textbf{อะวิตะถะตา}\\
\indent เป็นอวิตถตา คือความไม่ผิดไปจากความเป็นอย่างนั้น\\
\textbf{อะนัญญะถะตา}\\
\indent เป็นอนัญญถตา คือความไม่เป็นไปโดยประการอื่น\\
\textbf{อิทัปปัจจะยะตา}\\
\indent เป็นอิทัปปัจจยตา คือความที่เมื่อ สิ่งนี้ สิ่งนี้ เป็นปัจจัย สิ่งนี้ สิ่งนี้จึงเกิดขึ้น\\
\textbf{อะยัง วุจจะติ ภิกขะเว ปะฏิจจะสะมุปปาโท}\\
\indent ดูก่อนภิกษุทั้งหลาย ธรรมนั้น เราเรียกว่า ปฏิจจสมุปบาท\\
\indent คือธรรมอันเป็นธรรมชาติ อาศัยกันแล้วเกิดขึ้น\\
\\
\textbf{(๒) ภะวะปัจจะยา ภิกขะเว ชาติ}\\
\indent ดูก่อนภิกษุทั้งหลาย เพราะภพเป็นปัจจัย ชาติย่อมมี\\
\textbf{อุปปาทา วา ภิกขะเว ตะถาคะตานัง\\
อะนุปปาทา วา ตะถาคะตานัง}\\
\indent ดูก่อนภิกษุทั้งหลาย\\
\indent เพราะเหตุที่พระตถาคตทั้งหลาย จะบังเกิดขึ้นก็ตาม จะไม่บังเกิดขึ้นก็ตาม\\
\textbf{ฐิตา วะ สา ธาตุ}\\
\indent ธรรมธาตุนั้น ย่อมตั้งอยู่แล้ว นั่นเทียว\\
\textbf{ธัมมัฏฐิตะตา}\\
\indent คือความตั้งอยู่แห่งธรรมดา\\
\textbf{ธัมมะนิยามะตา}\\
\indent คือความเป็นกฎตายตัวแห่งธรรมดา\\
\textbf{อิทัปปัจจะยะตา}\\
\indent คือความที่เมื่อสิ่งนี้ สิ่งนี้ เป็นปัจจัย สิ่งนี้ ๆ จึงเกิดขึ้น\\
\textbf{ตัง ตะถาคะโต อะภิสัมพุชฌะติ อะภิสะเมติ}\\
\indent ตถาคตย่อมรู้พร้อมเฉพาะ ย่อมถึงพร้อมเฉพาะ ซึ่งธรรมธาตุนั้น\\
\textbf{อะภิสัมพุชฌิตฺวา อะภิสะเมตฺวา}\\
\indent ครั้นรู้พร้อมเฉพาะแล้ว ถึงพร้อมเฉพาะแล้ว\\
\textbf{อาจิกขะติ เทเสติ}\\
\indent ย่อมบอก ย่อมแสดง\\
\textbf{ปัญญะเปติ ปัฏฐะเปติ}\\
\indent ย่อมบัญญัติ ย่อมตั้งขึ้นไว้\\
\textbf{วิวะระติ วิภะชะติ}\\
\indent ย่อมเปิดเผย ย่อมจำแนกแจกแจง\\
\textbf{อุตตานีกะโรติ}\\
\indent ย่อมทำให้เป็นเหมือนการหงายของที่คว่ำ\\
\textbf{ปัสสะถาติ จาหะ, ชาติปัจจะยา ภิกขะเว ชะรา มะระณัง}\\
\indent และได้กล่าวแล้วในบัดนี้ว่า \\
\indent ดูก่อนภิกษุทั้งหลาย ท่านทั้งหลายจงมาดู เพราะภพเป็นปัจจัย ชาติย่อมี\\
\textbf{อิติ โข ภิกขเว}\\
\indent ดูก่อนภิกษุทั้งหลาย เพราะเหตุดังนี้แล\\
\textbf{ยาตัตฺระ ตะถะตา}\\
\indent ธรรมธาตุใด ในกรณีนั้น อันเป็นตถตา คือความเป็นอย่างนั้น\\
\textbf{อะวิตะถะตา}\\
\indent เป็นอวิตถตา คือความไม่ผิดไปจากความเป็นอย่างนั้น\\
\textbf{อะนัญญะถะตา}\\
\indent เป็นอนัญญถตา คือความไม่เป็นไปโดยประการอื่น\\
\textbf{อิทัปปัจจะยะตา}\\
\indent เป็นอิทัปปัจจยตา คือความที่เมื่อ สิ่งนี้ สิ่งนี้ เป็นปัจจัย สิ่งนี้ สิ่งนี้จึงเกิดขึ้น\\
\textbf{อะยัง วุจจะติ ภิกขะเว ปะฏิจจะสะมุปปาโท}\\
\indent ดูก่อนภิกษุทั้งหลาย ธรรมนั้น เราเรียกว่า ปฏิจจสมุปบาท \\
\indent คือธรรมอันเป็นธรรมชาติ อาศัยกันแล้วเกิดขึ้น\\
\\
\textbf{(๓) อุปาทานะปัจจะยา ภิกขะเว ภะโว}\\
\indent ดูก่อนภิกษุทั้งหลาย เพราะอุปาทานเป็นปัจจัย ภพย่อมมี\\
\textbf{อุปปาทา วา ภิกขะเว ตะถาคะตานัง\\
อะนุปปาทา วา ตะถาคะตานัง}\\
\indent ดูก่อนภิกษุทั้งหลาย\\
\indent เพราะเหตุที่พระตถาคตทั้งหลาย จะบังเกิดขึ้นก็ตาม จะไม่บังเกิดขึ้นก็ตาม\\
\textbf{ฐิตา วะ สา ธาตุ}\\
\indent ธรรมธาตุนั้น ย่อมตั้งอยู่แล้ว นั่นเทียว\\
\textbf{ธัมมัฏฐิตะตา}\\
\indent คือความตั้งอยู่แห่งธรรมดา\\
\textbf{ธัมมะนิยามะตา}\\
\indent คือความเป็นกฎตายตัวแห่งธรรมดา\\
\textbf{อิทัปปัจจะยะตา}\\
\indent คือความที่เมื่อสิ่งนี้ สิ่งนี้ เป็นปัจจัย สิ่งนี้ ๆ จึงเกิดขึ้น\\
\textbf{ตัง ตะถาคะโต อะภิสัมพุชฌะติ อะภิสะเมติ}\\
\indent ตถาคตย่อมรู้พร้อมเฉพาะ ย่อมถึงพร้อมเฉพาะ ซึ่งธรรมธาตุนั้น\\
\textbf{อะภิสัมพุชฌิตฺวา อะภิสะเมตฺวา}\\
\indent ครั้นรู้พร้อมเฉพาะแล้ว ถึงพร้อมเฉพาะแล้ว\\
\textbf{อาจิกขะติ เทเสติ}\\
\indent ย่อมบอก ย่อมแสดง\\
\textbf{ปัญญะเปติ ปัฏฐะเปติ}\\
\indent ย่อมบัญญัติ ย่อมตั้งขึ้นไว้\\
\textbf{วิวะระติ วิภะชะติ}\\
\indent ย่อมเปิดเผย ย่อมจำแนกแจกแจง\\
\textbf{อุตตานีกะโรติ}\\
\indent ย่อมทำให้เป็นเหมือนการหงายของที่คว่ำ\\
\textbf{ปัสสะถาติ จาหะ, อุปาทานะปัจจะยา ภิกขะเว ภะโว}\\
\indent และได้กล่าวแล้วในบัดนี้ว่า \\
\indent ดูก่อนภิกษุทั้งหลาย ท่านทั้งหลายจงมาดู เพราะอุปาทานเป็นปัจจัย ภพย่อมมี\\
\textbf{อิติ โข ภิกขเว}\\
\indent ดูก่อนภิกษุทั้งหลาย เพราะเหตุดังนี้แล\\
\textbf{ยาตัตฺระ ตะถะตา}\\
\indent ธรรมธาตุใด ในกรณีนั้น อันเป็นตถตา คือความเป็นอย่างนั้น\\
\textbf{อะวิตะถะตา}\\
\indent เป็นอวิตถตา คือความไม่ผิดไปจากความเป็นอย่างนั้น\\
\textbf{อะนัญญะถะตา}\\
\indent เป็นอนัญญถตา คือความไม่เป็นไปโดยประการอื่น\\
\textbf{อิทัปปัจจะยะตา}\\
\indent เป็นอิทัปปัจจยตา คือความที่เมื่อ สิ่งนี้ สิ่งนี้ เป็นปัจจัย สิ่งนี้ สิ่งนี้จึงเกิดขึ้น\\
\textbf{อะยัง วุจจะติ ภิกขะเว ปะฏิจจะสะมุปปาโท}\\
\indent ดูก่อนภิกษุทั้งหลาย ธรรมนั้น เราเรียกว่า ปฏิจจสมุปบาท \\
\indent คือธรรมอันเป็นธรรมชาติ อาศัยกันแล้วเกิดขึ้น\\
\\
\textbf{(๔) ตัณหาปัจจะยา ภิกขะเว อุปาทานัง}\\
\indent ดูก่อนภิกษุทั้งหลาย เพราะตัณหาเป็นปัจจัย อุปาทานย่อมมี\\
\textbf{อุปปาทา วา ภิกขะเว ตะถาคะตานัง\\
อะนุปปาทา วา ตะถาคะตานัง}\\
\indent ดูก่อนภิกษุทั้งหลาย\\
\indent เพราะเหตุที่พระตถาคตทั้งหลาย จะบังเกิดขึ้นก็ตาม จะไม่บังเกิดขึ้นก็ตาม\\
\textbf{ฐิตา วะ สา ธาตุ}\\
\indent ธรรมธาตุนั้น ย่อมตั้งอยู่แล้ว นั่นเทียว\\
\textbf{ธัมมัฏฐิตะตา}\\
\indent คือความตั้งอยู่แห่งธรรมดา\\
\textbf{ธัมมะนิยามะตา}\\
\indent คือความเป็นกฎตายตัวแห่งธรรมดา\\
\textbf{อิทัปปัจจะยะตา}\\
\indent คือความที่เมื่อสิ่งนี้ สิ่งนี้ เป็นปัจจัย สิ่งนี้ ๆ จึงเกิดขึ้น\\
\textbf{ตัง ตะถาคะโต อะภิสัมพุชฌะติ อะภิสะเมติ}\\
\indent ตถาคตย่อมรู้พร้อมเฉพาะ ย่อมถึงพร้อมเฉพาะ ซึ่งธรรมธาตุนั้น\\
\textbf{อะภิสัมพุชฌิตฺวา อะภิสะเมตฺวา}\\
\indent ครั้นรู้พร้อมเฉพาะแล้ว ถึงพร้อมเฉพาะแล้ว\\
\textbf{อาจิกขะติ เทเสติ}\\
\indent ย่อมบอก ย่อมแสดง\\
\textbf{ปัญญะเปติ ปัฏฐะเปติ}\\
\indent ย่อมบัญญัติ ย่อมตั้งขึ้นไว้\\
\textbf{วิวะระติ วิภะชะติ}\\
\indent ย่อมเปิดเผย ย่อมจำแนกแจกแจง\\
\textbf{อุตตานีกะโรติ}\\
\indent ย่อมทำให้เป็นเหมือนการหงายของที่คว่ำ\\
\textbf{ปัสสะถาติ จาหะ, ตัณหาปัจจะยา ภิกขะเว อุปาทานัง}\\
\indent และได้กล่าวแล้วในบัดนี้ว่า \\
\indent ดูก่อนภิกษุทั้งหลาย ท่านทั้งหลายจงมาดู เพราะตัณหาเป็นปัจจัย อุปาทานย่อมมี\\
\textbf{อิติ โข ภิกขเว}\\
\indent ดูก่อนภิกษุทั้งหลาย เพราะเหตุดังนี้แล\\
\textbf{ยาตัตฺระ ตะถะตา}\\
\indent ธรรมธาตุใด ในกรณีนั้น อันเป็นตถตา คือความเป็นอย่างนั้น\\
\textbf{อะวิตะถะตา}\\
\indent เป็นอวิตถตา คือความไม่ผิดไปจากความเป็นอย่างนั้น\\
\textbf{อะนัญญะถะตา}\\
\indent เป็นอนัญญถตา คือความไม่เป็นไปโดยประการอื่น\\
\textbf{อิทัปปัจจะยะตา}\\
\indent เป็นอิทัปปัจจยตา คือความที่เมื่อ สิ่งนี้ สิ่งนี้ เป็นปัจจัย สิ่งนี้ สิ่งนี้จึงเกิดขึ้น\\
\textbf{อะยัง วุจจะติ ภิกขะเว ปะฏิจจะสะมุปปาโท}\\
\indent ดูก่อนภิกษุทั้งหลาย ธรรมนั้น เราเรียกว่า ปฏิจจสมุปบาท \\
\indent คือธรรมอันเป็นธรรมชาติ อาศัยกันแล้วเกิดขึ้น\\
\\
\textbf{(๕) เวทะนาปัจจะยา ภิกขะเว ตัณหา}\\
\indent ดูก่อนภิกษุทั้งหลาย เพราะเวทนาเป็นปัจจัย ตัณหาย่อมมี\\
\textbf{อุปปาทา วา ภิกขะเว ตะถาคะตานัง\\
อะนุปปาทา วา ตะถาคะตานัง}\\
\indent ดูก่อนภิกษุทั้งหลาย\\
\indent เพราะเหตุที่พระตถาคตทั้งหลาย จะบังเกิดขึ้นก็ตาม จะไม่บังเกิดขึ้นก็ตาม\\
\textbf{ฐิตา วะ สา ธาตุ}\\
\indent ธรรมธาตุนั้น ย่อมตั้งอยู่แล้ว นั่นเทียว\\
\textbf{ธัมมัฏฐิตะตา}\\
\indent คือความตั้งอยู่แห่งธรรมดา\\
\textbf{ธัมมะนิยามะตา}\\
\indent คือความเป็นกฎตายตัวแห่งธรรมดา\\
\textbf{อิทัปปัจจะยะตา}\\
\indent คือความที่เมื่อสิ่งนี้ สิ่งนี้ เป็นปัจจัย สิ่งนี้ ๆ จึงเกิดขึ้น\\
\textbf{ตัง ตะถาคะโต อะภิสัมพุชฌะติ อะภิสะเมติ}\\
\indent ตถาคตย่อมรู้พร้อมเฉพาะ ย่อมถึงพร้อมเฉพาะ ซึ่งธรรมธาตุนั้น\\
\textbf{อะภิสัมพุชฌิตฺวา อะภิสะเมตฺวา}\\
\indent ครั้นรู้พร้อมเฉพาะแล้ว ถึงพร้อมเฉพาะแล้ว\\
\textbf{อาจิกขะติ เทเสติ}\\
\indent ย่อมบอก ย่อมแสดง\\
\textbf{ปัญญะเปติ ปัฏฐะเปติ}\\
\indent ย่อมบัญญัติ ย่อมตั้งขึ้นไว้\\
\textbf{วิวะระติ วิภะชะติ}\\
\indent ย่อมเปิดเผย ย่อมจำแนกแจกแจง\\
\textbf{อุตตานีกะโรติ}\\
\indent ย่อมทำให้เป็นเหมือนการหงายของที่คว่ำ\\
\textbf{ปัสสะถาติ จาหะ, เวทะนาปัจจะยา ภิกขะเว ตัณหา}\\
\indent และได้กล่าวแล้วในบัดนี้ว่า \\
\indent ดูก่อนภิกษุทั้งหลาย ท่านทั้งหลายจงมาดู เพราะเวทนาเป็นปัจจัย ตัณหาย่อมมี\\
\textbf{อิติ โข ภิกขเว}\\
\indent ดูก่อนภิกษุทั้งหลาย เพราะเหตุดังนี้แล\\
\textbf{ยาตัตฺระ ตะถะตา}\\
\indent ธรรมธาตุใด ในกรณีนั้น อันเป็นตถตา คือความเป็นอย่างนั้น\\
\textbf{อะวิตะถะตา}\\
\indent เป็นอวิตถตา คือความไม่ผิดไปจากความเป็นอย่างนั้น\\
\textbf{อะนัญญะถะตา}\\
\indent เป็นอนัญญถตา คือความไม่เป็นไปโดยประการอื่น\\
\textbf{อิทัปปัจจะยะตา}\\
\indent เป็นอิทัปปัจจยตา คือความที่เมื่อ สิ่งนี้ สิ่งนี้ เป็นปัจจัย สิ่งนี้ สิ่งนี้จึงเกิดขึ้น\\
\textbf{อะยัง วุจจะติ ภิกขะเว ปะฏิจจะสะมุปปาโท}\\
\indent ดูก่อนภิกษุทั้งหลาย ธรรมนั้น เราเรียกว่า ปฏิจจสมุปบาท \\
\indent คือธรรมอันเป็นธรรมชาติ อาศัยกันแล้วเกิดขึ้น\\
\\
\textbf{(๖) ผัสสะปัจจะยา ภิกขะเว เวทะนา}\\
\indent ดูก่อนภิกษุทั้งหลาย เพราะผัสสะเป็นปัจจัย เวทนาย่อมมี\\
\textbf{อุปปาทา วา ภิกขะเว ตะถาคะตานัง\\
อะนุปปาทา วา ตะถาคะตานัง}\\
\indent ดูก่อนภิกษุทั้งหลาย\\
\indent เพราะเหตุที่พระตถาคตทั้งหลาย จะบังเกิดขึ้นก็ตาม จะไม่บังเกิดขึ้นก็ตาม\\
\textbf{ฐิตา วะ สา ธาตุ}\\
\indent ธรรมธาตุนั้น ย่อมตั้งอยู่แล้ว นั่นเทียว\\
\textbf{ธัมมัฏฐิตะตา}\\
\indent คือความตั้งอยู่แห่งธรรมดา\\
\textbf{ธัมมะนิยามะตา}\\
\indent คือความเป็นกฎตายตัวแห่งธรรมดา\\
\textbf{อิทัปปัจจะยะตา}\\
\indent คือความที่เมื่อสิ่งนี้ สิ่งนี้ เป็นปัจจัย สิ่งนี้ ๆ จึงเกิดขึ้น\\
\textbf{ตัง ตะถาคะโต อะภิสัมพุชฌะติ อะภิสะเมติ}\\
\indent ตถาคตย่อมรู้พร้อมเฉพาะ ย่อมถึงพร้อมเฉพาะ ซึ่งธรรมธาตุนั้น\\
\textbf{อะภิสัมพุชฌิตฺวา อะภิสะเมตฺวา}\\
\indent ครั้นรู้พร้อมเฉพาะแล้ว ถึงพร้อมเฉพาะแล้ว\\
\textbf{อาจิกขะติ เทเสติ}\\
\indent ย่อมบอก ย่อมแสดง\\
\textbf{ปัญญะเปติ ปัฏฐะเปติ}\\
\indent ย่อมบัญญัติ ย่อมตั้งขึ้นไว้\\
\textbf{วิวะระติ วิภะชะติ}\\
\indent ย่อมเปิดเผย ย่อมจำแนกแจกแจง\\
\textbf{อุตตานีกะโรติ}\\
\indent ย่อมทำให้เป็นเหมือนการหงายของที่คว่ำ\\
\textbf{ปัสสะถาติ จาหะ, ผัสสะปัจจะยา ภิกขะเว เวทะนา}\\
\indent และได้กล่าวแล้วในบัดนี้ว่า \\
\indent ดูก่อนภิกษุทั้งหลาย ท่านทั้งหลายจงมาดู เพราะผัสสะเป็นปัจจัย เวทนาย่อมมี\\
\textbf{อิติ โข ภิกขเว}\\
\indent ดูก่อนภิกษุทั้งหลาย เพราะเหตุดังนี้แล\\
\textbf{ยาตัตฺระ ตะถะตา}\\
\indent ธรรมธาตุใด ในกรณีนั้น อันเป็นตถตา คือความเป็นอย่างนั้น\\
\textbf{อะวิตะถะตา}\\
\indent เป็นอวิตถตา คือความไม่ผิดไปจากความเป็นอย่างนั้น\\
\textbf{อะนัญญะถะตา}\\
\indent เป็นอนัญญถตา คือความไม่เป็นไปโดยประการอื่น\\
\textbf{อิทัปปัจจะยะตา}\\
\indent เป็นอิทัปปัจจยตา คือความที่เมื่อ สิ่งนี้ สิ่งนี้ เป็นปัจจัย สิ่งนี้ สิ่งนี้จึงเกิดขึ้น\\
\textbf{อะยัง วุจจะติ ภิกขะเว ปะฏิจจะสะมุปปาโท}\\
\indent ดูก่อนภิกษุทั้งหลาย ธรรมนั้น เราเรียกว่า ปฏิจจสมุปบาท \\
\indent คือธรรมอันเป็นธรรมชาติ อาศัยกันแล้วเกิดขึ้น\\
\\
\textbf{(๗) สะฬายะตะนะปัจจะยา ภิกขะเว ผัสโส}\\
\indent ดูก่อนภิกษุทั้งหลาย เพราะสฬายตนะเป็นปัจจัย ผัสสะย่อมมี\\
\textbf{อุปปาทา วา ภิกขะเว ตะถาคะตานัง\\
อะนุปปาทา วา ตะถาคะตานัง}\\
\indent ดูก่อนภิกษุทั้งหลาย\\
\indent เพราะเหตุที่พระตถาคตทั้งหลาย จะบังเกิดขึ้นก็ตาม จะไม่บังเกิดขึ้นก็ตาม\\
\textbf{ฐิตา วะ สา ธาตุ}\\
\indent ธรรมธาตุนั้น ย่อมตั้งอยู่แล้ว นั่นเทียว\\
\textbf{ธัมมัฏฐิตะตา}\\
\indent คือความตั้งอยู่แห่งธรรมดา\\
\textbf{ธัมมะนิยามะตา}\\
\indent คือความเป็นกฎตายตัวแห่งธรรมดา\\
\textbf{อิทัปปัจจะยะตา}\\
\indent คือความที่เมื่อสิ่งนี้ สิ่งนี้ เป็นปัจจัย สิ่งนี้ ๆ จึงเกิดขึ้น\\
\textbf{ตัง ตะถาคะโต อะภิสัมพุชฌะติ อะภิสะเมติ}\\
\indent ตถาคตย่อมรู้พร้อมเฉพาะ ย่อมถึงพร้อมเฉพาะ ซึ่งธรรมธาตุนั้น\\
\textbf{อะภิสัมพุชฌิตฺวา อะภิสะเมตฺวา}\\
\indent ครั้นรู้พร้อมเฉพาะแล้ว ถึงพร้อมเฉพาะแล้ว\\
\textbf{อาจิกขะติ เทเสติ}\\
\indent ย่อมบอก ย่อมแสดง\\
\textbf{ปัญญะเปติ ปัฏฐะเปติ}\\
\indent ย่อมบัญญัติ ย่อมตั้งขึ้นไว้\\
\textbf{วิวะระติ วิภะชะติ}\\
\indent ย่อมเปิดเผย ย่อมจำแนกแจกแจง\\
\textbf{อุตตานีกะโรติ}\\
\indent ย่อมทำให้เป็นเหมือนการหงายของที่คว่ำ\\
\textbf{ปัสสะถาติ จาหะ, สะฬายะตะนะปัจจะยา ภิกขะเว ผัสโส}\\
\indent และได้กล่าวแล้วในบัดนี้ว่า \\
\indent ดูก่อนภิกษุทั้งหลาย ท่านทั้งหลายจงมาดู เพราะสฬายตนะเป็นปัจจัย ผัสสะย่อมมี\\
\textbf{อิติ โข ภิกขเว}\\
\indent ดูก่อนภิกษุทั้งหลาย เพราะเหตุดังนี้แล\\
\textbf{ยาตัตฺระ ตะถะตา}\\
\indent ธรรมธาตุใด ในกรณีนั้น อันเป็นตถตา คือความเป็นอย่างนั้น\\
\textbf{อะวิตะถะตา}\\
\indent เป็นอวิตถตา คือความไม่ผิดไปจากความเป็นอย่างนั้น\\
\textbf{อะนัญญะถะตา}\\
\indent เป็นอนัญญถตา คือความไม่เป็นไปโดยประการอื่น\\
\textbf{อิทัปปัจจะยะตา}\\
\indent เป็นอิทัปปัจจยตา คือความที่เมื่อ สิ่งนี้ สิ่งนี้ เป็นปัจจัย สิ่งนี้ สิ่งนี้จึงเกิดขึ้น\\
\textbf{อะยัง วุจจะติ ภิกขะเว ปะฏิจจะสะมุปปาโท}\\
\indent ดูก่อนภิกษุทั้งหลาย ธรรมนั้น เราเรียกว่า ปฏิจจสมุปบาท \\
\indent คือธรรมอันเป็นธรรมชาติ อาศัยกันแล้วเกิดขึ้น\\
\\
\textbf{(๘) นามะรูปะปัจจะยา ภิกขะเว สะฬายะตะนัง}\\
\indent ดูก่อนภิกษุทั้งหลาย เพราะนามรูปเป็นปัจจัย สฬายตนะย่อมมี\\
\textbf{อุปปาทา วา ภิกขะเว ตะถาคะตานัง\\
อะนุปปาทา วา ตะถาคะตานัง}\\
\indent ดูก่อนภิกษุทั้งหลาย\\
\indent เพราะเหตุที่พระตถาคตทั้งหลาย จะบังเกิดขึ้นก็ตาม จะไม่บังเกิดขึ้นก็ตาม\\
\textbf{ฐิตา วะ สา ธาตุ}\\
\indent ธรรมธาตุนั้น ย่อมตั้งอยู่แล้ว นั่นเทียว\\
\textbf{ธัมมัฏฐิตะตา}\\
\indent คือความตั้งอยู่แห่งธรรมดา\\
\textbf{ธัมมะนิยามะตา}\\
\indent คือความเป็นกฎตายตัวแห่งธรรมดา\\
\textbf{อิทัปปัจจะยะตา}\\
\indent คือความที่เมื่อสิ่งนี้ สิ่งนี้ เป็นปัจจัย สิ่งนี้ ๆ จึงเกิดขึ้น\\
\textbf{ตัง ตะถาคะโต อะภิสัมพุชฌะติ อะภิสะเมติ}\\
\indent ตถาคตย่อมรู้พร้อมเฉพาะ ย่อมถึงพร้อมเฉพาะ ซึ่งธรรมธาตุนั้น\\
\textbf{อะภิสัมพุชฌิตฺวา อะภิสะเมตฺวา}\\
\indent ครั้นรู้พร้อมเฉพาะแล้ว ถึงพร้อมเฉพาะแล้ว\\
\textbf{อาจิกขะติ เทเสติ}\\
\indent ย่อมบอก ย่อมแสดง\\
\textbf{ปัญญะเปติ ปัฏฐะเปติ}\\
\indent ย่อมบัญญัติ ย่อมตั้งขึ้นไว้\\
\textbf{วิวะระติ วิภะชะติ}\\
\indent ย่อมเปิดเผย ย่อมจำแนกแจกแจง\\
\textbf{อุตตานีกะโรติ}\\
\indent ย่อมทำให้เป็นเหมือนการหงายของที่คว่ำ\\
\textbf{ปัสสะถาติ จาหะ, นามะรูปะปัจจะยา ภิกขะเว สะฬายะตะนัง}\\
\indent และได้กล่าวแล้วในบัดนี้ว่า \\
\indent ดูก่อนภิกษุทั้งหลาย ท่านทั้งหลายจงมาดู เพราะนามรูปเป็นปัจจัย สฬายตนะย่อมมี\\
\textbf{อิติ โข ภิกขเว}\\
\indent ดูก่อนภิกษุทั้งหลาย เพราะเหตุดังนี้แล\\
\textbf{ยาตัตฺระ ตะถะตา}\\
\indent ธรรมธาตุใด ในกรณีนั้น อันเป็นตถตา คือความเป็นอย่างนั้น\\
\textbf{อะวิตะถะตา}\\
\indent เป็นอวิตถตา คือความไม่ผิดไปจากความเป็นอย่างนั้น\\
\textbf{อะนัญญะถะตา}\\
\indent เป็นอนัญญถตา คือความไม่เป็นไปโดยประการอื่น\\
\textbf{อิทัปปัจจะยะตา}\\
\indent เป็นอิทัปปัจจยตา คือความที่เมื่อ สิ่งนี้ สิ่งนี้ เป็นปัจจัย สิ่งนี้ สิ่งนี้จึงเกิดขึ้น\\
\textbf{อะยัง วุจจะติ ภิกขะเว ปะฏิจจะสะมุปปาโท}\\
\indent ดูก่อนภิกษุทั้งหลาย ธรรมนั้น เราเรียกว่า ปฏิจจสมุปบาท \\
\indent คือธรรมอันเป็นธรรมชาติ อาศัยกันแล้วเกิดขึ้น\\
\\
\textbf{(๙) วิญญาณะปัจจะยา ภิกขะเว นามะรูปัง}\\
\indent ดูก่อนภิกษุทั้งหลาย เพราะวิญญาณเป็นปัจจัย นามรูปย่อมมี\\
\textbf{อุปปาทา วา ภิกขะเว ตะถาคะตานัง\\
อะนุปปาทา วา ตะถาคะตานัง}\\
\indent ดูก่อนภิกษุทั้งหลาย\\
\indent เพราะเหตุที่พระตถาคตทั้งหลาย จะบังเกิดขึ้นก็ตาม จะไม่บังเกิดขึ้นก็ตาม\\
\textbf{ฐิตา วะ สา ธาตุ}\\
\indent ธรรมธาตุนั้น ย่อมตั้งอยู่แล้ว นั่นเทียว\\
\textbf{ธัมมัฏฐิตะตา}\\
\indent คือความตั้งอยู่แห่งธรรมดา\\
\textbf{ธัมมะนิยามะตา}\\
\indent คือความเป็นกฎตายตัวแห่งธรรมดา\\
\textbf{อิทัปปัจจะยะตา}\\
\indent คือความที่เมื่อสิ่งนี้ สิ่งนี้ เป็นปัจจัย สิ่งนี้ ๆ จึงเกิดขึ้น\\
\textbf{ตัง ตะถาคะโต อะภิสัมพุชฌะติ อะภิสะเมติ}\\
\indent ตถาคตย่อมรู้พร้อมเฉพาะ ย่อมถึงพร้อมเฉพาะ ซึ่งธรรมธาตุนั้น\\
\textbf{อะภิสัมพุชฌิตฺวา อะภิสะเมตฺวา}\\
\indent ครั้นรู้พร้อมเฉพาะแล้ว ถึงพร้อมเฉพาะแล้ว\\
\textbf{อาจิกขะติ เทเสติ}\\
\indent ย่อมบอก ย่อมแสดง\\
\textbf{ปัญญะเปติ ปัฏฐะเปติ}\\
\indent ย่อมบัญญัติ ย่อมตั้งขึ้นไว้\\
\textbf{วิวะระติ วิภะชะติ}\\
\indent ย่อมเปิดเผย ย่อมจำแนกแจกแจง\\
\textbf{อุตตานีกะโรติ}\\
\indent ย่อมทำให้เป็นเหมือนการหงายของที่คว่ำ\\
\textbf{ปัสสะถาติ จาหะ, วิญญาณะปัจจะยา ภิกขะเว นามะรูปัง}\\
\indent และได้กล่าวแล้วในบัดนี้ว่า \\
\indent ดูก่อนภิกษุทั้งหลาย ท่านทั้งหลายจงมาดู เพราะวิญญาณเป็นปัจจัย นามรูปย่อมมี\\
\textbf{อิติ โข ภิกขเว}\\
\indent ดูก่อนภิกษุทั้งหลาย เพราะเหตุดังนี้แล\\
\textbf{ยาตัตฺระ ตะถะตา}\\
\indent ธรรมธาตุใด ในกรณีนั้น อันเป็นตถตา คือความเป็นอย่างนั้น\\
\textbf{อะวิตะถะตา}\\
\indent เป็นอวิตถตา คือความไม่ผิดไปจากความเป็นอย่างนั้น\\
\textbf{อะนัญญะถะตา}\\
\indent เป็นอนัญญถตา คือความไม่เป็นไปโดยประการอื่น\\
\textbf{อิทัปปัจจะยะตา}\\
\indent เป็นอิทัปปัจจยตา คือความที่เมื่อ สิ่งนี้ สิ่งนี้ เป็นปัจจัย สิ่งนี้ สิ่งนี้จึงเกิดขึ้น\\
\textbf{อะยัง วุจจะติ ภิกขะเว ปะฏิจจะสะมุปปาโท}\\
\indent ดูก่อนภิกษุทั้งหลาย ธรรมนั้น เราเรียกว่า ปฏิจจสมุปบาท \\
\indent คือธรรมอันเป็นธรรมชาติ อาศัยกันแล้วเกิดขึ้น\\
\\
\textbf{(๑๐) สังขาระปัจจะยา ภิกขะเว วิญญาณัง}\\
\indent ดูก่อนภิกษุทั้งหลาย เพราะสังขารเป็นปัจจัย วิญญาณย่อมมี\\
\textbf{อุปปาทา วา ภิกขะเว ตะถาคะตานัง\\
อะนุปปาทา วา ตะถาคะตานัง}\\
\indent ดูก่อนภิกษุทั้งหลาย\\
\indent เพราะเหตุที่พระตถาคตทั้งหลาย จะบังเกิดขึ้นก็ตาม จะไม่บังเกิดขึ้นก็ตาม\\
\textbf{ฐิตา วะ สา ธาตุ}\\
\indent ธรรมธาตุนั้น ย่อมตั้งอยู่แล้ว นั่นเทียว\\
\textbf{ธัมมัฏฐิตะตา}\\
\indent คือความตั้งอยู่แห่งธรรมดา\\
\textbf{ธัมมะนิยามะตา}\\
\indent คือความเป็นกฎตายตัวแห่งธรรมดา\\
\textbf{อิทัปปัจจะยะตา}\\
\indent คือความที่เมื่อสิ่งนี้ สิ่งนี้ เป็นปัจจัย สิ่งนี้ ๆ จึงเกิดขึ้น\\
\textbf{ตัง ตะถาคะโต อะภิสัมพุชฌะติ อะภิสะเมติ}\\
\indent ตถาคตย่อมรู้พร้อมเฉพาะ ย่อมถึงพร้อมเฉพาะ ซึ่งธรรมธาตุนั้น\\
\textbf{อะภิสัมพุชฌิตฺวา อะภิสะเมตฺวา}\\
\indent ครั้นรู้พร้อมเฉพาะแล้ว ถึงพร้อมเฉพาะแล้ว\\
\textbf{อาจิกขะติ เทเสติ}\\
\indent ย่อมบอก ย่อมแสดง\\
\textbf{ปัญญะเปติ ปัฏฐะเปติ}\\
\indent ย่อมบัญญัติ ย่อมตั้งขึ้นไว้\\
\textbf{วิวะระติ วิภะชะติ}\\
\indent ย่อมเปิดเผย ย่อมจำแนกแจกแจง\\
\textbf{อุตตานีกะโรติ}\\
\indent ย่อมทำให้เป็นเหมือนการหงายของที่คว่ำ\\
\textbf{ปัสสะถาติ จาหะ, สังขาระปัจจะยา ภิกขะเว วิญญาณัง}\\
\indent และได้กล่าวแล้วในบัดนี้ว่า \\
\indent ดูก่อนภิกษุทั้งหลาย ท่านทั้งหลายจงมาดู เพราะสังขารเป็นปัจจัย วิญญาณย่อมมี\\
\textbf{อิติ โข ภิกขเว}\\
\indent ดูก่อนภิกษุทั้งหลาย เพราะเหตุดังนี้แล\\
\textbf{ยาตัตฺระ ตะถะตา}\\
\indent ธรรมธาตุใด ในกรณีนั้น อันเป็นตถตา คือความเป็นอย่างนั้น\\
\textbf{อะวิตะถะตา}\\
\indent เป็นอวิตถตา คือความไม่ผิดไปจากความเป็นอย่างนั้น\\
\textbf{อะนัญญะถะตา}\\
\indent เป็นอนัญญถตา คือความไม่เป็นไปโดยประการอื่น\\
\textbf{อิทัปปัจจะยะตา}\\
\indent เป็นอิทัปปัจจยตา คือความที่เมื่อ สิ่งนี้ สิ่งนี้ เป็นปัจจัย สิ่งนี้ สิ่งนี้จึงเกิดขึ้น\\
\textbf{อะยัง วุจจะติ ภิกขะเว ปะฏิจจะสะมุปปาโท}\\
\indent ดูก่อนภิกษุทั้งหลาย ธรรมนั้น เราเรียกว่า ปฏิจจสมุปบาท \\
\indent คือธรรมอันเป็นธรรมชาติ อาศัยกันแล้วเกิดขึ้น\\
\\
\textbf{(๑๑) อวิชชาปัจจะยา ภิกขะเว สังขารา}\\
\indent ดูก่อนภิกษุทั้งหลาย เพราะอวิชชาเป็นปัจจัย สังขารย่อมมี\\
\textbf{อุปปาทา วา ภิกขะเว ตะถาคะตานัง\\
อะนุปปาทา วา ตะถาคะตานัง}\\
\indent ดูก่อนภิกษุทั้งหลาย\\
\indent เพราะเหตุที่พระตถาคตทั้งหลาย จะบังเกิดขึ้นก็ตาม จะไม่บังเกิดขึ้นก็ตาม\\
\textbf{ฐิตา วะ สา ธาตุ}\\
\indent ธรรมธาตุนั้น ย่อมตั้งอยู่แล้ว นั่นเทียว\\
\textbf{ธัมมัฏฐิตะตา}\\
\indent คือความตั้งอยู่แห่งธรรมดา\\
\textbf{ธัมมะนิยามะตา}\\
\indent คือความเป็นกฎตายตัวแห่งธรรมดา\\
\textbf{อิทัปปัจจะยะตา}\\
\indent คือความที่เมื่อสิ่งนี้ สิ่งนี้ เป็นปัจจัย สิ่งนี้ ๆ จึงเกิดขึ้น\\
\textbf{ตัง ตะถาคะโต อะภิสัมพุชฌะติ อะภิสะเมติ}\\
\indent ตถาคตย่อมรู้พร้อมเฉพาะ ย่อมถึงพร้อมเฉพาะ ซึ่งธรรมธาตุนั้น\\
\textbf{อะภิสัมพุชฌิตฺวา อะภิสะเมตฺวา}\\
\indent ครั้นรู้พร้อมเฉพาะแล้ว ถึงพร้อมเฉพาะแล้ว\\
\textbf{อาจิกขะติ เทเสติ}\\
\indent ย่อมบอก ย่อมแสดง\\
\textbf{ปัญญะเปติ ปัฏฐะเปติ}\\
\indent ย่อมบัญญัติ ย่อมตั้งขึ้นไว้\\
\textbf{วิวะระติ วิภะชะติ}\\
\indent ย่อมเปิดเผย ย่อมจำแนกแจกแจง\\
\textbf{อุตตานีกะโรติ}\\
\indent ย่อมทำให้เป็นเหมือนการหงายของที่คว่ำ\\
\textbf{ปัสสะถาติ จาหะ, อวิชชาปัจจะยา ภิกขะเว สังขารา}\\
\indent และได้กล่าวแล้วในบัดนี้ว่า \\
\indent ดูก่อนภิกษุทั้งหลาย ท่านทั้งหลายจงมาดู เพราะอวิชชาเป็นปัจจัย สังขารย่อมมี\\
\textbf{อิติ โข ภิกขเว}\\
\indent ดูก่อนภิกษุทั้งหลาย เพราะเหตุดังนี้แล\\
\textbf{ยาตัตฺระ ตะถะตา}\\
\indent ธรรมธาตุใด ในกรณีนั้น อันเป็นตถตา คือความเป็นอย่างนั้น\\
\textbf{อะวิตะถะตา}\\
\indent เป็นอวิตถตา คือความไม่ผิดไปจากความเป็นอย่างนั้น\\
\textbf{อะนัญญะถะตา}\\
\indent เป็นอนัญญถตา คือความไม่เป็นไปโดยประการอื่น\\
\textbf{อิทัปปัจจะยะตา}\\
\indent เป็นอิทัปปัจจยตา คือความที่เมื่อ สิ่งนี้ สิ่งนี้ เป็นปัจจัย สิ่งนี้ สิ่งนี้จึงเกิดขึ้น\\
\textbf{อะยัง วุจจะติ ภิกขะเว ปะฏิจจะสะมุปปาโท}\\
\indent ดูก่อนภิกษุทั้งหลาย ธรรมนั้น เราเรียกว่า ปฏิจจสมุปบาท \\
\indent คือธรรมอันเป็นธรรมชาติ อาศัยกันแล้วเกิดขึ้น\\
\\
\textbf{อิติ}\\
\indent ดังนี้แล

\pagebreak
\section{ธัมมนิยามสูตร}
\begin{center}
\textbf{(หันทะ มะยัง ธัมมะนิยามะสุตตะปาฐัง ภะณามะ เส)}
\end{center}
\textbf{อุปปาทา วา ภิกขะเว ตะถาคะตานัง อะนุปปาทา วา ตะถาคะตานัง}\\
\indent ดูก่อนภิกษุทั้งหลาย เพราะเหตุที่พระตถาคตทั้งหลาย \\
\indent จะบังเกิดขึ้นก็ตาม, จะไม่บังเกิดขึ้นก็ตาม\\
\textbf{ฐิตา วะ สา ธาตุ}\\
\indent ธรรมธาตุนั้น ย่อมตั้งอยู่แล้วนั่นเทียว\\
\textbf{ธัมมัฏฐิตะตา}\\
\indent คือความตั้งอยู่แห่งธรรมดา\\
\textbf{ธัมมะนิยามะตา}\\
\indent คือความเป็นกฏตายตัวแห่งธรรมดา\\
\textbf{สัพเพ สังขารา อะนิจจาติ}\\
\indent ว่าสังขารทั้งหลายทั้งปวงไม่เที่ยง ดังนี้\\
\textbf{ตัง ตะถาคะโต อะภิสัมพุชฌะติ อะภิสะเมติ}\\
\indent ตถาคตย่อมรู้พร้อมเฉพาะ ย่อมถึงพร้อมเฉพาะซึ่งธรรมธาตุนั้น\\
\textbf{อะภิสัมพุชฌิตวา อะภิสะเมตวา}\\
\indent ครั้นรู้พร้อมเฉพาะแล้ว ถึงพร้อมเฉพาะแล้ว\\
\textbf{อาจิกขะติ เทเสติ}\\
\indent ย่อมบอก ย่อมแสดง\\
\textbf{ปัญญะเปติ ปัฏฐะเปติ}\\
\indent ย่อมบัญญัติ ย่อมตั้งขึ้นไว้\\
\textbf{วิวะระติ วิภะชะติ}\\
\indent ย่อมเปิดเผย ย่อมจำแนกแจกแจง\\
\textbf{อุตตานีกะโรติ}\\
\indent ย่อมทำให้เป็นเหมือนการหงายของที่คว่ำ\\
\textbf{สัพเพ สังขารา อะนิจจาติ}\\
\indent ว่าสังขารทั้งหลายทั้งปวงไม่เที่ยง ดังนี้\\

\pagebreak
\section{มลคลสูตร}
\hrule
\begin{center}
\textbf{(หันทะ มะยัง มังคะละสุตตะปาฐัง ภะณามะ เส)}
\end{center}
\textbf{พะหู เทวา มะนุสสา จะ มังคะลานิ อะจินตะยุง \\
อากังขะมานา โสตถานัง พรูหิ มังคะละมุตตะมัง}\\
\indent เทวดาองค์หนึ่งได้กราบทูลถามพระผู้มีพระภาคเจ้าว่า \\
\indent หมู่เทวดาและมนุษย์มากหลาย มุ่งหมายความเจริญก้าวหน้า \\
\indent ได้คิดถึงเรื่องมงคลแล้ว ขอพระองค์ทรงบอกทางมงคลอันสูงสุดเถิด\\
\indent สมเด็จพระผู้มีพระภาคเจ้าทรงตรัสตอบดังนี้ว่า\\
\textbf{อะเสวะนา จะ พาลานัง}\\
\indent การไม่คบคลพาล\\
\textbf{ปัณฑิตานัญจะ เสวะนา}\\
\indent การคบบัณฑิต\\
\textbf{ปูชา จะ ปูชะนียานัง}\\
\indent การบูชาต่อบุคคลที่ควรบูชา\\
\textbf{เอตัมมังคะละมุตตะมัง}\\
\indent กิจสามอย่างนี้เป็นมงคลอันสูงสุด\\
\textbf{ปะฎิรูปะเทสะวาโส จะ}\\
\indent การอยู่ในประเทศอันสมควร\\
\textbf{ปุพเพ จะ กะตะปุญญะตา}\\
\indent การเป็นผู้มีบุญได้ทำไว้ก่อนแล้ว\\
\textbf{อัตตะสัมมาปะณิธิ จะ}\\
\indent การตั้งตนไว้ชอบ\\
\textbf{เอตัมมังคะละมุตตะมัง}\\
\indent กิจสามอย่างนี้เป็นมงคลอันสูงสุด\\
\textbf{พาหุสัจจัญจะ}\\
\indent การเป็นผู้ได้ยินได้ฟังมาก\\
\textbf{สิปปัญจะ}\\
\indent การมีศิลปะวิทยา\\
\textbf{วินะโย จะ สุสิกขิโต}\\
\indent วินัยที่ศึกษาดีแล้ว\\
\textbf{สุภาสิตา จะ ยา วาจา}\\
\indent วาจาที่เป็นสุภาษิต\\
\textbf{เอตัมมังคะละมุตตะมัง}\\
\indent กิจสี่อย่างนี้เป็นมงคลอันสูงสุด\\
\textbf{มาตาปิตุอุปัฎฐานัง}\\
\indent การบำรุงเลี้ยงมารดาบิดา\\
\textbf{ปุตตะทารัสสะ สังคะโห}\\
\indent การสงเคราะห์บุตรและภรรยา\\
\textbf{อะนากุลา จะ กัมมันตา}\\
\indent การงานที่ไม่ยุ่งเหยิงสับสน\\
\textbf{เอตัมมังคะละมุตตะมัง}\\
\indent กิจสามอย่างนี้เป็นมงคลอันสูงสุด\\
\textbf{ทานัญญะ}\\
\indent การบำเพ็ญทาน\\
\textbf{ธัมมะจะริยา จะ}\\
\indent การประพฤติธรรม\\
\textbf{ญาตะกานัญจะ สังคะโห}\\
\indent การสงเคราะห์หมู่ญาติ\\
\textbf{อะนะวัชชานิ กัมมานิ}\\
\indent การงานอันปราศจากโทษ\\
\textbf{เอตัมมังคะละมุตตะมัง}\\
\indent กิจสี่อย่างนี้เป็นมงคลอันสูงสุด\\
\textbf{อาระตี วิระตี ปาปา}\\
\indent การงดเว้นจากบาปกรรม\\
\textbf{มัชชะปานา จะ สัญญะโม}\\
\indent การเหนี่ยวรั้งใจไว้ได้จากการดื่มน้ำเมา\\
\textbf{อัปปะมาโท จะ ธัมเมสุ}\\
\indent การไม่ประมาทในธรรมทั้งหลาย\\
\textbf{เอตัมมังคะละมุตตะมัง}\\
\indent กิจสามอย่างนี้เป็นมงคลอันสูงสุด\\
\textbf{คาระโว จะ}\\
\indent ความเคารพอ่อนน้อม\\
\textbf{นิวาโต จะ}\\
\indent ความถ่อมตัวไม่เย่อหยิ่ง\\
\textbf{สันตุฎฐี จะ}\\
\indent ความสันโดษยินดีในของที่มีอยู่\\
\textbf{กะตัญญุตา}\\
\indent ความเป็นคนกตัญญู\\
\textbf{กาเลนะ ธัมมัสสะวะนัง}\\
\indent การฟังธรรมตามกาล\\
\textbf{เอตัมมังคะละมุตตะมัง}\\
\indent กิจห้าอย่างนี้เป็นมงคลอันสูงสุด\\
\textbf{ขันตี จะ}\\
\indent ความอดทน\\
\textbf{โสวะจัสสะตา}\\
\indent ความเป็นคนว่าง่ายสอนง่าย\\
\textbf{สะมะณานัญจะ ทัสสนัง}\\
\indent การพบเห็นสมณะผู้สงบจากกิเลส\\
\textbf{กาเลนะ ธัมมะสากัจฉา}\\
\indent การสนทนาธรรมตามกาล\\
\textbf{เอตัมมังคะละมุตตะมัง}\\
\indent กิจสี่อย่างนี้เป็นมงคลอันสูงสุด\\
\textbf{ตะโป จะ}\\
\indent ความเพียรเผากิเลส\\
\textbf{พรหมะจะริยัญจะ}\\
\indent การประพฤติพรหมจรรย์\\
\textbf{อะริยะสัจจานะ ทัสสะนัง}\\
\indent การเห็นอริยสัจ\\
\textbf{นิพพานะสัจฉิกิริยา จะ}\\
\indent การทำพระนิพพานให้แจ้ง\\
\textbf{เอตัมมังคะละมุตตะมัง}\\
\indent กิจสี่อย่างนี้เป็นมงคลอันสูงสุด\\
\textbf{ผุฎฐัสสะ โลกะธัมเมหิ จิตตัง ยัสสะ นะ กัมปะติ}\\
\indent จิตของผู้ใดอันโลกธรรมทั้งหลาย ถูกต้องแล้ว ไม่หวั่นไหว\\
\textbf{อะโสกัง}\\
\indent เป็นจิตไม่เศร้าโศก\\
\textbf{วิระชัง}\\
\indent เป็นจิตไร้ธุลีกิเลส\\
\textbf{เขมัง}\\
\indent เป็นจิตอันเกษมศานต์\\
\textbf{เอตัมมังคะละมุตตะมัง}\\
\indent กิจสี่อย่างนี้เป็นมงคลอันสูงสุด\\
\textbf{เอตาทิสานิ กัตตะวานะ สัพพัตถะมะปะราชิตา,\\
สัพพัตถะ โสตถิง คัจฉันติ ตันเตสัง มังคะละมุตตะมันติ}\\
\indent หมู่เทวดาและมนุษย์ทั้งหลาย \\
\indent ได้กระทำมงคลทั้งสามสิบแปดประการเหล่านี้ให้มีในตนแล้ว \\
\indent จึงเป็นผู้ไม่พ้ายแพ้ในที่ทั้งปวง ย่อมถึงความสวัสดีในทุกสถาน \\
\indent ทั้งหมดนี้เป็นมงคล คือเหตุให้ถึงความเจริญก้าวหน้าอันสูงสุด\\ 
\indent ของเทวดาและมนุษย์ทั้งหลายเหล่านั้นโดยแท้\\
\textbf{อิติ}\\
\indent ด้วยประการฉะนี้แล\\

\pagebreak
\vspace*{\fill}
\begin{center}
  \scalebox{3}{\textbf{อนุโมทนาคาถา}}
\end{center}
\vspace{\fill}
\pagebreak

\section{อนุโมทนารัมภคาถา}
\hrule
\textbf{ยะถา วาริวะหา ปูรา ปะริปูเรนติ สาคะรัง}\\
\indent ห้วงน้ำที่เต็มย่อมยังสมุทรสาครให้เต็มได้ฉันใด\\
\textbf{เอวะเมวะ อิโต ทินนัง เปตานัง อุปะกัปปะติ}\\
\indent ทานที่ท่านอุทิศให้แล้วแต่โลกนี้, ย่อมสำเร็จ\\
\indent ประโยชน์แก่ผูที่ละโลกนี้ไปแล้วได้ ฉันนั้น\\
\textbf{อิจฉิตัง ปัตถิตัง ตุมหัง ขิปปะเมวะ สะมิชฌะตุ}\\
\indent ขออิฏฐผลที่ท่านปรารถนาแล้วตั้งใจแล้ว จงสำเร็จโดยฉับพลัน\\
\textbf{สัพเพ ปูเรนตุ สังกัปปา จันโท ปัณณะระโส ยะถา}\\
\indent ขอความดำริทั้งปวงจงเต็มที่ เหมือนพระจันทร์วันเพ็ญ\\
\textbf{มะณิ โชติระโส ยะถา ฯ}\\
\indent เหมือนแก้วมณีอันสว่างไสวควรยินดี ฯ

\pagebreak
\section{สามัญญานุโมทนาคาถา}
\hrule
\textbf{สัพพีติโย วิวัชชันตุ สัพพะโรโค วินัสสะตุ}\\
\indent ความจัญไรทั้งปวงจงบำราศไป โรคทั้งปวง(ของท่าน)จงหาย\\
\textbf{มา เต ภะวัตวันตะราโย สุขี ทีฆายุโก ภะวะ ฯ}\\
\indent อันตรายอย่ามีแก่ท่าน ท่านจงเป็นผู้มีความสุขมีอายุยืน\\
\textbf{อะภิวาทะนะสีลิสสะ นิจจัง วุฑฒาปะจายิโน\\
จัตตาโร ธัมมา วัฑฒันติ  อายุ วัณโณ สุขัง พะลัง ฯ}\\
\indent ธรรมสี่ประการ คือ อายุ วรรณะ สุขะ พละ ย่อมเจริญแก่\\
\indent ผู้มีปรกติไหว้กราบ,มีปรกติอ่อนน้อม (ต่อผู้ใหญ่) เป็นนิตย์ ฯ\\














\end{document}