%!TEX program = xelatex
%!TEX encoding = UTF-8 Unicode 
\documentclass{article}
\usepackage{fontspec}
\defaultfontfeatures{Mapping=tex-text}
\usepackage{xunicode}
\usepackage{xltxtra}
\setmainfont{TH SarabunPSK}
\XeTeXlinebreaklocale 'th_TH' 
\usepackage{amsmath}
\usepackage{amsfonts}
\usepackage{amssymb}
\usepackage{csquotes}
\usepackage{mathtools}
\usepackage{pgfplots}

\usepackage{color}
\definecolor{white}{rgb}{1,1,1}
\definecolor{dkgreen}{rgb}{0,0.6,0}
\definecolor{gray}{rgb}{0.5,0.5,0.5}
\definecolor{mauve}{rgb}{0.58,0,0.82}
\definecolor{cream}{rgb}{1, 0.992, 0.816}
\definecolor{lightyellow}{rgb}{1, 1, 0.875}

\title{หนังสือสวดมนต์แปล}
\date{\today}

\begin{document}
\pagecolor{lightyellow}
\maketitle
\newpage
\tableofcontents

\pagebreak
\section{คำบูชาพระรัตนตรัย}
โย โส ภะคะวา อะระหัง สัมมาสัมพุทโธ\\
\indent พระผู้มีพระภาคเจ้านั้น พระองค์ใด, เป็นพระอรหันต์, 
ดับเพลิงกิเลสเพลิงทุกข์สิ้นเชิง, ตรัสรู้ชอบได้โดยพระองค์เอง\\
ส๎วากขาโต เยนะ ภะคะวะตา ธัมโม \\
\indent พระธรรม เป็นธรรมอันพระผู้มีพระภาคเจ้า พระองค์ใด, ตรัสไว้ดีแล้ว\\
สุปะฏิปันโน ยัสสะ ภะคะวะโต สาวะกะสังโฆ\\
\indent พระสงฆ์สาวกของพระผู้มีพระภาคเจ้า พระองค์ใด, ปฏิบัติดีแล้ว\\
ตัมมะยัง ภะคะวันตัง สะธัมมัง สะสังฆัง\\
อิเมหิ สักกาเรหิ ยะถาระหัง อาโรปิเตหิ อะภิปูชะยามะ\\
\indent ข้าพเจ้าทั้งหลาย, ขอบูชาอย่างยิ่งซึ่งพระผู้มีพระภาคเจ้าพระองค์นั้น
,พร้อมทั้งพระธรรมและพระสงฆ์,
ด้วยเครื่องสักการะทั้งหลายเหล่านี้, อันยกขึ้นตามสมควรแล้วอย่างไร\\
สาธุ โน ภันเต ภะคะวา สุจิระปะรินิพพุโตปิ\\
\indent ข้าแต่พระองค์ผู้เจริญ, พระผู้มีพระภาคเจ้าแม้ปรินิพพานนานแล้ว,
ทรงสร้างคุณอันสำเร็จประโยชน์ไว้แก่ข้าพเจ้าทั้งหลาย.\\
ปัจฉิมาชะนะตานุกัมปะมานะสา\\
\indent ทรงมีพระหฤทัยอนุเคราะห์แก่พวกข้าพเจ้า อันเป็นชนรุ่นหลัง\\
อิเม สักกาเร ทุคคะตะปัณณาการะภูเต ปะฏิคคัณหาตุ\\
\indent ขอพระผู้มีพระภาคเจ้าจงรับเครื่องสักการะ อันเป็นบรรณาการของคน
ยากทั้งหลายเหล่านี้\\
อัมหากัง ทีฆะรัตตัง หิตายะ สุขายะ\\
\indent เพื่อประโยชน์และความสุขแก่ข้าพเจ้าทั้งหลาย ตลอดกาลนาน เทอญฯ\\
อะระหัง สัมมาสัมพุทโธ ภะคะวา\\
\indent พระผู้มีพระภาคเจ้า, เป็นพระอรหันต์, ดับเพลิงกิเลสเพลิงทุกข์สิ้นเชิง,
ตรัสรู้ชอบได้โดยพระองค์เอง\\
พุทธัง ภะคะวันตัง อะภิวาเทมิ\\
\indent ข้าพเจ้าอภิวาทพระผู้มีพระภาคเจ้า, ผู้รู้ ผู้ตื่น ผู้เบิกบาน (กราบ)\\
ส๎วากขาโต ภะคะวะตา ธัมโม \\
\indent พระธรรมเป็นธรรมที่พระผู้มีพระภาคเจ้า, ตรัสไว้ดีแล้ว\\
ธัมมัง นะมัสสามิ\\
\indent ข้าพเจ้านมัสการพระธรรม (กราบ)\\
สุปะฏิปันโน ภะคะวะโต สาวะกะสังโฆ,\\
\indent พระสงฆ์สาวกของพระผู้มีพระภาคเจ้า, ปฏิบัติดีแล้ว\\
สังฆัง นะมามิ.\\
\indent ข้าพเจ้านอบน้อมพระสงฆ์ (กราบ)

\pagebreak
\section{ปุพพภาคนมการ}
\begin{center}
(หันทะ มะยัง พุทธัสสะ ภะคะวะโต ปุพพะภาคะนะมะการัง กะโรมะ เส)\\
เชิญเถิด เราทั้งหลาย ทำความนอบน้อมอันเป็นส่วนเบื้องต้น แด่พระผู้มีพระภาคเจ้าเถิด
\end{center}
นะโม ตัสสะ ภะคะวะโต,\\
\indent ขอนอบน้อมแด่พระผู้มีพระภาคเจ้าพระองค์นั้น\\
อะระหะโต,\\
\indent ซึ่งเป็นผู้ไกลจากกิเลส\\
สัมมาสัมพุทธัสสะ.\\
\indent ตรัสรู้ชอบได้โดยพระองค์เอง.	
\begin{center}
(กล่าว ๓ ครั้ง)
\end{center}

\pagebreak

\section{พุทธาภิถุติ}
\begin{center}
(หันทะ มะยัง พุทธาภิถุติง กะโรมะ เส)\\
เชิญเถิด เราทั้งหลาย ทำความชมเชยเฉพาะพระพุทธเจ้าเถิด
\end{center}
โย โส ตะถาคะโต\newline
\indent พระตถาคตเจ้านั้น พระองค์ใด\\
อะระหัง\\
\indent เป็นผู้ไกลจากกิเลส\\
สัมมาสัมพุทโธ\\
\indent เป็นผู้ตรัสรู้ชอบได้โดยพระองค์เอง\\
วิชชาจะระณะสัมปันโน\\
\indent เป็นผู้ถึงพร้อมด้วยวิชชาและจรณะ\\
สุคะโต\\
\indent เป็นผู้ไปแล้วด้วยดี\\
โลกะวิท\\
\indent เป็นผู้รู้โลกอย่างแจ่มแจ้ง\\
อะนุตตะโร ปุริสะทัมมะสาระถิ\\
\indent เป็นผู้สามารถฝึกบุรุษที่สมควรฝึกได้อย่างไม่มีใครยิ่งกว่า\\
สัตถา เทวะมะนุสสานัง\\
\indent เป็นครูผู้สอนของเทวดาและมนุษย์ทั้งหลาย\\
พุทโธ\\
\indent เป็นผู้รู้ ผู้ตื่น ผู้เบิกบานด้วยธรรม\\
ภะคะวา\\
\indent เป็นผู้มีความจำเริญ จำแนกธรรมสั่งสอนสัตว์\\
โย อิมัง โลกัง สะเทวะกัง สะมาระกัง สะพ๎รัห๎มะกัง, \\
สัสสะมะณะพ๎ราห๎มะณิงปะชัง สะเทวะมะนุสสัง\\
 สะยัง อะภิญญา สัจฉิกัต๎วา ปะเวเทสิ\\
\indent พระผู้มีพระภาคเจ้าพระองค์ใด, ได้ทรงทำความดับทุกข์ให้แจ้ง 
ด้วยพระปัญญาอันยิ่งเองแล้ว, ทรงสอนโลกนี้พร้อมทั้งเทวดา 
มาร พรหม และหมู่สัตว์ พร้อมทั้งสมณพราหมณ, 
พร้อมทั้งเทวดาและมนุษย์ให้รู้ตาม\\
โย ธัมมัง เทเสสิ\\
\indent พระผู้มีพระภาคเจ้าพระองค์ใด ทรงแสดงธรรมแล้ว\\
อาทิกัล๎ยาณัง\\
\indent ไพเราะในเบื้องต้น\\
มัชเฌกัล๎ยาณัง\\
\indent ไพเราะในท่ามกลาง\\
ปะริโยสานะกัล๎ยาณัง\\
\indent ไพเราะในที่สุด\\
สาตถัง สะพ๎ยัญชะนัง เกวะละปะริปุณณัง ปะริสุทธัง พ๎รัห๎มะจะริยัง ปะกาเสสิ\\
\indent ทรงประกาศพรหมจรรย์ คือแบบแห่งการปฏิบัติอันประเสริฐ บริสุทธิ์
บริบูรณ์ สิ้นเชิง,พร้อมทั้งอรรถะ (คำอธิบาย) พร้อมทั้งพยัญชนะ (หัวข้อ)\\
ตะมะหัง ภะคะวันตัง อะภิปูชะยามิ\\
\indent ข้าพเจ้าบูชาอย่างยิ่ง เฉพาะพระผู้มีพระภาคเจ้าพระองค์นั้น\\
ตะมะหัง ภะคะวันตัง สิระสา นะมามิ\\
\indent ข้าพเจ้านอบน้อมพระผู้มีพระภาคเจ้า พระองค์นั้นด้วยเศียรเกล้า
\begin{center}
(กราบระลึกพระพุทธคุณ)
\end{center}
\pagebreak
\section{ธัมมาภิถุติ}
\begin{center}
(หันทะ มะยัง ธัมมาภิถุติง กะโรมะ เส)\\
เชิญเถิด เราทั้งหลาย ทำความชมเชยเฉพาะพระธรรมเถิด
\end{center}
โย โส ส๎วากขาโต ภะคะวะตา ธัมโม\\
\indent พระธรรมนั้นใด, เป็นสิ่งที่พระผู้มีพระภาคเจ้าได้ตรัสไว้ดีแล้ว\\
สันทิฏฐิโก\\
\indent เป็นสิ่งที่ผู้ศึกษาและปฏิบัติพึงเห็นได้ด้วยตนเอง\\
อะกาลิโก\\
\indent เป็นสิ่งที่ปฏิบัติได้ และให้ผลได้ไม่จำกัดกาล\\
เอหิปัสสิโก\\
\indent เป็นสิ่งที่ควรกล่าวกะผู้อื่นว่า ท่านจงมาดูเถิด\\
โอปะนะยิโก\\
\indent เป็นสิ่งที่ควรน้อมเข้ามาใส่ตัว\\
ปัจจัตตัง เวทิตัพโพ วิญญูหิ\\
\indent เป็นสิ่งที่ผู้รู้ก็รู้ได้เฉพาะตน\\
ตะมะหัง ธัมมัง อะภิปูชะยามิ\\
\indent ข้าพเจ้าบูชาอย่างยิ่ง เฉพาะพระธรรมนั้น\\
ตะมะหัง ธัมมัง สิระสา นะมามิ\\
\indent ข้าพเจ้านอบน้อมพระธรรมนั้น ด้วยเศียรเกล้า
\begin{center}
(กราบระลึกพระธรรมคุณ)
\end{center}
\pagebreak
\section{สังฆาภิถุติ}
\begin{center}
(หันทะ มะยัง สังฆาภิถุติง กะโรมะ เส)\\
เชิญเถิด เราทั้งหลาย ทำความชมเชยเฉพาะพระสงฆ์เถิด
\end{center}
โย โส สุปะฏิปันโน ภะคะวะโต สาวะกะสังโฆ\\
\indent สงฆ์สาวกของพระผู้มีพระภาคเจ้านั้นหมู่ใด ปฏิบัติดีแล้ว\\
อุชุปะฏิปันโน ภะคะวะโต สาวะกะสังโฆ\\
\indent สงฆ์สาวกของพระผู้มีพระภาคเจ้าหมู่ใด ปฏิบัติตรงแล้ว\\
ญายะปะฏิปันโน ภะคะวะโต สาวะกะสังโฆ\\
\indent สงฆ์สาวกของพระผู้มีพระภาคเจ้าหมู่ใด, ปฏิบัติเพื่อรู้ธรรมเป็นเครื่องออกจากทุกข์แล้ว\\
สามีจิปะฏิปันโน ภะคะวะโต สาวะกะสังโฆ\\
\indent สงฆ์สาวกของพระผู้มีพระภาคเจ้าหมู่ใด, ปฏิบัติสมควรแล้ว\\
ยะทิทัง\\
\indent ได้แก่บุคคลเหล่านี้คือ\\
จัตตาริ ปุริสะยุคานิ อัฏฐะ ปุริสะปุคคะลา\\
\indent คู่แห่งบุรุษ ๔ คู่, นับเรียงตัวบุรุษได้ ๘ บุรุษ*
\footnote{* สี่คู่คือ โสดาปัตติมรรค โสดาปัตติผล, สกิทาคามิมรรค สกิทาคามิผล,
อนาคามิมรรค อนาคามิผล, อรหัตตมรรค อรหัตตผล.}\\
เอสะ ภะคะวะโต สาวะกะสังโฆ\\
\indent นั่นแหละสงฆ์สาวกของพระผู้มีพระภาคเจ้า\\
อาหุเนยโย\\
\indent เป็นสงฆ์ควรแก่สักการะที่เขานำมาบูชา\\
ปาหุเนยโย\\
\indent เป็นสงฆ์ควรแก่สักการะที่เขาจัดไว้ต้อนรับ\\
ทักขิเณยโย\\
\indent เป็นผู้ควรรับทักษิณาทาน\\
อัญชะลิกะระณีโย\\
\indent เป็นผู้ที่บุคคลทั่วไปควรทำอัญชลี\\
อะนุตตะรัง ปุญญักเขตตัง โลกัสสะ\\
\indent เป็นเนื้อนาบุญของโลก, ไม่มีนาบุญอื่นยิ่งกว่า\\
ตะมะหัง สังฆัง อะภิปูชะยามิ\\
\indent ข้าพเจ้าบูชาอย่างยิ่ง เฉพาะพระสงฆ์หมู่นั้น\\
ตะมะหัง สังฆัง สิระสา นะมามิ\\
\indent ข้าพเจ้านอบน้อมพระสงฆ์หมู่นั้น ด้วยเศียรเกล้า
\begin{center}
(กราบระลึกพระสังฆคุณ)
\end{center}
\pagebreak

\section{รตนัตตยัปปณามคาถา}
\begin{center}
(หันทะ มะยัง ระตะนัตตะยัปปะณามะคาถาโย เจวะสังเวคะวัตถุปะริกิตตะนะปาฐัญจะ ภะณามะ เส)\\เชิญเถิด เราทั้งหลาย กล่าวคำนอบน้อมพระรัตนตรัยและบาลีที่กำหนดวัตถุเครื่องแสดงความสังเวชเถิด
\end{center}
พุทโธ สุสุทโธ กะรุณามะหัณณะโว\\
\indent พระพุทธเจ้าผู้บริสุทธิ์ มีพระกรุณาดุจห้วงมหรรณพ\\
โยจจันตะสุทธัพพะระญาณะโลจะโน\\
\indent พระองค์ใด มีตาคือญาณอันประเสริฐหมดจดถึงที่สุด\\
โลกัสสะ ปาปูปะกิเลสะฆาตะโก\\
\indent เป็นผู้ฆ่าเสียซึ่งบาปและอุปกิเลสของโลก\\
วันทามิ พุทธัง อะหะมาทะเรนะ ตัง\\
\indent ข้าพเจ้าไหว้พระพุทธเจ้าพระองค์นั้น โดยใจเคารพเอื้อเฟื้อ\\
ธัมโม ปะทีโป วิยะ ตัสสะ สัตถุโน\\
\indent พระธรรมของพระศาสดา สว่างรุ่งเรืองเปรียบดวงประทีป\\
โย มัคคะปากามะตะเภทะภินนะโก\\
\indent จำแนกประเภท คือ มรรค ผล นิพพาน, ส่วนใด\\
โลกุตตะโร โย จะ ตะทัตถะทีปะโน\\
\indent ซึ่งเป็นตัวโลกุตตระ, และส่วนใดที่ชี้แนวแห่งโลกุตตระนั้น\\
วันทามิ ธัมมัง อะหะมาทะเรนะ ตัง\\
\indent ข้าพเจ้าไหว้พระธรรมนั้น โดยใจเคารพเอื้อเฟื้อ\\
สังโฆ สุเขตตาภ๎ยะติเขตตะสัญญิโต\\
\indent พระสงฆ์เป็นนาบุญอันยิ่งใหญ่กว่านาบุญอันดีทั้งหลาย\\
โย ทิฏฐะสันโต สุคะตานุโพธะโก\\
\indent เป็นผู้เห็นพระนิพพาน, ตรัสรู้ตามพระสุคต, หมู่ใด\\
โลลัปปะหี\footnote{อ่านว่า ฮี} โน อะริโย สุเมธะโส\\
\indent เป็นผู้ละกิเลสเครื่องโลเลเป็นพระอริยเจ้า มีปัญญาดี\\
วันทามิ สังฆัง อะหะมาทะเรนะ ตัง\\
\indent ข้าพเจ้าไหว้พระสงฆ์หมู่นั้นโดยใจเคารพเอื้อเฟื้อ\\
อิจเจวะเมกันตะภิปูชะเนยยะกัง, วัตถุตตะยัง วันทะยะตาภิสังขะตัง,\\
ปุญญัง มะยา ยัง มะมะ สัพพุปัททะวา,มา โหนตุ เว ตัสสะ ปะภาวะสิทธิยา\\
\indent บุญใดที่ข้าพเจ้าผู้ไหว้อยู่ซึ่งวัตถุสาม, คือพระรัตนตรัยอันควรบูชายิ่งโดยส่วนเดียว,
ได้กระทำแล้วเป็นอย่างยิ่งเช่นนี้, ขออุปัทวะ(ความชั่ว) ทั้งหลาย,
จงอย่ามีแก่ข้าพเจ้าเลย, ด้วยอำนาจความสำเร็จอันเกิดจากบุญนั้น
\pagebreak
\section{สังเวคปริกิตตนปาฐะ}
อิธะ ตะถาคะโต โลเก อุปปันโน\\
\indent พระตถาคตเจ้าเกิดขึ้นแล้วในโลกนี้\\
อะระหัง สัมมาสัมพุทโธ\\
\indent เป็นผู้ไกลจากกิเลส ตรัสรู้ชอบได้โดยพระองค์เอง\\
ธัมโม จะ เทสิโต นิยยานิโก\\
\indent และพระธรรมที่ทรงแสดงเป็นธรรมเครื่องออกจากทุกข์\\
อุปะสะมิโก ปะรินิพพานิโก\\
\indent เป็นเครื่องสงบกิเลส, เป็นไปเพื่อปรินิพพาน\\
สัมโพธะคามี สุคะตัปปะเวทิโต\\
\indent เป็นไปเพื่อความรู้พร้อม, เป็นธรรมที่พระสุคตประกาศ\\
มะยันตัง ธัมมัง สุต๎วา เอวัง ชานามะ\\
\indent พวกเราเมื่อได้ฟังธรรมนั้นแล้ว, จึงได้รู้อย่างนี้ว่า\\
ชาติปิ ทุกขา\\
\indent แม้ความเกิดก็เป็นทุกข์\\
ชะราปิ ทุกขา\\
\indent แม้ความแก่ก็เป็นทุกข์\\
มะระณัมปิ ทุกขัง\\
\indent แม้ความตายก็เป็นทุกข์\\
โสกะปะริเทวะทุกขะโทมะนัสสุปายาสาปิ ทุกขา\\
\indent แม้ความโศก ความร่ำไรรำพัน ความไม่สบายกาย
ความไม่สบายใจ ความคับแค้นใจ ก็เป็นทุกข์\\
อัปปิเยหิ สัมปะโยโค ทุกโข\\
\indent ความประสบกับสิ่งไม่เป็นที่รักที่พอใจ ก็เป็นทุกข์\\
ปิเยหิ วิปปะโยโค ทุกโข\\
\indent ความพลัดพรากจากสิ่งเป็นที่รักที่พอใจ ก็เป็นทุกข์\\
ยัมปิจฉัง นะ ละภะติ ตัมปิ ทุกขัง\\
\indent มีความปรารถนาสิ่งใดไม่ได้สิ่งนั้นนั่นก็เป็นทุกข์\\
สังขิตเตนะ ปัญจุปาทานักขันธา ทุกขา\\
\indent ว่าโดยย่อ อุปาทานขันธ์ทั้ง ๕ เป็นตัวทุกข์\\
เสยยะถีทัง\\
\indent ได้แก่สิ่งเหล่านี้คือ\\
รูปูปาทานักขันโธ\\
\indent ขันธ์ อันเป็นที่ตั้งแห่งความยึดมั่น คือรูป\\
เวทะนูปาทานักขันโธ\\
\indent ขันธ์ อันเป็นที่ตั้งแห่งความยึดมั่น คือเวทนา\\
สัญญูปาทานักขันโธ\\
\indent ขันธ์ อันเป็นที่ตั้งแห่งความยึดมั่น คือสัญญา\\
สังขารูปาทานักขันโธ\\
\indent ขันธ์ อันเป็นที่ตั้งแห่งความยึดมั่น คือสังขาร\\
วิญญาณูปาทานักขันโธ\\
\indent ขันธ์ อันเป็นที่ตั้งแห่งความยึดมั่น คือวิญญาณ\\
เยสัง ปะริญญายะ\\
\indent เพื่อให้สาวกกำหนดรอบรู้อุปาทานขันธ์เหล่านี้เอง\\
ธะระมาโน โส ภะคะวา\\
\indent จึงพระผู้มีพระภาคเจ้านั้น เมื่อยังทรงพระชนม์อยู่\\
เอวัง พะหุลัง สาวะเก วิเนติ\\
\indent ย่อมทรงแนะนำสาวกทั้งหลาย, เช่นนี้เป็นส่วนมาก\\
เอวัง ภาคา จะ ปะนัสสะ ภะคะวะโต สาวะเกสุ อะนุสาสะนี พะหุลา ปะวัตตะติ\\
\indent อนึ่งคำสั่งสอนของพระผู้มีพระภาคเจ้านั้นย่อมเป็นไปในสาวกทั้งหลาย,
ส่วนมากมีส่วนคือการจำแนกอย่างนี้ว่า\\
รูปัง อะนิจจัง\\
\indent รูปไม่เที่ยง\\
เวทะนา อะนิจจา\\
\indent เวทนาไม่เที่ยง\\
สัญญา อะนิจจา\\
\indent สัญญาไม่เที่ยง\\
สังขารา อะนิจจา\\
\indent สังขารไม่เที่ยง\\
วิญญาณัง อะนิจจัง\\
\indent วิญญาณไม่เที่ยง\\
รูปัง อะนัตตา\\
\indent รูปไม่ใช่ตัวตน\\
เวทะนา อะนัตตา\\
\indent เวทนาไม่ใช่ตัวตน\\
สัญญา อะนัตตา\\
\indent สัญญาไม่ใช่ตัวตน\\
สังขารา อะนัตตา\\
\indent สังขารไม่ใช่ตัวตน\\
วิญญาณัง อะนัตตา\\
\indent วิญญาณไม่ใช่ตัวตน\\
สัพเพ สังขารา อะนิจจา\\
\indent สังขารทั้งหลายทั้งปวง ไม่เที่ยง\\
สัพเพ ธัมมา อะนัตตาติ\\
\indent ธรรมทั้งหลายทั้งปวง ไม่ใช่ตัวตน, ดังนี้\\
เต (ตา)  มะยัง โอติณณามหะ\\
\indent พวกเราทั้งหลายเป็นผู้ถูกครอบงำแล้ว\\
ชาติยา\\
\indent โดยความเกิด\\
ชะรามะระเณนะ\\
\indent โดยความแก่และความตาย\\
โสเกหิ ปะริเทเวหิ ทุกเขหิ โทมะนัสเสหิ อุปายาเสหิ\\
\indent โดยความโศก ความร่ำไรรำพัน ความไม่สบายกาย\\
ความไม่สบายใจ ความคับแค้นใจทั้งหลาย\\
ทุกโขติณณา\\
\indent เป็นผู้ถูกความทุกข์หยั่งเอาแล้ว\\
ทุกขะปะเรตา\\
\indent เป็นผู้มีความทุกข์เป็นเบื้องหน้าแล้ว\\
อัปเปวะนามิมัสสะ เกวะลัสสะ ทุกขักขันธัสสะ อันตะกิริยา ปัญญาเยถาติ\\
\indent ทำไฉนการทำที่สุดแห่งกองทุกข์ทั้งสิ้นนี้, จะพึงปรากฏชัดแก่เราได้
\begin{center}สำหรับ พระภิกษุ - สามเณรสวด\end{center}
จิระปะรินิพพุตัมปิ ตัง ภะคะวันตัง อุททิสสะ อะระหันตัง สัมมาสัมพุทธัง\\
\indent เราทั้งหลาย อุทิศเฉพาะพระผู้มีพระภาคเจ้า, ผู้ไกลจากกิเลส
ตรัสรู้ชอบได้โดยพระองค์เอง, แม้ปรินิพพานนานแล้ว, พระองค์นั้น\\
สัทธา อะคารัส๎มา อะนะคาริยัง ปัพพะชิตา\\
\indent เป็นผู้มีศรัทธา ออกบวชจากเรือน ไม่เกี่ยวข้องด้วยเรือนแล้ว\\
ตัส๎มิง ภะคะวะติ พ๎รห๎มะจะริยัง จะรามะ\\
\indent ประพฤติอยู่ซึ่งพรหมจรรย์ ในพระผู้มีพระภาคเจ้า พระองค์นั้น\\
ภิกขูนัง สิกขาสาชีวะสะมาปันนา\\
\indent ถึงพร้อมด้วยสิกขาและธรรมเป็นเครื่องเลี้ยงชีวิตของภิกษุทั้งหลาย\\
ตัง โน พ๎รห๎มะจะริยัง อิมัสสะ เกวะลัสสะ ทุกขักขันธัสสะ อันตะ กิริยายะ สังวัตตะตุ\\
\indent ขอให้พรหมจรรย์ของเราทั้งหลายนั้น, จงเป็นไปเพื่อการทำที่สุดแห่งกอง ทุกข์ทั้งสิ้นนี้ เทอญ
\begin{center}สำหรับอุบาสก, อุบาสิกา\end{center}
จิระปะรินิพพุตัมปิ ตัง ภะคะวันตัง สะระณังคะตา\\
\indent เราทั้งหลาย ผู้ถึงแล้วซึ่งพระผู้มีพระภาคเจ้า,
แม้ปรินิพพานนานแล้วพระองค์นั้น เป็นสรณะ\\
ธัมมัญจะ สังฆัญจะ\\
\indent ถึงพระธรรมด้วย ถึงพระสงฆ์ด้วย\\
ตัสสะ ภะคะวะโต สาสะนัง, ยะถาสะติ, ยะถาพะลัง\\
มะนะสิกะโรมะ อะนุปะฏิปัชชามะ\\
\indent จักทำในใจอยู่ ปฏิบัติตามอยู่ ซึ่งคำสั่งสอนของพระผู้มีพระภาคเจ้านั้น
ตามสติกำลัง\\
สา สา โน ปะฏิปัตติ\\
\indent ขอให้ความปฏิบัตินั้นๆ ของเราทั้งหลาย\\
อิมัสสะ เกวะลัสสะ ทุกขักขันธัสสะ อันตะกิริยายะ สังวัตตะตุ\\
\indent จงเป็นไปเพื่อการทำที่สุดแห่งกองทุกข์ทั้งสิ้นนี้ เทอญ\\
\begin{center}
(จบคำทำวัตรเช้า)
\end{center}
\pagebreak
\section{คำแผ่เมตตา}
\begin{center}(หันทะ มะยัง เมตตาผะระณัง กะโรมะ เส)\end{center}
อะหัง สุขิโต โหมิ\\
\indent ขอให้ข้าพเจ้าจงเป็นผู้ถึงสุข\\
นิททุกโข โหมิ\\
\indent จงเป็นผู้ไร้ทุกข์\\
อะเวโร โหมิ\\
\indent จงเป็นผู้ไม่มีเวร\\
อัพฺยาปัชโฌ โหมิ \\
\indent จงเป็นผู้ไม่เบียดเบียนซึ่งกันและกัน\\
อะนีโฆ โหมิ\\
\indent จงเป็นผู้ไม่มีทุกข์\\
สุขี อัตตานัง ปะริหะรามิ\\
\indent จงรักษาตนอยู่เป็นสุขเถิด\\
สัพเพ สัตตา สุขิตา โหนตุ\\
\indent ขอสัตว์ทั้งหลายทั้งปวงจงเป็นผู้ถึงความสุข\\
สัพเพ สัตตา อะเวรา โหนตุ\\
\indent ขอสัตว์ทั้งหลายทั้งปวงจงเป็นผู้ไม่มีเวร\\
สัพเพ สัตตา อัพฺยาปัชฌา โหนตุ\\
\indent ขอสัตว์ทั้งหลายทั้งปวงจงอย่าได้เบียดเบียนซึ่งกันและกัน\\
สัพเพ สัตตา อะนีฆา โหนตุ\\
\indent ขอสัตว์ทั้งหลายทั้งปวงจงเป็นผู้ไม่มีทุกข์\\
สัพเพ สัตตา สุขี อัตตานัง ปะริหะรันตุ\\
\indent ขอสัตว์ทั้งหลายทั้งปวงจงรักษาตนอยู่เป็นสุขเถิด\\
สัพเพ สัตตา สัพพะทุกขา ปะมุญจันตุ\\
\indent ขอสัตว์ทั้งหลายทั้งปวงจงพ้นจากทุกข์ทั้งมวล\\
สัพเพ สัตตา ลัทธะสัมปัตติโต มา วิคัจฉันตุ\\
\indent ขอสัตว์ทั้งหลายทั้งปวงจงอย่าได้พรากจากสมบัติอันตนได้แล้ว\\
สัพเพ สัตตา กัมมัสสะกา กัมมะทายาทา\\
กัมมะโยนิ กัมมะพันธุ กัมมะปะฏิสะระณา\\
\indent สัตว์ทั้งหลายทั้งปวงมีกรรมเป็นของของตน, มีกรรมเป็นผู้ให้ผล,
มีกรรมเป็นแดนเกิด, มีกรรมเป็นผู้ติดตาม,มีกรรมเป็นที่พึ่งอาศ้ย\\
ยัง กัมมัง กะรัสสันติ, กัลฺยาณัง วา ปาปะกัง วา, ตัสสะ ทายาทา ภาวิสสันติ\\
\indent จักทำกรรมอันใดไว้, เป็นบุญหรือเป็นบาป, จักต้องเป็นผู้ได้รับผลกรรมนั้นๆ สืบไป

\pagebreak
\section{ท๎วัตติงสาการปาฐะ}
\begin{center}
(หันทะ มะยัง ท๎วัตติงสาการะปาฐัง ภะณามะ เส)\\
เชิญเถิด เราทั้งหลาย จงกล่าวคาถาแสดงอาการ ๓๒ ในร่างกายเถิด
\end{center}
อะยัง โข เม กาโย\\
\indent กายของเรานี้แล\\
อุทธัง ปาทะตะลา\\
\indent เบื้องบนแต่พื้นเท้าขึ้นมา\\
อะโธ เกสะมัตถะกา\\
\indent เบื้องต่ำแต่ปลายผมลงไป\\
ตะจะปะริยันโต\\
\indent มีหนังหุ้มอยู่เป็นที่สุดรอบ\\
ปูโรนานัปปะการัสสะ อะสุจิโน\\
\indent เต็มไปด้วยของไม่สะอาด มีประการต่างๆ\\
อัตถิ อิมัสมิง กาเย \\
\indent มีอยู่ในกายนี้\\
เกสา \\
\indent คือผมทั้งหลาย\\
โลมา \\
\indent คือขนทั้งหลาย\\
นะขา \\
\indent คือเล็บทั้งหลาย\\
ทันตา \\
\indent คือฟันทั้งหลาย\\
ตะโจ \\
\indent หนัง\\
มังสัง \\
\indent เนื้อ\\
นะหารู \\
\indent เอ็นทั้งหลาย\\
อัฏฐี \\
\indent กระดูกทั้งหลาย\\
อัฏฐิมิญชัง \\
\indent เยื่อในกระดูก\\
วักกัง \\
\indent ไต\\
หะทะยัง \\
\indent หัวใจ\\
ยะกะนัง \\
\indent ตับ\\
กิโลมะกัง \\
\indent พังผืด\\
ปิหะกัง \\
\indent ม้าม\\
ปัปผาสัง \\
\indent ปอด\\
อันตัง \\
\indent ไส้ใหญ่\\
อันตะคุณัง \\
\indent สายรัดไส้\\
อุทะริยัง \\
\indent อาหารใหม่\\
กะรีสัง \\
\indent อาหารเก่า\\
ปิตตัง \\
\indent น้ำดี\\
เสมหัง \\
\indent น้ำเสลด\\
ปุพโพ \\
\indent น้ำเหลือง\\
โลหิตัง \\
\indent น้ำเลือด\\
เสโท \\
\indent น้ำเหงื่อ\\
เมโท \\
\indent น้ำมันข้น\\
อัสสุ \\
\indent น้ำตา\\
วะสา \\
\indent น้ำมันเหลว\\
เขโฬ \\
\indent น้ำลาย\\
สิงฆาณิกา \\
\indent น้ำมูก\\
ละสิกา \\
\indent น้ำมันไขข้อ\\
มุตตัง \\
\indent น้ำมูตร\\
มัตถะเกมัตถะลุงคัง \\
\indent เยื่อในสมอง\\
เอวะมะยังเม กาโย \\
\indent กายของเรานี้อย่างนี้\\
อุทธังปาทะตะลา \\
\indent เบื้องบนแต่พื้นเท้าขึ้นมา\\
อะโธเกสะมัตถะกา \\
\indent เบื้องต่ำแต่ปลายผมลงไป\\
ตะจะปะริยันโต \\
\indent มีหนังหุ้มอยู่เป็นที่สุดรอบ\\
ปูโรนานัปปะการัสสะ อะสุจิโน\\
\indent เต็มไปด้วยของไม่สะอาด มีประการต่างๆ อย่างนี้แลฯ\\

\pagebreak
\section{อภิณหปัจจเวกขณปาฐะ}
\begin{center}
(หันทะ มะยัง อะภิณหะปัจจะเวกขะณะปาฐัง ภะณามะ เส)\\
เชิญเถิดเราทั้งหลาย มาสวดอภิณหปัจจเวกขณะปาฐะกันเถิด
\end{center}
ชะราธัมโมมหิ ชะรัง อะนะตีโต (อะนะตีตา) \\
\indent เรามีความแก่เป็นธรรมดาจะล่วงพ้นความแก่ไปไม่ได้\\
พ๎ยาธิธัมโมมหิ พ๎ยาธิง อะนะตีโต (อะนะตีตา)  \\
\indent เรามีความเจ็บไข้เป็นธรรมดา จะล่วงพ้นความเจ็บไข้ไปไม่ได้\\
มะระณะธัมโมมหิ มะระณัง อะนะตีโต (อะนะตีตา) \\
\indent เรามีความตายเป็นธรรมดาจะล่วงพ้นความตายไปไม่ได้\\
สัพเพหิ เม ปิเยหิ มะนาเปหิ นานาภาโว วินาภาโว\\
\indent เราจะละเว้นเป็นต่างๆ, คือว่าจะต้องพลัดพรากจากของรักของเจริญใจ ทั้งหลายทั้งปวง\\
กัมมัสสะโกมหิ กัมมะทายาโท กัมมะโยนิ\\
กัมมะพันธุ กัมมะปะฏิสะระโณ (ณา)  \\
\indent เราเป็นผู้มีกรรมเป็นของๆตน, มีกรรมเป็นผู้ให้ผล,
มีกรรมเป็นแดนเกิด, มีกรรมเป็นผู้ติดตาม, มีกรรมเป็นที่พึ่งอาศัย\\
ยัง กัมมัง กะริสสามิ, กัลยาณัง วา ปาปะกัง วา,\\
ตัสสะ ทายาโท (ทา)  ภะวิสสาม\\
\indent เราทำกรรมอันใดไว้, เป็นบุญหรือเป็นบาป,
เราจะเป็นทายาท, คือว่าเราจะต้องได้รับผลของกรรมนั้นๆ สืบไป
เอวัง อัมเหหิ อะภิณหัง ปัจจะเวกขิตัพพัง\\
\indent เราทั้งหลายควรพิจารณาอย่างนี้ทุกวันๆ เถิด

\pagebreak
\section{บทพิจารณาสังขาร }
\begin{center}
 (หันทะ มะยัง ธัมมะสังเวคะปัจจะเวกขะณะปาฐัง ภะณามะ เส)\\
เชิญเถิด เราทั้งหลาย จงกล่าวคาถาพิจารณาธรรมสังเวชเถิด
\end{center}
สัพเพ สังขารา อะนิจจา\\
\indent สังขารคือร่างกายจิตใจ, แลรูปธรรมนามธรรมทั้งหมดทั้งสิ้น,\\
มันไม่เที่ยง, เกิดขึ้นแล้วดับไปมีแล้วหายไป\\
สัพเพ สังขารา ทุกขา\\
\indent สังขารคือร่างกายจิตใจ, แลรูปธรรมนามธรรมทั้งหมดทั้งสิ้น,\\
มันเป็นทุกข์ทนยาก, เพราะเกิดขึ้นแล้วแก่เจ็บตายไป\\
สัพเพ ธัมมา อะนัตตา\\
\indent สิ่งทั้งหลายทั้งปวง, ทั้งที่เป็นสังขารแลมิใช่สังขารทั้งหมดทั้งสิ้น,\\
ไม่ใช่ตัวไม่ใช่ตน, ไม่ควรถือว่าเราว่าของเราว่าตัวว่าตนของเรา\\
อะธุวัง ชีวิตัง\\
\indent ชีวิตเป็นของไม่ยั่งยืน\\
ธุวัง มะระณัง\\
\indent ความตายเป็นของยั่งยืน\\
อะวัสสัง มะยา มะริตัพพัง\\
\indent อันเราจะพึงตายเป็นแท้\\
มะระณะปะริโยสานัง เม ชีวิตัง\\
\indent ชีวิตของเรามีความตายเป็นที่สุดรอบ\\
ชีวิตัง เม อะนิยะตัง\\
\indent ชีวิตของเราเป็นของไม่เที่ยง\\
มะระณัง เม นิยะตัง\\
\indent ความตายของเราเป็นของเที่ยง\\
วะตะ\\
\indent ควรที่จะสังเวช\\
อะยัง กาโย อะจิรัง\\
\indent ร่างกายนี้มิได้ตั้งอยู่นาน\\
อะเปตะวิญญาโณ\\
\indent ครั้นปราศจากวิญญาณ\\
ฉุฑโฑ\\
\indent อันเขาทิ้งเสียแล้ว\\
อธิเสสสะติ\\
\indent จักนอนทับ\\
ปะฐะวิง\\
\indent ซึ่งแผ่นดิน\\
กะลิงคะรัง อิวะ\\
\indent ประดุจดังว่าท่อนไม้และท่อนฟืน\\
นิรัตถัง\\
\indent หาประโยชน์มิได้\\
อะนิจจา วะตะ สังขารา\\
\indent สังขารทั้งหลายไม่เที่ยงหนอ\\
อุปปาทะวะยะธัมมิโน\\
\indent มีความเกิดขึ้นแล้วมีความเสื่อมไปเป็นธรรมดา\\
อุปปัชชิตฺวา นิรุชฌันติ\\
\indent ครั้นเกิดขึ้นแล้วย่อมดับไป\\
เตสัง วูปะสะโม สุโข\\
\indent ความเข้าไปสงบระงับสังขารทั้งหลาย, เป็นสุขอย่างยิ่ง, ดังนี้\\





\end{document}