%!TEX program = xelatex
%!TEX encoding = UTF-8 Unicode 
\documentclass{article}
\usepackage{fontspec}
\defaultfontfeatures{Mapping=tex-text}
\usepackage{xunicode}
\usepackage{xltxtra}
\setmainfont{TH SarabunPSK}
\XeTeXlinebreaklocale 'th_TH' 
\usepackage{amsmath}
\usepackage{amsfonts}
\usepackage{amssymb}
\usepackage{csquotes}
\usepackage{mathtools}
\usepackage{pgfplots}

\usepackage{color}
\definecolor{white}{rgb}{1,1,1}
\definecolor{dkgreen}{rgb}{0,0.6,0}
\definecolor{gray}{rgb}{0.5,0.5,0.5}
\definecolor{mauve}{rgb}{0.58,0,0.82}
\definecolor{cream}{rgb}{1, 0.992, 0.816}
\definecolor{lightyellow}{rgb}{1, 1, 0.875}

\title{หนังสือสวดมนต์แปล}
\date{\today}

\begin{document}
\pagecolor{lightyellow}
\maketitle
\newpage
\tableofcontents

\pagebreak
\section{คำบูชาพระรัตนตรัย}
\bf โย โส ภะคะวา อะระหัง สัมมาสัมพุทโธ

พระผู้มีพระภาคเจ้านั้น พระองค์ใด, เป็นพระอรหันต์, 
ดับเพลิงกิเลสเพลิงทุกข์สิ้นเชิง, ตรัสรู้ชอบได้โดยพระองค์เอง\\
ส๎วากขาโต เยนะ ภะคะวะตา ธัมโม 

พระธรรม เป็นธรรมอันพระผู้มีพระภาคเจ้า พระองค์ใด, ตรัสไว้ดีแล้ว\\
สุปะฏิปันโน ยัสสะ ภะคะวะโต สาวะกะสังโฆ

พระสงฆ์สาวกของพระผู้มีพระภาคเจ้า พระองค์ใด, ปฏิบัติดีแล้ว\\
ตัมมะยัง ภะคะวันตัง สะธัมมัง สะสังฆัง
อิเมหิ สักกาเรหิ ยะถาระหัง อาโรปิเตหิ อะภิปูชะยามะ

ข้าพเจ้าทั้งหลาย, ขอบูชาอย่างยิ่งซึ่งพระผู้มีพระภาคเจ้าพระองค์นั้น
,พร้อมทั้งพระธรรมและพระสงฆ์,
ด้วยเครื่องสักการะทั้งหลายเหล่านี้, อันยกขึ้นตามสมควรแล้วอย่างไร\\
สาธุ โน ภันเต ภะคะวา สุจิระปะรินิพพุโตปิ

ข้าแต่พระองค์ผู้เจริญ, พระผู้มีพระภาคเจ้าแม้ปรินิพพานนานแล้ว,
ทรงสร้างคุณอันสำเร็จประโยชน์ไว้แก่ข้าพเจ้าทั้งหลาย.\\
ปัจฉิมาชะนะตานุกัมปะมานะสา

ทรงมีพระหฤทัยอนุเคราะห์แก่พวกข้าพเจ้า อันเป็นชนรุ่นหลัง\\
อิเม สักกาเร ทุคคะตะปัณณาการะภูเต ปะฏิคคัณหาตุ

ขอพระผู้มีพระภาคเจ้าจงรับเครื่องสักการะ อันเป็นบรรณาการของคน
ยากทั้งหลายเหล่านี้\\
อัมหากัง ทีฆะรัตตัง หิตายะ สุขายะ

เพื่อประโยชน์และความสุขแก่ข้าพเจ้าทั้งหลาย ตลอดกาลนาน เทอญฯ\\
อะระหัง สัมมาสัมพุทโธ ภะคะวา

พระผู้มีพระภาคเจ้า, เป็นพระอรหันต์, ดับเพลิงกิเลสเพลิงทุกข์สิ้นเชิง,
ตรัสรู้ชอบได้โดยพระองค์เอง\\
พุทธัง ภะคะวันตัง อะภิวาเทมิ

ข้าพเจ้าอภิวาทพระผู้มีพระภาคเจ้า, ผู้รู้ ผู้ตื่น ผู้เบิกบาน (กราบ)\\
ส๎วากขาโต ภะคะวะตา ธัมโม 

พระธรรมเป็นธรรมที่พระผู้มีพระภาคเจ้า, ตรัสไว้ดีแล้ว\\
ธัมมัง นะมัสสามิ

ข้าพเจ้านมัสการพระธรรม (กราบ)\\
สุปะฏิปันโน ภะคะวะโต สาวะกะสังโฆ,

พระสงฆ์สาวกของพระผู้มีพระภาคเจ้า, ปฏิบัติดีแล้ว\\
สังฆัง นะมามิ.

ข้าพเจ้านอบน้อมพระสงฆ์ (กราบ)

\pagebreak
\section{ปุพพภาคนมการ}
\begin{center}
(หันทะ มะยัง พุทธัสสะ ภะคะวะโต ปุพพะภาคะนะมะการัง กะโรมะ เส)\\
เชิญเถิด เราทั้งหลาย ทำความนอบน้อมอันเป็นส่วนเบื้องต้น แด่พระผู้มีพระภาคเจ้าเถิด
\end{center}
นะโม ตัสสะ ภะคะวะโต,

ขอนอบน้อมแด่พระผู้มีพระภาคเจ้าพระองค์นั้น\\
อะระหะโต,

ซึ่งเป็นผู้ไกลจากกิเลส\\
สัมมาสัมพุทธัสสะ.

ตรัสรู้ชอบได้โดยพระองค์เอง.	
\begin{center}
(กล่าว ๓ ครั้ง)
\end{center}

\pagebreak

\section{พุทธาภิถุติ}
\begin{center}
(หันทะ มะยัง พุทธาภิถุติง กะโรมะ เส)\\
เชิญเถิด เราทั้งหลาย ทำความชมเชยเฉพาะพระพุทธเจ้าเถิด
\end{center}
โย โส ตะถาคะโต

พระตถาคตเจ้านั้น พระองค์ใด\\
อะระหัง

เป็นผู้ไกลจากกิเลส\\
สัมมาสัมพุทโธ

เป็นผู้ตรัสรู้ชอบได้โดยพระองค์เอง\\
วิชชาจะระณะสัมปันโน

เป็นผู้ถึงพร้อมด้วยวิชชาและจรณะ\\
สุคะโต

เป็นผู้ไปแล้วด้วยดี\\
โลกะวิท

เป็นผู้รู้โลกอย่างแจ่มแจ้ง\\
อะนุตตะโร ปุริสะทัมมะสาระถิ

เป็นผู้สามารถฝึกบุรุษที่สมควรฝึกได้อย่างไม่มีใครยิ่งกว่า\\
สัตถา เทวะมะนุสสานัง

เป็นครูผู้สอนของเทวดาและมนุษย์ทั้งหลาย\\
พุทโธ

เป็นผู้รู้ ผู้ตื่น ผู้เบิกบานด้วยธรรม\\
ภะคะวา

เป็นผู้มีความจำเริญ จำแนกธรรมสั่งสอนสัตว์\\
โย อิมัง โลกัง สะเทวะกัง สะมาระกัง สะพ๎รัห๎มะกัง, 
สัสสะมะณะพ๎ราห๎มะณิงปะชัง สะเทวะมะนุสสัง
 สะยัง อะภิญญา สัจฉิกัต๎วา ปะเวเทสิ
 
พระผู้มีพระภาคเจ้าพระองค์ใด, ได้ทรงทำความดับทุกข์ให้แจ้ง 
ด้วยพระปัญญาอันยิ่งเองแล้ว, ทรงสอนโลกนี้พร้อมทั้งเทวดา 
มาร พรหม และหมู่สัตว์ พร้อมทั้งสมณพราหมณ, 
พร้อมทั้งเทวดาและมนุษย์ให้รู้ตาม\\
โย ธัมมัง เทเสสิ

พระผู้มีพระภาคเจ้าพระองค์ใด ทรงแสดงธรรมแล้ว\\
อาทิกัล๎ยาณัง

ไพเราะในเบื้องต้น\\
มัชเฌกัล๎ยาณัง

ไพเราะในท่ามกลาง\\
ปะริโยสานะกัล๎ยาณัง

ไพเราะในที่สุด\\
สาตถัง สะพ๎ยัญชะนัง เกวะละปะริปุณณัง ปะริสุทธัง พ๎รัห๎มะจะริยัง ปะกาเสสิ
ทรงประกาศพรหมจรรย์ คือแบบแห่งการปฏิบัติอันประเสริฐ บริสุทธิ์

บริบูรณ์ สิ้นเชิง,พร้อมทั้งอรรถะ (คำอธิบาย) พร้อมทั้งพยัญชนะ (หัวข้อ)\\
ตะมะหัง ภะคะวันตัง อะภิปูชะยามิ

ข้าพเจ้าบูชาอย่างยิ่ง เฉพาะพระผู้มีพระภาคเจ้าพระองค์นั้น\\
ตะมะหัง ภะคะวันตัง สิระสา นะมามิ

ข้าพเจ้านอบน้อมพระผู้มีพระภาคเจ้า พระองค์นั้นด้วยเศียรเกล้า
\begin{center}
(กราบระลึกพระพุทธคุณ)
\end{center}
\pagebreak
\section{ธัมมาภิถุติ}
\begin{center}
(หันทะ มะยัง ธัมมาภิถุติง กะโรมะ เส)\\
เชิญเถิด เราทั้งหลาย ทำความชมเชยเฉพาะพระธรรมเถิด
\end{center}
โย โส ส๎วากขาโต ภะคะวะตา ธัมโม

พระธรรมนั้นใด, เป็นสิ่งที่พระผู้มีพระภาคเจ้าได้ตรัสไว้ดีแล้ว\\
สันทิฏฐิโก

เป็นสิ่งที่ผู้ศึกษาและปฏิบัติพึงเห็นได้ด้วยตนเอง\\
อะกาลิโก

เป็นสิ่งที่ปฏิบัติได้ และให้ผลได้ไม่จำกัดกาล\\
เอหิปัสสิโก

เป็นสิ่งที่ควรกล่าวกะผู้อื่นว่า ท่านจงมาดูเถิด\\
โอปะนะยิโก

เป็นสิ่งที่ควรน้อมเข้ามาใส่ตัว\\
ปัจจัตตัง เวทิตัพโพ วิญญูหิ

เป็นสิ่งที่ผู้รู้ก็รู้ได้เฉพาะตน\\
ตะมะหัง ธัมมัง อะภิปูชะยามิ

ข้าพเจ้าบูชาอย่างยิ่ง เฉพาะพระธรรมนั้น\\
ตะมะหัง ธัมมัง สิระสา นะมามิ

ข้าพเจ้านอบน้อมพระธรรมนั้น ด้วยเศียรเกล้า
\begin{center}
(กราบระลึกพระธรรมคุณ)
\end{center}
\pagebreak
\section{สังฆาภิถุติ}
\begin{center}
หันทะ มะยัง สังฆาภิถุติง กะโรมะ เส
เชิญเถิด เราทั้งหลาย ทำความชมเชยเฉพาะพระสงฆ์เถิด
\end{center}
โย โส สุปะฏิปันโน ภะคะวะโต สาวะกะสังโฆ

สงฆ์สาวกของพระผู้มีพระภาคเจ้านั้นหมู่ใด ปฏิบัติดีแล้ว\\
อุชุปะฏิปันโน ภะคะวะโต สาวะกะสังโฆ

สงฆ์สาวกของพระผู้มีพระภาคเจ้าหมู่ใด ปฏิบัติตรงแล้ว\\
ญายะปะฏิปันโน ภะคะวะโต สาวะกะสังโฆ

สงฆ์สาวกของพระผู้มีพระภาคเจ้าหมู่ใด, ปฏิบัติเพื่อรู้ธรรมเป็นเครื่องออกจากทุกข์แล้ว\\
สามีจิปะฏิปันโน ภะคะวะโต สาวะกะสังโฆ

สงฆ์สาวกของพระผู้มีพระภาคเจ้าหมู่ใด, ปฏิบัติสมควรแล้ว\\
ยะทิทัง

ได้แก่บุคคลเหล่านี้คือ\\
จัตตาริ ปุริสะยุคานิ อัฏฐะ ปุริสะปุคคะลา

คู่แห่งบุรุษ ๔ คู่, นับเรียงตัวบุรุษได้ ๘ บุรุษ*
\footnote{* สี่คู่คือ โสดาปัตติมรรค โสดาปัตติผล, สกิทาคามิมรรค สกิทาคามิผล,
อนาคามิมรรค อนาคามิผล, อรหัตตมรรค อรหัตตผล.}\\
เอสะ ภะคะวะโต สาวะกะสังโฆ

นั่นแหละสงฆ์สาวกของพระผู้มีพระภาคเจ้า\\
อาหุเนยโย

เป็นสงฆ์ควรแก่สักการะที่เขานำมาบูชา\\
ปาหุเนยโย

เป็นสงฆ์ควรแก่สักการะที่เขาจัดไว้ต้อนรับ\\
ทักขิเณยโย

เป็นผู้ควรรับทักษิณาทาน\\
อัญชะลิกะระณีโย

เป็นผู้ที่บุคคลทั่วไปควรทำอัญชลี\\
อะนุตตะรัง ปุญญักเขตตัง โลกัสสะ

เป็นเนื้อนาบุญของโลก, ไม่มีนาบุญอื่นยิ่งกว่า\\
ตะมะหัง สังฆัง อะภิปูชะยามิ

ข้าพเจ้าบูชาอย่างยิ่ง เฉพาะพระสงฆ์หมู่นั้น\\
ตะมะหัง สังฆัง สิระสา นะมามิ

ข้าพเจ้านอบน้อมพระสงฆ์หมู่นั้น ด้วยเศียรเกล้า
\begin{center}
(กราบระลึกพระสังฆคุณ)
\end{center}
\pagebreak

\section{รตนัตตยัปปณามคาถา}
\begin{center}
(หันทะ มะยัง ระตะนัตตะยัปปะณามะคาถาโย เจวะสังเวคะวัตถุปะริกิตตะนะปาฐัญจะ ภะณามะ เส)\\เชิญเถิด เราทั้งหลาย กล่าวคำนอบน้อมพระรัตนตรัยและบาลีที่กำหนดวัตถุเครื่องแสดงความสังเวชเถิด
\end{center}
พุทโธ สุสุทโธ กะรุณามะหัณณะโว

พระพุทธเจ้าผู้บริสุทธิ์ มีพระกรุณาดุจห้วงมหรรณพ\\
โยจจันตะสุทธัพพะระญาณะโลจะโน

พระองค์ใด มีตาคือญาณอันประเสริฐหมดจดถึงที่สุด\\
โลกัสสะ ปาปูปะกิเลสะฆาตะโก

เป็นผู้ฆ่าเสียซึ่งบาปและอุปกิเลสของโลก\\
วันทามิ พุทธัง อะหะมาทะเรนะ ตัง

ข้าพเจ้าไหว้พระพุทธเจ้าพระองค์นั้น โดยใจเคารพเอื้อเฟื้อ\\
ธัมโม ปะทีโป วิยะ ตัสสะ สัตถุโน

พระธรรมของพระศาสดา สว่างรุ่งเรืองเปรียบดวงประทีป\\
โย มัคคะปากามะตะเภทะภินนะโก

จำแนกประเภท คือ มรรค ผล นิพพาน, ส่วนใด\\
โลกุตตะโร โย จะ ตะทัตถะทีปะโน

ซึ่งเป็นตัวโลกุตตระ, และส่วนใดที่ชี้แนวแห่งโลกุตตระนั้น\\
วันทามิ ธัมมัง อะหะมาทะเรนะ ตัง

ข้าพเจ้าไหว้พระธรรมนั้น โดยใจเคารพเอื้อเฟื้อ\\
สังโฆ สุเขตตาภ๎ยะติเขตตะสัญญิโต

พระสงฆ์เป็นนาบุญอันยิ่งใหญ่กว่านาบุญอันดีทั้งหลาย\\
โย ทิฏฐะสันโต สุคะตานุโพธะโก

เป็นผู้เห็นพระนิพพาน, ตรัสรู้ตามพระสุคต, หมู่ใด\\
โลลัปปะหี\footnote{อ่านว่า ฮี} โน อะริโย สุเมธะโส

เป็นผู้ละกิเลสเครื่องโลเลเป็นพระอริยเจ้า มีปัญญาดี\\
วันทามิ สังฆัง อะหะมาทะเรนะ ตัง

ข้าพเจ้าไหว้พระสงฆ์หมู่นั้นโดยใจเคารพเอื้อเฟื้อ\\
อิจเจวะเมกันตะภิปูชะเนยยะกัง, วัตถุตตะยัง วันทะยะตาภิสังขะตัง,
ปุญญัง มะยา ยัง มะมะ สัพพุปัททะวา,มา โหนตุ เว ตัสสะ ปะภาวะสิทธิยา

บุญใดที่ข้าพเจ้าผู้ไหว้อยู่ซึ่งวัตถุสาม, คือพระรัตนตรัยอันควรบูชายิ่งโดยส่วนเดียว,
ได้กระทำแล้วเป็นอย่างยิ่งเช่นนี้, ขออุปัทวะ(ความชั่ว) ทั้งหลาย,
จงอย่ามีแก่ข้าพเจ้าเลย, ด้วยอำนาจความสำเร็จอันเกิดจากบุญนั้น






\end{document}