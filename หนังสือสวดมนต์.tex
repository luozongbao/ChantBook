%!TEX program = xelatex
%!TEX encoding = UTF-8 Unicode 
\documentclass[12pt]{article}
\usepackage{fontspec}
\usepackage[a5paper]{geometry}
\defaultfontfeatures{Mapping=tex-text}
\usepackage{xunicode}
\usepackage{xltxtra}
\setmainfont{TH SarabunPSK}
\XeTeXlinebreaklocale 'th_TH' 
\usepackage{amsmath}
\usepackage{amsfonts}
\usepackage{amssymb}
\usepackage{csquotes}
\usepackage{mathtools}
\usepackage{pgfplots}

\usepackage{color}
\definecolor{white}{rgb}{1,1,1}
\definecolor{dkgreen}{rgb}{0,0.6,0}
\definecolor{gray}{rgb}{0.5,0.5,0.5}
\definecolor{mauve}{rgb}{0.58,0,0.82}
\definecolor{cream}{rgb}{1, 0.992, 0.816}
\definecolor{lightyellow}{rgb}{1, 1, 0.875}



\title{หนังสือสวดมนต์แปล}
\date{\today}
\newfontfamily{\Title}[Scale=3]{TH SarabunPSK}

\begin{document}
\pagecolor{lightyellow}
\maketitle
\newpage
\tableofcontents

\pagebreak

\begin{titlepage}

\vspace*{\fill}
\begin{center}
  \scalebox{3}{\textbf{คำทำวัตรเช้า}}
\end{center}
\vspace{\fill}

\end{titlepage}
\pagebreak


\section{คำบูชาพระรัตนตรัย}
\hrule
\textbf{โย โส ภะคะวา อะระหัง สัมมาสัมพุทโธ}\\
\indent พระผู้มีพระภาคเจ้านั้น พระองค์ใด, เป็นพระอรหันต์, \\
\indent ดับเพลิงกิเลสเพลิงทุกข์สิ้นเชิง, ตรัสรู้ชอบได้โดยพระองค์เอง\\
\textbf{ส๎วากขาโต เยนะ ภะคะวะตา ธัมโม}\\
\indent พระธรรม เป็นธรรมอันพระผู้มีพระภาคเจ้า พระองค์ใด, ตรัสไว้ดีแล้ว\\
\textbf{สุปะฏิปันโน ยัสสะ ภะคะวะโต สาวะกะสังโฆ}\\
\indent พระสงฆ์สาวกของพระผู้มีพระภาคเจ้า พระองค์ใด, ปฏิบัติดีแล้ว\\
\textbf{ตัมมะยัง ภะคะวันตัง สะธัมมัง สะสังฆัง\\
อิเมหิ สักกาเรหิ ยะถาระหัง อาโรปิเตหิ อะภิปูชะยามะ}\\
\indent ข้าพเจ้าทั้งหลาย, ขอบูชาอย่างยิ่งซึ่งพระผู้มีพระภาคเจ้าพระองค์นั้น,\\
\indent พร้อมทั้งพระธรรมและพระสงฆ์,\\
\indent ด้วยเครื่องสักการะทั้งหลายเหล่านี้, อันยกขึ้นตามสมควรแล้วอย่างไร\\
\textbf{สาธุ โน ภันเต ภะคะวา สุจิระปะรินิพพุโตปิ}\\
\indent ข้าแต่พระองค์ผู้เจริญ, พระผู้มีพระภาคเจ้าแม้ปรินิพพานนานแล้ว,\\
\indent ทรงสร้างคุณอันสำเร็จประโยชน์ไว้แก่ข้าพเจ้าทั้งหลาย.\\
\textbf{ปัจฉิมาชะนะตานุกัมปะมานะสา}\\
\indent ทรงมีพระหฤทัยอนุเคราะห์แก่พวกข้าพเจ้า อันเป็นชนรุ่นหลัง\\
\textbf{อิเม สักกาเร ทุคคะตะปัณณาการะภูเต ปะฏิคคัณหาตุ}\\
\indent ขอพระผู้มีพระภาคเจ้าจงรับเครื่องสักการะ อันเป็นบรรณาการของคน\\
\indent ยากทั้งหลายเหล่านี้\\
\textbf{อัมหากัง ทีฆะรัตตัง หิตายะ สุขายะ}\\
\indent เพื่อประโยชน์และความสุขแก่ข้าพเจ้าทั้งหลาย ตลอดกาลนาน เทอญฯ\\
\textbf{อะระหัง สัมมาสัมพุทโธ ภะคะวา}\\
\indent พระผู้มีพระภาคเจ้า, เป็นพระอรหันต์, ดับเพลิงกิเลสเพลิงทุกข์สิ้นเชิง,\\
\indent ตรัสรู้ชอบได้โดยพระองค์เอง\\
\textbf{พุทธัง ภะคะวันตัง อะภิวาเทมิ}\\
\indent ข้าพเจ้าอภิวาทพระผู้มีพระภาคเจ้า, ผู้รู้ ผู้ตื่น ผู้เบิกบาน (กราบ)\\
\textbf{ส๎วากขาโต ภะคะวะตา ธัมโม} \\
\indent พระธรรมเป็นธรรมที่พระผู้มีพระภาคเจ้า, ตรัสไว้ดีแล้ว\\
\textbf{ธัมมัง นะมัสสามิ}\\
\indent ข้าพเจ้านมัสการพระธรรม (กราบ)\\
\textbf{สุปะฏิปันโน ภะคะวะโต สาวะกะสังโฆ,}\\
\indent พระสงฆ์สาวกของพระผู้มีพระภาคเจ้า, ปฏิบัติดีแล้ว\\
\textbf{สังฆัง นะมามิ}\\
\indent ข้าพเจ้านอบน้อมพระสงฆ์ (กราบ)

\pagebreak
\section{ปุพพภาคนมการ}
\hrule
\begin{center}
\textbf{(หันทะ มะยัง พุทธัสสะ ภะคะวะโต ปุพพะภาคะนะมะการัง กะโรมะ เส)}\\
เชิญเถิด เราทั้งหลาย ทำความนอบน้อมอันเป็นส่วนเบื้องต้น แด่พระผู้มีพระภาคเจ้าเถิด
\end{center}
\textbf{นะโม ตัสสะ ภะคะวะโต,}\\
\indent ขอนอบน้อมแด่พระผู้มีพระภาคเจ้าพระองค์นั้น\\
\textbf{อะระหะโต,}\\
\indent ซึ่งเป็นผู้ไกลจากกิเลส\\
\textbf{สัมมาสัมพุทธัสสะ}\\
\indent ตรัสรู้ชอบได้โดยพระองค์เอง.	
\begin{center}
(กล่าว ๓ ครั้ง)
\end{center}

\pagebreak

\section{พุทธาภิถุติ}
\hrule
\begin{center}
\textbf{(หันทะ มะยัง พุทธาภิถุติง กะโรมะ เส)}\\
เชิญเถิด เราทั้งหลาย ทำความชมเชยเฉพาะพระพุทธเจ้าเถิด
\end{center}
\textbf{โย โส ตะถาคะโต\\}
\indent พระตถาคตเจ้านั้น พระองค์ใด\\
\textbf{อะระหัง\\}
\indent เป็นผู้ไกลจากกิเลส\\
\textbf{สัมมาสัมพุทโธ\\}
\indent เป็นผู้ตรัสรู้ชอบได้โดยพระองค์เอง\\
\textbf{วิชชาจะระณะสัมปันโน\\}
\indent เป็นผู้ถึงพร้อมด้วยวิชชาและจรณะ\\
\textbf{สุคะโต\\}
\indent เป็นผู้ไปแล้วด้วยดี\\
\textbf{โลกะวิท\\}
\indent เป็นผู้รู้โลกอย่างแจ่มแจ้ง\\
\textbf{อะนุตตะโร ปุริสะทัมมะสาระถิ\\}
\indent เป็นผู้สามารถฝึกบุรุษที่สมควรฝึกได้อย่างไม่มีใครยิ่งกว่า\\
\textbf{สัตถา เทวะมะนุสสานัง\\}
\indent เป็นครูผู้สอนของเทวดาและมนุษย์ทั้งหลาย\\
\textbf{พุทโธ\\}
\indent เป็นผู้รู้ ผู้ตื่น ผู้เบิกบานด้วยธรรม\\
\textbf{ภะคะวา\\}
\indent เป็นผู้มีความจำเริญ จำแนกธรรมสั่งสอนสัตว์\\
\textbf{โย อิมัง โลกัง สะเทวะกัง สะมาระกัง สะพ๎รัห๎มะกัง, \\
สัสสะมะณะพ๎ราห๎มะณิงปะชัง สะเทวะมะนุสสัง\\
 สะยัง อะภิญญา สัจฉิกัต๎วา ปะเวเทสิ\\}
\indent พระผู้มีพระภาคเจ้าพระองค์ใด, ได้ทรงทำความดับทุกข์ให้แจ้ง \\
\indent ด้วยพระปัญญาอันยิ่งเองแล้ว, ทรงสอนโลกนี้พร้อมทั้งเทวดา \\
\indent มาร พรหม และหมู่สัตว์ พร้อมทั้งสมณพราหมณ์\\
\indent พร้อมทั้งเทวดาและมนุษย์ให้รู้ตาม\\
\textbf{โย ธัมมัง เทเสสิ\\}
\indent พระผู้มีพระภาคเจ้าพระองค์ใด ทรงแสดงธรรมแล้ว\\
\textbf{อาทิกัล๎ยาณัง\\}
\indent ไพเราะในเบื้องต้น\\
\textbf{มัชเฌกัล๎ยาณัง\\}
\indent ไพเราะในท่ามกลาง\\
\textbf{ปะริโยสานะกัล๎ยาณัง\\}
\indent ไพเราะในที่สุด\\
\textbf{สาตถัง สะพ๎ยัญชะนัง เกวะละปะริปุณณัง ปะริสุทธัง พ๎รัห๎มะจะริยัง ปะกาเสสิ\\}
\indent ทรงประกาศพรหมจรรย์ คือแบบแห่งการปฏิบัติอันประเสริฐ บริสุทธิ์\\
\indent บริบูรณ์ สิ้นเชิง,พร้อมทั้งอรรถะ (คำอธิบาย) พร้อมทั้งพยัญชนะ (หัวข้อ)\\
\textbf{ตะมะหัง ภะคะวันตัง อะภิปูชะยามิ\\}
\indent ข้าพเจ้าบูชาอย่างยิ่ง เฉพาะพระผู้มีพระภาคเจ้าพระองค์นั้น\\
\textbf{ตะมะหัง ภะคะวันตัง สิระสา นะมามิ\\}
\indent ข้าพเจ้านอบน้อมพระผู้มีพระภาคเจ้า พระองค์นั้นด้วยเศียรเกล้า\\
\begin{center}
(กราบระลึกพระพุทธคุณ)
\end{center}
\pagebreak
\section{ธัมมาภิถุติ}
\hrule
\begin{center}
\textbf{(หันทะ มะยัง ธัมมาภิถุติง กะโรมะ เส)\\}
เชิญเถิด เราทั้งหลาย ทำความชมเชยเฉพาะพระธรรมเถิด
\end{center}
\textbf{โย โส ส๎วากขาโต ภะคะวะตา ธัมโม\\}
\indent พระธรรมนั้นใด, เป็นสิ่งที่พระผู้มีพระภาคเจ้าได้ตรัสไว้ดีแล้ว\\
\textbf{สันทิฏฐิโก\\}
\indent เป็นสิ่งที่ผู้ศึกษาและปฏิบัติพึงเห็นได้ด้วยตนเอง\\
\textbf{อะกาลิโก\\}
\indent เป็นสิ่งที่ปฏิบัติได้ และให้ผลได้ไม่จำกัดกาล\\
\textbf{เอหิปัสสิโก\\}
\indent เป็นสิ่งที่ควรกล่าวกะผู้อื่นว่า ท่านจงมาดูเถิด\\
\textbf{โอปะนะยิโก\\}
\indent เป็นสิ่งที่ควรน้อมเข้ามาใส่ตัว\\
\textbf{ปัจจัตตัง เวทิตัพโพ วิญญูหิ\\}
\indent เป็นสิ่งที่ผู้รู้ก็รู้ได้เฉพาะตน\\
\textbf{ตะมะหัง ธัมมัง อะภิปูชะยามิ\\}
\indent ข้าพเจ้าบูชาอย่างยิ่ง เฉพาะพระธรรมนั้น\\
\textbf{ตะมะหัง ธัมมัง สิระสา นะมามิ\\}
\indent ข้าพเจ้านอบน้อมพระธรรมนั้น ด้วยเศียรเกล้า
\begin{center}
(กราบระลึกพระธรรมคุณ)
\end{center}
\pagebreak
\section{สังฆาภิถุติ}
\hrule
\begin{center}
\textbf{(หันทะ มะยัง สังฆาภิถุติง กะโรมะ เส)\\}
เชิญเถิด เราทั้งหลาย ทำความชมเชยเฉพาะพระสงฆ์เถิด
\end{center}
\textbf{โย โส สุปะฏิปันโน ภะคะวะโต สาวะกะสังโฆ\\}
\indent สงฆ์สาวกของพระผู้มีพระภาคเจ้านั้นหมู่ใด ปฏิบัติดีแล้ว\\
\textbf{อุชุปะฏิปันโน ภะคะวะโต สาวะกะสังโฆ\\}
\indent สงฆ์สาวกของพระผู้มีพระภาคเจ้าหมู่ใด ปฏิบัติตรงแล้ว\\
\textbf{ญายะปะฏิปันโน ภะคะวะโต สาวะกะสังโฆ\\}
\indent สงฆ์สาวกของพระผู้มีพระภาคเจ้าหมู่ใด, ปฏิบัติเพื่อรู้ธรรมเป็นเครื่องออกจากทุกข์แล้ว\\
\textbf{สามีจิปะฏิปันโน ภะคะวะโต สาวะกะสังโฆ\\}
\indent สงฆ์สาวกของพระผู้มีพระภาคเจ้าหมู่ใด, ปฏิบัติสมควรแล้ว\\
\textbf{ยะทิทัง\\}
\indent ได้แก่บุคคลเหล่านี้คือ\\
\textbf{จัตตาริ ปุริสะยุคานิ อัฏฐะ ปุริสะปุคคะลา\\}
\indent คู่แห่งบุรุษ ๔ คู่, นับเรียงตัวบุรุษได้ ๘ บุรุษ*
\footnote{* สี่คู่คือ โสดาปัตติมรรค โสดาปัตติผล, สกิทาคามิมรรค สกิทาคามิผล,
อนาคามิมรรค อนาคามิผล, อรหัตตมรรค อรหัตตผล.}\\
\textbf{เอสะ ภะคะวะโต สาวะกะสังโฆ\\}
\indent นั่นแหละสงฆ์สาวกของพระผู้มีพระภาคเจ้า\\
\textbf{อาหุเนยโย\\}
\indent เป็นสงฆ์ควรแก่สักการะที่เขานำมาบูชา\\
\textbf{ปาหุเนยโย\\}
\indent เป็นสงฆ์ควรแก่สักการะที่เขาจัดไว้ต้อนรับ\\
\textbf{ทักขิเณยโย\\}
\indent เป็นผู้ควรรับทักษิณาทาน\\
\textbf{อัญชะลิกะระณีโย\\}
\indent เป็นผู้ที่บุคคลทั่วไปควรทำอัญชลี\\
\textbf{อะนุตตะรัง ปุญญักเขตตัง โลกัสสะ\\}
\indent เป็นเนื้อนาบุญของโลก, ไม่มีนาบุญอื่นยิ่งกว่า\\
\textbf{ตะมะหัง สังฆัง อะภิปูชะยามิ\\}
\indent ข้าพเจ้าบูชาอย่างยิ่ง เฉพาะพระสงฆ์หมู่นั้น\\
\textbf{ตะมะหัง สังฆัง สิระสา นะมามิ\\}
\indent ข้าพเจ้านอบน้อมพระสงฆ์หมู่นั้น ด้วยเศียรเกล้า\\
\begin{center}
(กราบระลึกพระสังฆคุณ)
\end{center}
\pagebreak

\section{รตนัตตยัปปณามคาถา}
\hrule
\begin{center}
\textbf{(หันทะ มะยัง ระตะนัตตะยัปปะณามะคาถาโย เจวะสังเวคะวัตถุปะริกิตตะนะปาฐัญจะ ภะณามะ เส)\\}
เชิญเถิด เราทั้งหลาย กล่าวคำนอบน้อมพระรัตนตรัยและบาลีที่กำหนดวัตถุเครื่องแสดงความสังเวชเถิด
\end{center}
\textbf{พุทโธ สุสุทโธ กะรุณามะหัณณะโว\\}
\indent พระพุทธเจ้าผู้บริสุทธิ์ มีพระกรุณาดุจห้วงมหรรณพ\\
\textbf{โยจจันตะสุทธัพพะระญาณะโลจะโน\\}
\indent พระองค์ใด มีตาคือญาณอันประเสริฐหมดจดถึงที่สุด\\
\textbf{โลกัสสะ ปาปูปะกิเลสะฆาตะโก\\}
\indent เป็นผู้ฆ่าเสียซึ่งบาปและอุปกิเลสของโลก\\
\textbf{วันทามิ พุทธัง อะหะมาทะเรนะ ตัง\\}
\indent ข้าพเจ้าไหว้พระพุทธเจ้าพระองค์นั้น โดยใจเคารพเอื้อเฟื้อ\\
\textbf{ธัมโม ปะทีโป วิยะ ตัสสะ สัตถุโน\\}
\indent พระธรรมของพระศาสดา สว่างรุ่งเรืองเปรียบดวงประทีป\\
\textbf{โย มัคคะปากามะตะเภทะภินนะโก\\}
\indent จำแนกประเภท คือ มรรค ผล นิพพาน, ส่วนใด\\
\textbf{โลกุตตะโร โย จะ ตะทัตถะทีปะโน\\}
\indent ซึ่งเป็นตัวโลกุตตระ, และส่วนใดที่ชี้แนวแห่งโลกุตตระนั้น\\
\textbf{วันทามิ ธัมมัง อะหะมาทะเรนะ ตัง\\}
\indent ข้าพเจ้าไหว้พระธรรมนั้น โดยใจเคารพเอื้อเฟื้อ\\
\textbf{สังโฆ สุเขตตาภ๎ยะติเขตตะสัญญิโต\\}
\indent พระสงฆ์เป็นนาบุญอันยิ่งใหญ่กว่านาบุญอันดีทั้งหลาย\\
\textbf{โย ทิฏฐะสันโต สุคะตานุโพธะโก\\}
\indent เป็นผู้เห็นพระนิพพาน, ตรัสรู้ตามพระสุคต, หมู่ใด\\
\textbf{โลลัปปะหี\footnote{อ่านว่า ฮี} โน อะริโย สุเมธะโส\\}
\indent เป็นผู้ละกิเลสเครื่องโลเลเป็นพระอริยเจ้า มีปัญญาดี\\
\textbf{วันทามิ สังฆัง อะหะมาทะเรนะ ตัง\\}
\indent ข้าพเจ้าไหว้พระสงฆ์หมู่นั้นโดยใจเคารพเอื้อเฟื้อ\\
\textbf{อิจเจวะเมกันตะภิปูชะเนยยะกัง, วัตถุตตะยัง วันทะยะตาภิสังขะตัง,\\
ปุญญัง มะยา ยัง มะมะ สัพพุปัททะวา,มา โหนตุ เว ตัสสะ ปะภาวะสิทธิยา\\}
\indent บุญใดที่ข้าพเจ้าผู้ไหว้อยู่ซึ่งวัตถุสาม, คือพระรัตนตรัยอันควรบูชายิ่งโดยส่วนเดียว,\\
\indent ได้กระทำแล้วเป็นอย่างยิ่งเช่นนี้, ขออุปัทวะ(ความชั่ว) ทั้งหลาย,\\
\indent จงอย่ามีแก่ข้าพเจ้าเลย, ด้วยอำนาจความสำเร็จอันเกิดจากบุญนั้น\\
\pagebreak
\section{สังเวคปริกิตตนปาฐะ}
\hrule
\textbf{อิธะ ตะถาคะโต โลเก อุปปันโน\\}
\indent พระตถาคตเจ้าเกิดขึ้นแล้วในโลกนี้\\
\textbf{อะระหัง สัมมาสัมพุทโธ\\}
\indent เป็นผู้ไกลจากกิเลส ตรัสรู้ชอบได้โดยพระองค์เอง\\
\textbf{ธัมโม จะ เทสิโต นิยยานิโก\\}
\indent และพระธรรมที่ทรงแสดงเป็นธรรมเครื่องออกจากทุกข์\\
\textbf{อุปะสะมิโก ปะรินิพพานิโก\\}
\indent เป็นเครื่องสงบกิเลส, เป็นไปเพื่อปรินิพพาน\\
\textbf{สัมโพธะคามี สุคะตัปปะเวทิโต\\}
\indent เป็นไปเพื่อความรู้พร้อม, เป็นธรรมที่พระสุคตประกาศ\\
\textbf{มะยันตัง ธัมมัง สุต๎วา เอวัง ชานามะ\\}
\indent พวกเราเมื่อได้ฟังธรรมนั้นแล้ว, จึงได้รู้อย่างนี้ว่า\\
\textbf{ชาติปิ ทุกขา\\}
\indent แม้ความเกิดก็เป็นทุกข์\\
\textbf{ชะราปิ ทุกขา\\}
\indent แม้ความแก่ก็เป็นทุกข์\\
\textbf{มะระณัมปิ ทุกขัง\\}
\indent แม้ความตายก็เป็นทุกข์\\
\textbf{โสกะปะริเทวะทุกขะโทมะนัสสุปายาสาปิ ทุกขา\\}
\indent แม้ความโศก ความร่ำไรรำพัน ความไม่สบายกาย\\
\indent ความไม่สบายใจ ความคับแค้นใจ ก็เป็นทุกข์\\
\textbf{อัปปิเยหิ สัมปะโยโค ทุกโข\\}
\indent ความประสบกับสิ่งไม่เป็นที่รักที่พอใจ ก็เป็นทุกข์\\
\textbf{ปิเยหิ วิปปะโยโค ทุกโข\\}
\indent ความพลัดพรากจากสิ่งเป็นที่รักที่พอใจ ก็เป็นทุกข์\\
\textbf{ยัมปิจฉัง นะ ละภะติ ตัมปิ ทุกขัง\\}
\indent มีความปรารถนาสิ่งใดไม่ได้สิ่งนั้นนั่นก็เป็นทุกข์\\
\textbf{สังขิตเตนะ ปัญจุปาทานักขันธา ทุกขา\\}
\indent ว่าโดยย่อ อุปาทานขันธ์ทั้ง ๕ เป็นตัวทุกข์\\
\textbf{เสยยะถีทัง\\}
\indent ได้แก่สิ่งเหล่านี้คือ\\
\textbf{รูปูปาทานักขันโธ\\}
\indent ขันธ์ อันเป็นที่ตั้งแห่งความยึดมั่น คือรูป\\
\textbf{เวทะนูปาทานักขันโธ\\}
\indent ขันธ์ อันเป็นที่ตั้งแห่งความยึดมั่น คือเวทนา\\
\textbf{สัญญูปาทานักขันโธ\\}
\indent ขันธ์ อันเป็นที่ตั้งแห่งความยึดมั่น คือสัญญา\\
\textbf{สังขารูปาทานักขันโธ\\}
\indent ขันธ์ อันเป็นที่ตั้งแห่งความยึดมั่น คือสังขาร\\
\textbf{วิญญาณูปาทานักขันโธ\\}
\indent ขันธ์ อันเป็นที่ตั้งแห่งความยึดมั่น คือวิญญาณ\\
\textbf{เยสัง ปะริญญายะ\\}
\indent เพื่อให้สาวกกำหนดรอบรู้อุปาทานขันธ์เหล่านี้เอง\\
\textbf{ธะระมาโน โส ภะคะวา\\}
\indent จึงพระผู้มีพระภาคเจ้านั้น เมื่อยังทรงพระชนม์อยู่\\
\textbf{เอวัง พะหุลัง สาวะเก วิเนติ\\}
\indent ย่อมทรงแนะนำสาวกทั้งหลาย, เช่นนี้เป็นส่วนมาก\\
\textbf{เอวัง ภาคา จะ ปะนัสสะ ภะคะวะโต สาวะเกสุ อะนุสาสะนี พะหุลา ปะวัตตะติ\\}
\indent อนึ่งคำสั่งสอนของพระผู้มีพระภาคเจ้านั้นย่อมเป็นไปในสาวกทั้งหลาย,\\
\indent ส่วนมากมีส่วนคือการจำแนกอย่างนี้ว่า\\
\textbf{รูปัง อะนิจจัง\\}
\indent รูปไม่เที่ยง\\
\textbf{เวทะนา อะนิจจา\\}
\indent เวทนาไม่เที่ยง\\
\textbf{สัญญา อะนิจจา\\}
\indent สัญญาไม่เที่ยง\\
\textbf{สังขารา อะนิจจา\\}
\indent สังขารไม่เที่ยง\\
\textbf{วิญญาณัง อะนิจจัง\\}
\indent วิญญาณไม่เที่ยง\\
\textbf{รูปัง อะนัตตา\\}
\indent รูปไม่ใช่ตัวตน\\
\textbf{เวทะนา อะนัตตา\\}
\indent เวทนาไม่ใช่ตัวตน\\
\textbf{สัญญา อะนัตตา\\}
\indent สัญญาไม่ใช่ตัวตน\\
\textbf{สังขารา อะนัตตา\\}
\indent สังขารไม่ใช่ตัวตน\\
\textbf{วิญญาณัง อะนัตตา\\}
\indent วิญญาณไม่ใช่ตัวตน\\
\textbf{สัพเพ สังขารา อะนิจจา\\}
\indent สังขารทั้งหลายทั้งปวง ไม่เที่ยง\\
\textbf{สัพเพ ธัมมา อะนัตตาติ\\}
\indent ธรรมทั้งหลายทั้งปวง ไม่ใช่ตัวตน, ดังนี้\\
\textbf{เต (ตา)  มะยัง โอติณณามหะ\\}
\indent พวกเราทั้งหลายเป็นผู้ถูกครอบงำแล้ว\\
\textbf{ชาติยา\\}
\indent โดยความเกิด\\
\textbf{ชะรามะระเณนะ\\}
\indent โดยความแก่และความตาย\\
\textbf{โสเกหิ ปะริเทเวหิ ทุกเขหิ โทมะนัสเสหิ อุปายาเสหิ\\}
\indent โดยความโศก ความร่ำไรรำพัน ความไม่สบายกาย\\
\indent ความไม่สบายใจ ความคับแค้นใจทั้งหลาย\\
\textbf{ทุกโขติณณา\\}
\indent เป็นผู้ถูกความทุกข์หยั่งเอาแล้ว\\
\textbf{ทุกขะปะเรตา\\}
\indent เป็นผู้มีความทุกข์เป็นเบื้องหน้าแล้ว\\
\textbf{อัปเปวะนามิมัสสะ เกวะลัสสะ ทุกขักขันธัสสะ อันตะกิริยา ปัญญาเยถาติ\\}
\indent ทำไฉนการทำที่สุดแห่งกองทุกข์ทั้งสิ้นนี้, จะพึงปรากฏชัดแก่เราได้\\
\begin{center}\textbf{สำหรับ พระภิกษุ - สามเณรสวด}\end{center}
\textbf{จิระปะรินิพพุตัมปิ ตัง ภะคะวันตัง อุททิสสะ อะระหันตัง สัมมาสัมพุทธัง\\}
\indent เราทั้งหลาย อุทิศเฉพาะพระผู้มีพระภาคเจ้า, ผู้ไกลจากกิเลส\\
\indent ตรัสรู้ชอบได้โดยพระองค์เอง, แม้ปรินิพพานนานแล้ว, พระองค์นั้น\\
\textbf{สัทธา อะคารัส๎มา อะนะคาริยัง ปัพพะชิตา\\}
\indent เป็นผู้มีศรัทธา ออกบวชจากเรือน ไม่เกี่ยวข้องด้วยเรือนแล้ว\\
\textbf{ตัส๎มิง ภะคะวะติ พ๎รห๎มะจะริยัง จะรามะ\\}
\indent ประพฤติอยู่ซึ่งพรหมจรรย์ ในพระผู้มีพระภาคเจ้า พระองค์นั้น\\
\textbf{ภิกขูนัง สิกขาสาชีวะสะมาปันนา\\}
\indent ถึงพร้อมด้วยสิกขาและธรรมเป็นเครื่องเลี้ยงชีวิตของภิกษุทั้งหลาย\\
\textbf{ตัง โน พ๎รห๎มะจะริยัง อิมัสสะ เกวะลัสสะ ทุกขักขันธัสสะ อันตะ กิริยายะ สังวัตตะตุ\\}
\indent ขอให้พรหมจรรย์ของเราทั้งหลายนั้น, จงเป็นไปเพื่อการทำที่สุดแห่งกอง ทุกข์ทั้งสิ้นนี้ เทอญ\\
\begin{center}\textbf{สำหรับอุบาสก, อุบาสิกา}\end{center}
\textbf{จิระปะรินิพพุตัมปิ ตัง ภะคะวันตัง สะระณังคะตา\\}
\indent เราทั้งหลาย ผู้ถึงแล้วซึ่งพระผู้มีพระภาคเจ้า,\\
\indent แม้ปรินิพพานนานแล้วพระองค์นั้น เป็นสรณะ\\
\textbf{ธัมมัญจะ สังฆัญจะ}\\
\indent ถึงพระธรรมด้วย ถึงพระสงฆ์ด้วย\\
\textbf{ตัสสะ ภะคะวะโต สาสะนัง, ยะถาสะติ, ยะถาพะลัง\\
มะนะสิกะโรมะ อะนุปะฏิปัชชามะ}\\
\indent จักทำในใจอยู่ ปฏิบัติตามอยู่ ซึ่งคำสั่งสอนของพระผู้มีพระภาคเจ้านั้น\\
\indent ตามสติกำลัง\\
\textbf{สา สา โน ปะฏิปัตติ}\\
\indent ขอให้ความปฏิบัตินั้นๆ ของเราทั้งหลาย\\
\textbf{อิมัสสะ เกวะลัสสะ ทุกขักขันธัสสะ อันตะกิริยายะ สังวัตตะตุ}\\
\indent จงเป็นไปเพื่อการทำที่สุดแห่งกองทุกข์ทั้งสิ้นนี้ เทอญ\\
\begin{center}
(จบคำทำวัตรเช้า)
\end{center}
\pagebreak
\section{คำแผ่เมตตา}
\hrule
\begin{center}(หันทะ มะยัง เมตตาผะระณัง กะโรมะ เส)\end{center}
\textbf{อะหัง สุขิโต โหมิ\\}
\indent ขอให้ข้าพเจ้าจงเป็นผู้ถึงสุข\\
\textbf{นิททุกโข โหมิ\\}
\indent จงเป็นผู้ไร้ทุกข์\\
\textbf{อะเวโร โหมิ\\}
\indent จงเป็นผู้ไม่มีเวร\\
\textbf{อัพฺยาปัชโฌ โหมิ \\}
\indent จงเป็นผู้ไม่เบียดเบียนซึ่งกันและกัน\\
\textbf{อะนีโฆ โหมิ\\}
\indent จงเป็นผู้ไม่มีทุกข์\\
\textbf{สุขี อัตตานัง ปะริหะรามิ\\}
\indent จงรักษาตนอยู่เป็นสุขเถิด\\
\textbf{สัพเพ สัตตา สุขิตา โหนตุ\\}
\indent ขอสัตว์ทั้งหลายทั้งปวงจงเป็นผู้ถึงความสุข\\
\textbf{สัพเพ สัตตา อะเวรา โหนตุ\\}
\indent ขอสัตว์ทั้งหลายทั้งปวงจงเป็นผู้ไม่มีเวร\\
\textbf{สัพเพ สัตตา อัพฺยาปัชฌา โหนตุ\\}
\indent ขอสัตว์ทั้งหลายทั้งปวงจงอย่าได้เบียดเบียนซึ่งกันและกัน\\
\textbf{สัพเพ สัตตา อะนีฆา โหนตุ\\}
\indent ขอสัตว์ทั้งหลายทั้งปวงจงเป็นผู้ไม่มีทุกข์\\
\textbf{สัพเพ สัตตา สุขี อัตตานัง ปะริหะรันตุ\\}
\indent ขอสัตว์ทั้งหลายทั้งปวงจงรักษาตนอยู่เป็นสุขเถิด\\
\textbf{สัพเพ สัตตา สัพพะทุกขา ปะมุญจันตุ\\}
\indent ขอสัตว์ทั้งหลายทั้งปวงจงพ้นจากทุกข์ทั้งมวล\\
\textbf{สัพเพ สัตตา ลัทธะสัมปัตติโต มา วิคัจฉันตุ\\}
\indent ขอสัตว์ทั้งหลายทั้งปวงจงอย่าได้พรากจากสมบัติอันตนได้แล้ว\\
\textbf{สัพเพ สัตตา กัมมัสสะกา กัมมะทายาทา\\
กัมมะโยนิ กัมมะพันธุ กัมมะปะฏิสะระณา\\}
\indent สัตว์ทั้งหลายทั้งปวงมีกรรมเป็นของของตน, มีกรรมเป็นผู้ให้ผล,\\
\indent มีกรรมเป็นแดนเกิด, มีกรรมเป็นผู้ติดตาม,มีกรรมเป็นที่พึ่งอาศ้ย\\
\textbf{ยัง กัมมัง กะรัสสันติ, กัลฺยาณัง วา ปาปะกัง วา, ตัสสะ ทายาทา ภาวิสสันติ\\}
\indent จักทำกรรมอันใดไว้, เป็นบุญหรือเป็นบาป, จักต้องเป็นผู้ได้รับผลกรรมนั้นๆ สืบไป

\pagebreak
\section{ท๎วัตติงสาการปาฐะ}
\hrule
\begin{center}
\textbf{(หันทะ มะยัง ท๎วัตติงสาการะปาฐัง ภะณามะ เส)\\}
เชิญเถิด เราทั้งหลาย จงกล่าวคาถาแสดงอาการ ๓๒ ในร่างกายเถิด
\end{center}
\textbf{อะยัง โข เม กาโย\\}
\indent กายของเรานี้แล\\
\textbf{อุทธัง ปาทะตะลา\\}
\indent เบื้องบนแต่พื้นเท้าขึ้นมา\\
\textbf{อะโธ เกสะมัตถะกา\\}
\indent เบื้องต่ำแต่ปลายผมลงไป\\
\textbf{ตะจะปะริยันโต\\}
\indent มีหนังหุ้มอยู่เป็นที่สุดรอบ\\
\textbf{ปูโรนานัปปะการัสสะ อะสุจิโน\\}
\indent เต็มไปด้วยของไม่สะอาด มีประการต่างๆ\\
\textbf{อัตถิ อิมัสมิง กาเย \\}
\indent มีอยู่ในกายนี้\\
\textbf{เกสา \\}
\indent คือผมทั้งหลาย\\
\textbf{โลมา} \\
\indent คือขนทั้งหลาย\\
\textbf{นะขา} \\
\indent คือเล็บทั้งหลาย\\
\textbf{ทันตา} \\
\indent คือฟันทั้งหลาย\\
\textbf{ตะโจ} \\
\indent หนัง\\
\textbf{มังสัง} \\
\indent เนื้อ\\
\textbf{นะหารู} \\
\indent เอ็นทั้งหลาย\\
\textbf{อัฏฐิ} \\
\indent กระดูกทั้งหลาย\\
\textbf{อัฏฐิมิญชัง} \\
\indent เยื่อในกระดูก\\
\textbf{วักกัง} \\
\indent ไต\\
\textbf{หะทะยัง} \\
\indent หัวใจ\\
\textbf{ยะกะนัง} \\
\indent ตับ\\
\textbf{กิโลมะกัง} \\
\indent พังผืด\\
\textbf{ปิหะกัง} \\
\indent ม้าม\\
\textbf{ปัปผาสัง} \\
\indent ปอด\\
\textbf{อันตัง} \\
\indent ไส้ใหญ่\\
\textbf{อันตะคุณัง} \\
\indent สายรัดไส้\\
\textbf{อุทะริยัง} \\
\indent อาหารใหม่\\
\textbf{กะรีสัง} \\
\indent อาหารเก่า\\
\textbf{ปิตตัง} \\
\indent น้ำดี\\
\textbf{เสมหัง} \\
\indent น้ำเสลด\\
\textbf{ปุพโพ} \\
\indent น้ำเหลือง\\
\textbf{โลหิตัง} \\
\indent น้ำเลือด\\
\textbf{เสโท} \\
\indent น้ำเหงื่อ\\
\textbf{เมโท} \\
\indent น้ำมันข้น\\
\textbf{อัสสุ} \\
\indent น้ำตา\\
\textbf{วะสา} \\
\indent น้ำมันเหลว\\
\textbf{เขโฬ} \\
\indent น้ำลาย\\
\textbf{สิงฆาณิกา} \\
\indent น้ำมูก\\
\textbf{ละสิกา} \\
\indent น้ำมันไขข้อ\\
\textbf{มุตตัง} \\
\indent น้ำมูตร\\
\textbf{มัตถะเกมัตถะลุงคัง} \\
\indent เยื่อในสมอง\\
\textbf{เอวะมะยังเม กาโย} \\
\indent กายของเรานี้อย่างนี้\\
\textbf{อุทธังปาทะตะลา} \\
\indent เบื้องบนแต่พื้นเท้าขึ้นมา\\
\textbf{อะโธเกสะมัตถะกา}\\
\indent เบื้องต่ำแต่ปลายผมลงไป\\
\textbf{ตะจะปะริยันโต} \\
\indent มีหนังหุ้มอยู่เป็นที่สุดรอบ\\
\textbf{ปูโรนานัปปะการัสสะ อะสุจิโน}\\
\indent เต็มไปด้วยของไม่สะอาด มีประการต่างๆ อย่างนี้แลฯ\\

\pagebreak
\section{อภิณหปัจจเวกขณปาฐะ}
\hrule
\begin{center}
\textbf{(หันทะ มะยัง อะภิณหะปัจจะเวกขะณะปาฐัง ภะณามะ เส)}\\
เชิญเถิดเราทั้งหลาย มาสวดอภิณหปัจจเวกขณะปาฐะกันเถิด
\end{center}
\textbf{ชะราธัมโมมหิ ชะรัง อะนะตีโต (อะนะตีตา) \\}
\indent เรามีความแก่เป็นธรรมดาจะล่วงพ้นความแก่ไปไม่ได้\\
\textbf{พ๎ยาธิธัมโมมหิ พ๎ยาธิง อะนะตีโต (อะนะตีตา) } \\
\indent เรามีความเจ็บไข้เป็นธรรมดา จะล่วงพ้นความเจ็บไข้ไปไม่ได้\\
\textbf{มะระณะธัมโมมหิ มะระณัง อะนะตีโต (อะนะตีตา)} \\
\indent เรามีความตายเป็นธรรมดาจะล่วงพ้นความตายไปไม่ได้\\
\textbf{สัพเพหิ เม ปิเยหิ มะนาเปหิ นานาภาโว วินาภาโว\\}
\indent เราจะละเว้นเป็นต่างๆ, คือว่าจะต้องพลัดพรากจากของรักของเจริญใจ ทั้งหลายทั้งปวง\\
\textbf{กัมมัสสะโกมหิ กัมมะทายาโท กัมมะโยนิ\\
กัมมะพันธุ กัมมะปะฏิสะระโณ (ณา)  \\}
\indent เราเป็นผู้มีกรรมเป็นของๆตน, มีกรรมเป็นผู้ให้ผล,\\
\indent มีกรรมเป็นแดนเกิด, มีกรรมเป็นผู้ติดตาม, มีกรรมเป็นที่พึ่งอาศัย\\
\textbf{ยัง กัมมัง กะริสสามิ, กัลยาณัง วา ปาปะกัง วา,\\
ตัสสะ ทายาโท (ทา)  ภะวิสสาม\\}
\indent เราทำกรรมอันใดไว้, เป็นบุญหรือเป็นบาป,\\
\indent เราจะเป็นทายาท, คือว่าเราจะต้องได้รับผลของกรรมนั้นๆ สืบไป\\
\textbf{เอวัง อัมเหหิ อะภิณหัง ปัจจะเวกขิตัพพัง\\}
\indent เราทั้งหลายควรพิจารณาอย่างนี้ทุกวันๆ เถิด

\pagebreak
\section{บทพิจารณาสังขาร }
\hrule
\begin{center}
\textbf{ (หันทะ มะยัง ธัมมะสังเวคะปัจจะเวกขะณะปาฐัง ภะณามะ เส)\\}
เชิญเถิด เราทั้งหลาย จงกล่าวคาถาพิจารณาธรรมสังเวชเถิด
\end{center}
\textbf{สัพเพ สังขารา อะนิจจา\\}
\indent สังขารคือร่างกายจิตใจ, แลรูปธรรมนามธรรมทั้งหมดทั้งสิ้น,\\
\indent มันไม่เที่ยง, เกิดขึ้นแล้วดับไปมีแล้วหายไป\\
\textbf{สัพเพ สังขารา ทุกขา\\}
\indent สังขารคือร่างกายจิตใจ, แลรูปธรรมนามธรรมทั้งหมดทั้งสิ้น,\\
\indent มันเป็นทุกข์ทนยาก, เพราะเกิดขึ้นแล้วแก่เจ็บตายไป\\
\textbf{สัพเพ ธัมมา อะนัตตา\\}
\indent สิ่งทั้งหลายทั้งปวง, ทั้งที่เป็นสังขารแลมิใช่สังขารทั้งหมดทั้งสิ้น,\\
\indent ไม่ใช่ตัวไม่ใช่ตน, ไม่ควรถือว่าเราว่าของเราว่าตัวว่าตนของเรา\\
\textbf{อะธุวัง ชีวิตัง\\}
\indent ชีวิตเป็นของไม่ยั่งยืน\\
\textbf{ธุวัง มะระณัง\\}
\indent ความตายเป็นของยั่งยืน\\
\textbf{อะวัสสัง มะยา มะริตัพพัง\\}
\indent อันเราจะพึงตายเป็นแท้\\
\textbf{มะระณะปะริโยสานัง เม ชีวิตัง\\}
\indent ชีวิตของเรามีความตายเป็นที่สุดรอบ\\
\textbf{ชีวิตัง เม อะนิยะตัง\\}
\indent ชีวิตของเราเป็นของไม่เที่ยง\\
\textbf{มะระณัง เม นิยะตัง\\}
\indent ความตายของเราเป็นของเที่ยง\\
\textbf{วะตะ\\}
\indent ควรที่จะสังเวช\\
\textbf{อะยัง กาโย อะจิรัง\\}
\indent ร่างกายนี้มิได้ตั้งอยู่นาน\\
\textbf{อะเปตะวิญญาโณ\\}
\indent ครั้นปราศจากวิญญาณ\\
\textbf{ฉุฑโฑ\\}
\indent อันเขาทิ้งเสียแล้ว\\
\textbf{อธิเสสสะติ\\}
\indent จักนอนทับ\\
\textbf{ปะฐะวิง\\}
\indent ซึ่งแผ่นดิน\\
\textbf{กะลิงคะรัง อิวะ\\}
\indent ประดุจดังว่าท่อนไม้และท่อนฟืน\\
\textbf{นิรัตถัง\\}
\indent หาประโยชน์มิได้\\
\textbf{อะนิจจา วะตะ สังขารา\\}
\indent สังขารทั้งหลายไม่เที่ยงหนอ\\
\textbf{อุปปาทะวะยะธัมมิโน\\}
\indent มีความเกิดขึ้นแล้วมีความเสื่อมไปเป็นธรรมดา\\
\textbf{อุปปัชชิตฺวา นิรุชฌันติ\\}
\indent ครั้นเกิดขึ้นแล้วย่อมดับไป\\
\textbf{เตสัง วูปะสะโม สุโข\\}
\indent ความเข้าไปสงบระงับสังขารทั้งหลาย, เป็นสุขอย่างยิ่ง, ดังนี้\\

\pagebreak
\section{ตังขณิกปัจจเวกขณปาฐะ}
\hrule
\begin{center}
\textbf{(นำ) หันทะ มะยัง ตังขณิกกะปัจจะเวกขณะปาฐัง ภะณามะ เส}
\end{center}
\textbf{ปะฏิสังขา โยนิโส จีวะรัง ปะฏิเสวามิ\\}
\indent เราย่อมพิจาราโดยแยบคายแล้วนุ่งห่มจีวร\\
\textbf{ยาวะเทวะสีตัสสะปฏิฆาตายะ\\}
\indent เพียงเพื่อบำบัดความหนาว\\
\textbf{อุณหัสสะ ปฏิฆาตายะ\\}
\indent เพื่อบำบัดความร้อน\\
\textbf{ฑังสะมะกะสะวาตาตะปะสิริงสะปะสัมผัสสานังปฏิฆาตายะ\\}
\indent เพื่อบำบัดสัมผัสอันเกิดจากเหลือบ ยุง ลม แดดและสัตว์เลื้อยคลานทั้งหลาย\\
\textbf{ยาวะเทวะ หิริโกปินนะปะฏิจฉาทะนัตถัง\\}
\indent และเพียงเพื่อปกปิดอวัยวะอันให้เกิดจากความละอาย\\
\textbf{ปฏิสังขา โยนิโส ปิณฑะปาตัง ปะฏิเสวามิ\\}
\indent เราย่อมพิจารณาโดยแยบคายแล้วฉันบิณฑบาต\\
\textbf{เนวะ ทะวายะ\\}
\indent ไม่ให้เป็นไปเพื่อความเพลิดเพลินสนุกสนาน\\
\textbf{นะ มะทายะ\\}
\indent ไม่ให้เป็นไปเพื่อความเมามันเกิดกำลังพลังทางกาย\\
\textbf{นะ มัณฑะนายะ\\}
\indent ไม่ให้เป็นไปเพื่อประดับ\\
\textbf{นะ วิภูสะนายะ\\}
\indent ไม่ให้เป็นไปพื่อตกแต่ง\\
\textbf{ยะวะเทวะ อิมัสสะ กายัสสะ ฐิติยา\\}
\indent แต่ให้เป็นไปเพียงเพื่อความตั้งอยู่ได้แห่งกายนี้\\
\textbf{ยาปะนายะ\\}
\indent เพื่อความเป็นไปได้ของอัตภาพ\\
\textbf{วิหิงสุปะระติยา\\}
\indent เพื่อความสิ้นไปแห่ความลำบากทางกาย\\
\textbf{พรัหมะจะริยานุคคะหายะ\\}
\indent เพื่ออนุเคราะห์แก่การประพฤติพรหมจรรย์\\
\textbf{อิติ ปุราณัญจะ เวทะนัง ปะฏิหังขามิ\\}
\indent ด้วยการทำอย่างนี้ เราย่อมระงับเสียได้ซึ่งทุกขเวทนาเก่า คือความหิว\\
\textbf{นะวัญจะเวทะนัง นะ อุปปาเทสสามิ\\}
\indent และไม่ทำทุกขเวทนาใหม่ให้เกิดขึ้น\\
\textbf{ยาตฺรา จะ เม ภะวิสสะติ อะนะวัชชะตา จะ ผาสุวิหาโร จาติ\\}
\indent อนึ่งความเป็นไปโดยสะดวกแห่งอัตภาพนี้ด้วย ความเป็นผู้หาโทษมิได้ด้วย\\ 
\indent และความเป็นอยู่โดยผาสุขด้วย จักมีแก่เรา ดังนี้\\
\textbf{ปะฏิสังขา โยนิโส เสนาสะนังปะฏิเสวามิ\\}
\indent เราย่อมพิจารณาโดยแยบคายแล้วใช้สอยเสนาสนะ
\textbf{ยาวะเทวะสีตัสสะ ปะฏิฆาตายะ}\\
\indent เพียงเพื่อบำบัดความหนาว\\
\textbf{อุณหัสสะปะฏิฆาตายะ}\\
\indent เพื่อบำบัดความร้อน\\
\textbf{ฑังสะมะกะสะวาตาตะปะสิริงสะปะสัมผัสสานัง ปะฏิฆาตายะ}\\
\indent เพื่อบำบัดสัมผัสอันเกิดจาก เหลือบยุงลมแดดและสัตว์เลื้อยคลานทั้งหลาย\\
\textbf{ยะวะเทวะ อุตุปะริสสะยะวิโนทะนัง ปะฏิสัลลานารามัตถัง}\\
\indent เพียงเพื่อบรรเทาอันตรายอันจะพึงมีจากดินฟ้าอากาศ\\ 
\indent และเพื่อความเป็นผู้ยินดีอยู่ได้ในที่หลีกเร้นสำหรับภาวนา\\
\textbf{ปะฏิสังขา โยนิโส คิลานะปัจจะยะเภสัชชปะริกขารัง ปะฏิเสวามิ}\\
\indent เราย่อมพิจาณาโยแยบคายแล้วบริโภคเภสัชบริชารอันเกื้อกูลแก่คนไข้\\
\textbf{ยะวะเวทวะ อุปปันนานัง เวยยาพาธิกานัง เวทะนานัง ปะฏิฆาตายะ}\\
\indent เพียงเพื่อบำบัดทุกขเวทนาอันบังเกิดขึ้นแล้ว มีอาพาธต่าง ๆ เป็นมูล\\
\textbf{อัพฺยาปัชณะปะระมะตายาติ}\\
\indent เพื่อความเป็นผู้ไม่มีโรคเบียดเบียนเป็นอย่างยิ่ง ดังนี้\\

\pagebreak
\section{สัพพปัตติทานคาถา}
\hrule
\begin{center}
\textbf{(หันทะ มะยัง สัพพะปัตติทานะคาถาโย ภะณามะ เส)}\\
เชิญเถิด เราทั้งหลาย จงกล่าวคำแผ่ส่วนบุญให้แก่สรรพสัตว์ทั้งหลายเถิด
\end{center}
\textbf{ปุญญัสสิทานิ กะตัสสะ ยานัญญานิ กะตานิ เม,\\
เตสัญจะ ภาคิโน โหนตุ สัตตานันตาปปะมาณะกา}\\
\indent สัตว์ทั้งหลาย ไม่มีที่สุด ไม่มีประมาณ, จงมีส่วนแห่งบุญที่ข้าพเจ้าได้ทำในบัดนี้,\\
\indent และแห่งบุญอื่นที่ได้ทำไว้ก่อนแล้ว \\
\textbf{เย ปิยา คุณะวันตา จะ มัยหัง มาตาปิตาทะโย,\\
ทิฏฐา เม จาป๎ยะทิฏฐา วา อัญเญ มัชฌัตตะเวริโน}\\
\indent คือจะเป็นสัตว์เหล่าใด, ซึ่งเป็นที่รักใคร่และมีบุญคุณ,\\
\indent เช่นมารดาบิดาของข้าพเจ้าเป็นต้น ก็ดี ที่ข้าพเจ้าเห็นแล้วหรือไม่ได้เห็น \\
\indent ก็ดี, สัตว์เหล่าอื่นที่เป็นกลางๆหรือเป็นคู่เวรกัน ก็ดี\\
\textbf{สัตตา ติฏฐันติ โลกัส๎มิง เตภุมมา จะตุโยนิกา,\\
ปัญเจกะจะตุโวการา สังสะรันตา ภะวาภะเว}\\
\indent สัตว์ทั้งหลายตั้งอยู่ในโลก, อยู่ในภูมิทั้งสาม, อยู่ในกำเนิดทั้งสี่,\\
\indent มีขันธ์ห้าขันธ์ มีขันธ์ขันธ์เดียว มีขันธ์สี่ขันธ์, กำลังท่องเที่ยวอยู่ในภพ\\
\indent น้อยภพใหญ่ ก็ดี\\
\textbf{ญาตัง เย ปัตติทานัมเม อะนุโมทันตุ เต สะยัง,\\
เย จิมัง นัปปะชานันติ เทวา เตสัง นิเวทะยุง,}\\ 
\indent \m สัตว์เหล่าใด รู้ส่วนบุญที่ข้าพเจ้าแผ่ให้แล้ว, สัตว์เหล่านั้นจงอนุโมทนาเองเถิด,\\
\indent ส่วนสัตว์เหล่าใดยังไม่รู้ส่วนบุญนี้, ขอเทวดาทั้งหลาย, จงบอกสัตว์เหล่านั้นให้รู้\\
\indent มะยา ทินนานะ ปุญญานัง อะนุโมทะนะเหตุนา,\\
\textbf{สัพเพ สัตตา สะทา โหนตุ อะเวรา สุขะชีวิโน,\\
เขมัปปะทัญจะ ปัปโปนตุ เตสาสา สิชฌะตัง สุภา}\\
\indent เพราะเหตุที่ได้อนุโมทนาส่วนบุญที่ข้าพเจ้าแผ่ให้แล้ว,\\
\indent สัตว์ทั้งหลายทั้งปวง, จงเป็นผู้ไม่มีเวร, อยู่เป็นสุขทุกเมื่อ, จงถึงบทอัน\\
\indent เกษม, กล่าวคือพระนิพพาน, ความปรารถนาที่ดีงามของสัตว์เหล่านั้นจงสำเร็จ เถิด\\

\pagebreak
\vspace*{\fill}
\begin{center}
  \scalebox{3}{\textbf{คำทำวัตรเย็น}}
\end{center}
\vspace{\fill}

\pagebreak

\section{คำบูชาพระรัตนตรัย}
\hrule
\textbf{โย โส ภะคะวา อะระหัง สัมมาสัมพุทโธ,}\\
\indent พระผู้มีพระภาคเจ้านั้น พระองค์ใด, เป็นพระอรหันต์,\\
\indent ดับเพลิงกิเลสเพลิงทุกข์สิ้นเชิง, ตรัสรู้ชอบได้โดยพระองค์เอง\\
\textbf{ส๎วากขาโต เยนะ ภะคะวะตา ธัมโม, }\\
\indent พระธรรม เป็นธรรมอันพระผู้มีพระภาคเจ้า พระองค์ใด, ตรัสไว้ดีแล้ว\\
\textbf{สุปะฏิปันโน ยัสสะ ภะคะวะโต สาวะกะสังโฆ,}\\
\indent พระสงฆ์สาวกของพระผู้มีพระภาคเจ้า พระองค์ใด, ปฏิบัติดีแล้ว\\
\textbf{ตัมมะยัง ภะคะวันตัง สะธัมมัง สะสังฆัง,\\
อิเมหิ สักกาเรหิ ยะถาระหัง อาโรปิเตหิ อะภิปูชะยามะ,}\\
\indent ข้าพเจ้าทั้งหลาย, ขอบูชาอย่างยิ่งซึ่งพระผู้มีพระภาคเจ้าพระองค์นั้น,\\
\indent พร้อมทั้งพระธรรมและพระสงฆ์,\\
\indent ด้วยเครื่องสักการะทั้งหลายเหล่านี้, อันยกขึ้นตามสมควรแล้วอย่างไร\\
\textbf{สาธุ โน ภันเต ภะคะวา สุจิระปะรินิพพุโตปิ,}\\
\indent ข้าแต่พระองค์ผู้เจริญ, พระผู้มีพระภาคเจ้าแม้ปรินิพพานนานแล้ว,\\
\indent ทรงสร้างคุณอันสำเร็จประโยชน์ไว้แก่ข้าพเจ้าทั้งหลาย.\\
\textbf{ปัจฉิมาชะนะตานุกัมปะมานะสา,}\\
\indent ทรงมีพระหฤทัยอนุเคราะห์แก่ข้าพเจ้า อันเป็นชนรุ่นหลัง\\
\textbf{อิเม สักกาเร ทุคคะตะปัณณาการะภูเต ปะฏิคคัณหาตุ,}\\
\indent ขอพระผู้มีพระภาคเจ้าจงรับเครื่องสักการะ อันเป็นบรรณาการของคนยากทั้งหลายเหล่านี้\\
\textbf{อัมหากัง ทีฆะรัตตัง หิตายะ สุขายะ,}\\
\indent เพื่อประโยชน์และความสุขแก่พวกข้าพเจ้าทั้งหลาย ตลอดกาลนาน เทอญฯ\\
\textbf{อะระหัง สัมมาสัมพุทโธ ภะคะวา,}\\
\indent พระผู้มีพระภาคเจ้า, เป็นพระอรหันต์, ดับเพลิงกิเลสเพลิงทุกข์สิ้นเชิง,\\
\indent ตรัสรู้ชอบได้โดยพระองค์เอง\\
\textbf{พุทธัง ภะคะวันตัง อะภิวาเทมิ,}\\
\indent ข้าพเจ้าอภิวาทพระผู้มีพระภาคเจ้า, ผู้รู้ ผู้ตื่น ผู้เบิกบาน (กราบ)\\
\textbf{ส๎วากขาโต ภะคะวะตา ธัมโม}\\
\indent พระธรรมเป็นธรรมที่พระผู้มีพระภาคเจ้า, ตรัสไว้ดีแล้ว\\
\textbf{ธัมมัง นะมัสสามิ}\\
\indent ข้าพเจ้านมัสการพระธรรม (กราบ)\\
\textbf{สุปะฏิปันโน ภะคะวะโต สาวะกะสังโฆ,}\\
\indent พระสงฆ์สาวกของพระผู้มีพระภาคเจ้า, ปฏิบัติดีแล้ว\\
\textbf{สังฆัง นะมามิ.}\\
\indent ข้าพเจ้านอบน้อมพระสงฆ์ (กราบ)\\

\pagebreak

\section{ปุพพภาคนมการ}
\hrule
\begin{center}
\textbf{(หันทะ มะยัง พุทธัสสะ ภะคะวะโต ปุพพะภาคะนะมะการัง กะโร มะ เส)}\\
เชิญเถิด เราทั้งหลาย ทำความนอบน้อมอันเป็นส่วนเบื้องต้น แด่พระผู้มีพระภาคเจ้าเถิด
\end{center}
\textbf{นะโม ตัสสะ ภะคะวะโต,}\\
\indent ขอนอบน้อมแด่พระผู้มีพระภาคเจ้าพระองค์นั้น\\
\textbf{อะระหะโต,}\\
\indent ซึ่งเป็นผู้ไกลจากกิเลส\\
\textbf{สัมมาสัมพุทธัสสะ.}\\
\indent ตรัสรู้ชอบได้โดยพระองค์เอง\\.
\begin{center}
(กล่าว ๓ ครั้ง)
\end{center}

\pagebreak

\section{พุทธานุสสติ}
\hrule
\begin{center}
\textbf{(หันทะ มะยัง พุทธานุสสะตินะยัง กะโรมะ เส)}\\
เชิญเถิด เราทั้งหลาย ทำความตามระลึกถึงพระพุทธเจ้าเถิด
\end{center}
\textbf{ตัง โข ปะนะ ภะคะวันตัง เอวัง กัล๎ยาโณ กิตติสัทโท อัพภุคคะโต,}\\
\indent ก็กิตติศัพท์อันงามของพระผู้มีพระภาคเจ้านั้น, ได้ฟุ้งไปแล้วอย่างนี้ว่า:-\\
\textbf{อิติปิ โส ภะคะวา,}\\
\indent เพราะเหตุอย่างนี้ๆ, พระผู้มีพระภาคเจ้านั้น\\
\textbf{อะระหัง,}\\
\indent เป็นผู้ไกลจากกิเลส\\
\textbf{สัมมาสัมพุทโธ,}\\
\indent เป็นผู้ตรัสรู้ชอบได้โดยพระองค์เอง\\
\textbf{วิชชาจะระณะสัมปันโน,}\\
\indent เป็นผู้ถึงพร้อมด้วยวิชชาและจรณะ\\
\textbf{สุคะโต,}\\
\indent เป็นผู้ไปแล้วด้วยดี\\
\textbf{โลกะวิทู,}\\
\indent เป็นผู้รู้โลกอย่างแจ่มแจ้ง\\
\textbf{อะนุตตะโร ปุริสะทัมมะสาระถิ,}\\
\indent เป็นผู้สามารถฝึกบุรุษที่สมควรฝึกได้อย่างไม่มีใครยิ่งกว่า\\
\textbf{สัตถา เทวะมะนุสสานัง,}\\
\indent เป็นครูผู้สอนของเทวดาและมนุษย์ทั้งหลาย\\
\textbf{พุทโธ,}\\
\indent เป็นผู้รู้ ผู้ตื่น ผู้เบิกบานด้วยธรรม\\
\textbf{ภะคะวา ติ.}\\
\indent เป็นผู้มีความจำเริญจำแนกธรรมสั่งสอนสัตว์, ดังนี้\\

\pagebreak

\section{พุทธาภิคีติ}
\hrule
\begin{center}
\textbf{(หันทะ มะยัง พุทธาภิคีติง กะโรมะ เส)}\\
เชิญเถิด เราทั้งหลาย ทำความขับคาถา พรรณนาเฉพาะพระพุทธเจ้าเถิด
\end{center}
\textbf{พุทธ๎ะวาระหันตะวะระตาทิคุณาภิยุตโต,}\\
\indent พระพุทธเจ้าประกอบด้วยคุณ, มีความประเสริฐแห่งอรหันตคุณเป็นต้น\\
\textbf{สุทธาภิญาณะกะรุณาหิ สะมาคะตัตโต,}\\
\indent มีพระองค์อันประกอบด้วยพระญาณ, และพระกรุณาอันบริสุทธิ์\\
\textbf{โพเธสิ โย สุชะนะตัง กะมะลังวะ สูโร,}\\
\indent พระองค์ใดทรงกระทำชนที่ดีให้เบิกบาน, ดุจอาทิตย์ทำบัวให้บาน\\
\textbf{วันทามะหัง ตะมะระณัง สิระสา ชิเนนทัง,}\\
\indent ข้าพเจ้าไหว้พระชินสีห์ผู้ไม่มีกิเลสพระองค์นั้นด้วยเศียรเกล้า\\
\textbf{พุทโธ โย สัพพะปาณีนัง สะระณัง เขมะมุตตะมัง,}\\
\indent พระพุทธเจ้าพระองค์ใดเป็นสรณะอันเกษมสูงสุดของสัตว์ทั้งหลาย\\
\textbf{ปะฐะมานุสสะติฏฐานัง วันทามิ ตัง สิเรนะหัง,}\\
\indent ข้าพเจ้าไหว้พระพุทธเจ้าพระองค์นั้นอันเป็นที่ตั้งแห่งความระลึกองค์ที่หนึ่ง ด้วยเศียรเกล้า\\
\textbf{พุทธัสสาหัส๎มิ ทาโสวะ (ทาสีวะ) พุทโธ เม สามิกิสสะโร,}\\
\indent ข้าพเจ้าเป็นทาสของพระพุทธเจ้า, พระพุทธเจ้าเป็นนายมีอิสระเหนือ ข้าพเจ้า,\\
\textbf{พุทโธ ทุกขัสสะ ฆาตา จะ วิธาตา จะ หิตัสสะ เม,}\\
\indent พระพุทธเจ้าเป็นเครื่องกำจัดทุกข์, และทรงไว้ซึ่งประโยชน์แก่ข้าพเจ้า\\
\textbf{พุทธัสสาหัง นิยยาเทมิ สะรีรัญชีวิตัญจิทัง,}\\
\indent ข้าพเจ้ามอบกายถวายชีวิตนี้แด่พระพุทธเจ้า\\
\textbf{วันทันโตหัง (วันทันตีหัง) จะริสสามิ, พุทธัสเสวะ สุโพธิตัง,}\\
\indent ข้าพเจ้าผู้ไหว้อยู่จักประพฤติตาม, ซึ่งความตรัสรู้ดีของพระพุทธเจ้า\\
\textbf{นัตถิ เม สะระณัง อัญญัง, พุทโธ เม สะระณัง วะรัง,}\\
\indent สรณะอื่นของข้าพเจ้าไม่มี, พระพุทธเจ้าเป็นสรณะอันประเสริฐของข้าพเจ้า\\
\textbf{เอเตนะ สัจจะวัชเชนะ วัฑเฒยยัง สัตถุสาสะเน,}\\
\indent ด้วยการกล่าวคำสัตย์นี้, ข้าพเจ้าพึงเจริญในพระศาสนาของพระศาสดา\\
\textbf{พุทธัง เม วันทะมาเนนะ (วันทะมานายะ), ยัง ปุญญัง ปะสุตัง อิธะ,}\\
\indent ข้าพเจ้าผู้ไหว้อยู่ซึ่งพระพุทธเจ้า, ได้ขวนขวายบุญใดในบัดนี้\\
\textbf{สัพเพปิ อันตะรายา เม, มาเหสุง ตัสสะ เตชะสา.}\\
\indent อันตรายทั้งปวงอย่าได้มีแก่ข้าพเจ้าด้วยเดชแห่งบุญนั้น\\
\begin{center}
(หมอบกราบ)
\end{center}
\textbf{กาเยนะ วาจายะ วะ เจตะสา วา,}\\
\indent ด้วยกายก็ดี ด้วยวาจาก็ดี ด้วยใจก็ดี\\
\textbf{พุทเธ กุกัมมัง ปะกะตัง มะยา ยัง,}\\
\indent กรรมน่าติเตียนอันใด ที่ข้าพเจ้ากระทำแล้ว ในพระพุทธเจ้า\\
\textbf{พุทโธ ปะฏิคคัณ๎หะตุ อัจจะยันตัง,}\\
\indent ขอพระพุทธเจ้าจงงดซึ่งโทษล่วงเกินอันนั้น\\
\textbf{กาลันตะเร สังวะริตุง วะ พุทเธ.}\\
\indent เพื่อการสำรวมระวัง ในพระพุทธเจ้า ในกาลต่อไป\\
\begin{center}
\emph{บทขอให้งดโทษนี้ มิได้เป็นการล้างบาป, เป็นเพียงการเปิดเผยตัวเอง;
และคำว่าโทษในที่นี้มิได้หมายถึงกรรม : หมายถึงโทษเพียงเล็กน้อยซึ่งเป็น “ส่วนตัว” ระหว่างกัน ที่พึงอโหสิกันได้.
การขอขมาชนิดนี้สำเร็จผลได้ ในเมื่อผู้ขอตั้งใจทำจริงๆ, และเป็นเพียงศีลธรรม และสิ่งที่ควรประพฤติ.}
\end{center}
\pagebreak

\section{ธัมมานุสสติ}
\hrule
\begin{center}
\textbf{(หันทะ มะยัง ธัมมานุสสะตินะยัง กะโรมะ เส)}\\
เชิญเถิด เราทั้งหลาย ทำความตามระลึกถึงพระธรรมเถิด
\end{center}
\textbf{ส๎วากขาโต ภะคะวะตา ธัมโม,}\\
\indent พระธรรมเป็นสิ่งที่พระผู้มีพระภาคเจ้า ได้ตรัสไว้ดีแล้ว\\
\textbf{สันทิฏฐิโก,}\\
\indent เป็นสิ่งที่ผู้ศึกษาและปฏิบัติพึงเห็นได้ด้วยตนเอง\\
\textbf{อะกาลิโก,}\\
\indent เป็นสิ่งที่ปฏิบัติได้ และให้ผลได้ ไม่จำกัดกาล\\
\textbf{เอหิปัสสิโก,}\\
\indent เป็นสิ่งที่ควรกล่าวกะผู้อื่นว่า ท่านจงมาดูเถิด\\
\textbf{โอปะนะยิโก,}\\
\indent เป็นสิ่งที่ควรน้อมเข้ามาใส่ตัว\\
\textbf{ปัจจัตตัง เวทิตัพโพ วิญญูหี  ติ.}\\
\indent เป็นสิ่งที่ผู้รู้ก็รู้ได้เฉพาะตน, ดังนี้.

\pagebreak
\section{ธัมมาภิคีติ}
\hrule
\begin{center}
\textbf{(หันทะ มะยัง ธัมมาภิคีติง กะโรมะ เส)}\\
เชิญเถิด เราทั้งหลาย ทำความขับคาถา พรรณนาเฉพาะพระธรรมเถิด
\end{center}
\textbf{ส๎วากขาตะตาทิคุณะโยคะวะเสนะ เสยโย,}\\
\indent พระธรรมเป็นสิ่งที่ประเสริฐเพราะประกอบด้วยคุณ,\\ 
\indent คือความที่พระผู้มีพระภาคเจ้าตรัสไว้ดีแล้วเป็นต้น\\
\textbf{โย มัคคะปากะปะริยัตติวิโมกขะเภโท,}\\
\indent เป็นธรรมอันจำแนกเป็นมรรคผลปริยัติและนิพพาน\\
\textbf{ธัมโม กุโลกะปะตะนา ตะทะธาริธารี,}\\
\indent เป็นธรรมทรงไว้ซึ่งผู้ทรงธรรม จากการตกไปสู่โลกที่ชั่ว\\
\textbf{วันทามะหัง ตะมะหะรัง วะระธัมมะเมตัง,}\\
\indent ข้าพเจ้าไหว้พระธรรมอันประเสริฐนั้น อันเป็นเครื่องขจัดเสียซึ่งความมืด\\
\textbf{ธัมโม โย สัพพะปาณีนัง สะระณัง เขมะมุตตะมัง,}\\
\indent พระธรรมใดเป็นสรณะอันเกษมสูงสุดของสัตว์ทั้งหลาย\\
\textbf{ทุติยานุสสะติฏฐานัง วันทามิ ตัง สิเรนะหัง,}\\
\indent ข้าพเจ้าไหว้พระธรรมนั้นอันเป็นที่ตั้งแห่งความระลึกองค์ที่สองด้วยเศียรเกล้า\\
\textbf{ธัมมัสสาหัส๎มิ ทาโสวะ (ทาสีวะ), ธัมโม เม สามิกิสสะโร,}\\
\indent ข้าพเจ้าเป็นทาสของพระธรรม, พระธรรมเป็นนายมีอิสระเหนือข้าพเจ้า\\
\textbf{ธัมโม ทุกขัสสะ ฆาตา จะ วิธาตา จะ หิตัสสะ เม,}\\
\indent พระธรรมเป็นเครื่องกำจัดทุกข์, และทรงไว้ซึ่งประโยชน์แก่ข้าพเจ้า\\
\textbf{ธัมมัสสาหัง นิยยาเทมิ สะรีรัญชีวิตัญจิทัง,}\\
\indent ข้าพเจ้ามอบกายถวายชีวิตนี้แด่พระธรรม\\
\textbf{วันทันโตหัง (วันทันตีหัง) \footnote{คำในวงเล็บสำหรับผู้หญิงว่า} จะริสสามิ, ธัมมัสเสวะ สุธัมมะตัง,}\\
\indent ข้าพเจ้าผู้ไหว้อยู่จักประพฤติตาม, ซึ่งความเป็นธรรมดีของพระธรรม\\
\textbf{นัตถิ เม สะระณัง อัญญัง, ธัมโม เม สะระณัง วะรัง,}\\
\indent สรณะอื่นของข้าพเจ้าไม่มี, พระธรรมเป็นสรณะอันประเสริฐของข้าพเจ้า\\
\textbf{เอเตนะ สัจจะวัชเชนะ, วัฑเฒยยัง สัตถุสาสะเน,}\\
\indent ด้วยการกล่าวคำสัตย์นี้, ข้าพเจ้าพึงเจริญในพระศาสนาของพระศาสดา\\
\textbf{ธัมมัง เม วันทะมาเนนะ (วันทะมานายะ), ยัง ปุญญัง ปะสุตัง อิธะ,}\\
\indent ข้าพเจ้าผู้ไหว้อยู่ซึ่งพระธรรม, ได้ขวนขวายบุญใดในบัดนี้\\
\textbf{สัพเพปิ อันตะรายา เม, มาเหสุง ตัสสะ เตชะสา.}\\
\indent อันตรายทั้งปวงอย่าได้มีแก่ข้าพเจ้าด้วยเดชแห่งบุญนั้น.
\begin{center}
(หมอบกราบ)
\end{center}
\textbf{กาเยนะ วาจายะ วะ เจตะสา วา,}\\
\indent ด้วยกายก็ดี ด้วยวาจาก็ดี ด้วยใจก็ดี\\
\textbf{ธัมเม กุกัมมัง ปะกะตัง มะยา ยัง,}\\
\indent กรรมน่าติเตียนอันใดที่ข้าพเจ้ากระทำแล้วในพระธรรม\\
\textbf{ธัมโม ปะฏิคคัณ๎หะตุ อัจจะยันตัง,}\\
\indent ขอพระธรรมจงงดซึ่งโทษล่วงเกินอันนั้น\\
\textbf{กาลันตะเร สังวะริตุง วะ ธัมเม.}\\
\indent เพื่อการสำรวมระวังในพระธรรมในกาลต่อไป\\

\pagebreak
\section{สังฆานุสสติ}
\hrule
\begin{center}
\textbf{(หันทะ มะยัง สังฆานุสสะตินะยัง กะโรมะ เส)}\\
เชิญเถิด เราทั้งหลาย ทำความตามระลึกถึงพระสงฆ์เถิด
\end{center}
\textbf{สุปะฏิปันโน ภะคะวะโต สาวะกะสังโฆ,}\\
\indent สงฆ์สาวกของพระผู้มีพระภาคเจ้า หมู่ใด, ปฏิบัติดีแล้ว\\
\textbf{อุชุปะฏิปันโน ภะคะวะโต สาวะกะสังโฆ,}\\
\indent สงฆ์สาวกของพระผู้มีพระภาคเจ้า หมู่ใด, ปฏิบัติตรงแล้ว\\
\textbf{ญายะปะฏิปันโน ภะคะวะโต สาวะกะสังโฆ,}\\
\indent สงฆ์สาวกของพระผู้มีพระภาคเจ้า หมู่ใด, ปฏิบัติเพื่อรู้ธรรมเป็นเครื่องออกจากทุกข์แล้ว\\
\textbf{สามีจิปะฏิปันโน ภะคะวะโต สาวะกะสังโฆ,}\\
\indent สงฆ์สาวกของพระผู้มีพระภาคเจ้าหมู่ใด, ปฏิบัติสมควรแล้ว\\
\textbf{ยะทิทัง, ได้แก่บุคคลเหล่านี้คือ :-\\
\indent จัตตาริ ปุริสะยุคานิ อัฏฐะ ปุริสะปุคคะลา,}\\
\indent คู่แห่งบุรุษ ๔ คู่, นับเรียงตัวบุรุษได้ ๘ บุรุษ \\
\textbf{เอสะ ภะคะวะโต สาวะกะสังโฆ,}\\
\indent นั่นแหละ สงฆ์สาวกของพระผู้มีพระภาคเจ้า\\
\textbf{อาหุเนยโย,}\\
\indent เป็นสงฆ์ควรแก่สักการะ ที่เขานำมาบูชา\\
\textbf{ปาหุเนยโย,}\\
\indent เป็นสงฆ์ควรแก่สักการะที่เขาจัดไว้ต้อนรับ\\
\textbf{ทักขิเณยโย,}\\
\indent เป็นผู้ควรรับทักษิณาทาน\\
\textbf{อัญชะลิกะระณีโย,}\\
\indent เป็นผู้ที่บุคคลทั่วไปควรทำอัญชลี\\
\textbf{อะนุตตะรัง ปุญญักเขตตัง โลกัสสา ติ.}\\
\indent เป็นเนื้อนาบุญของโลก, ไม่มีนาบุญอื่นยิ่งกว่า ดังนี้\\

\pagebreak
\section{สังฆาภิคีติ}
\hrule
\begin{center}
\textbf{(หันทะ มะยัง สังฆาภิคีติง กะโรมะ เส)}\\
เชิญเถิด เราทั้งหลาย ทำความขับคาถา พรรณนาเฉพาะพระสงฆ์เถิด
\end{center}
\textbf{สัทธัมมะโช สุปะฏิปัตติคุณาทิยุตโต,}\\
\indent พระสงฆ์ที่เกิดโดยพระสัทธรรม, ประกอบด้วยคุณมีความปฏิบัติดีเป็นต้น\\
\textbf{โยฏฐัพพิโธ อะริยะปุคคะละสังฆะเสฏโฐ,}\\
\indent เป็นหมู่แห่งพระอริยบุคคลอันประเสริฐแปดจำพวก\\
\textbf{สีลาทิธัมมะปะวะราสะยะกายะจิตโต,}\\
\indent มีกายและจิตอันอาศัยธรรมมีศีลเป็นต้นอันบวร\\
\textbf{วันทามะหัง ตะมะริยานะ คะณัง สุสุทธัง,}\\
\indent ข้าพเจ้าไหว้หมู่แห่งพระอริยเจ้าเหล่านั้น อันบริสุทธิ์ด้วยดี\\
\textbf{สังโฆ โย สัพพะปาณีนัง สะระณัง เขมะมุตตะมัง,}\\
\indent พระสงฆ์หมู่ใดเป็นสรณะอันเกษมสูงสุดของสัตว์ทั้งหลาย\\
\textbf{ตะติยานุสสะติฏฐานัง วันทามิ ตัง สิเรนะหัง,}\\
\indent ข้าพเจ้าไหว้พระสงฆ์หมู่นั้น อันเป็นที่ตั้งแห่งความระลึกองค์ที่สาม ด้วยเศียรเกล้า\\
\textbf{สังฆัสสาหัส๎มิ ทาโสวะ (ทาสีวะ), สังโฆ เม สามิกิสสะโร,}\\
\indent ข้าพเจ้าเป็นทาสของพระสงฆ์, พระสงฆ์เป็นนายมีอิสระเหนือข้าพเจ้า\\
\textbf{สังโฆ ทุกขัสสะ ฆาตา จะ วิธาตา จะ หิตัสสะ เม,}\\
\indent พระสงฆ์เป็นเครื่องกำจัดทุกข์, และทรงไว้ซึ่งประโยชน์แก่ข้าพเจ้า\\
\textbf{สังฆัสสาหัง นิยยาเทมิ สะรีรัญชีวิตัญจิทัง,}\\
\indent ข้าพเจ้ามอบกายถวายชีวิตนี้แด่พระสงฆ์\\
\textbf{วันทันโตหัง (วันทันตีหัง) สังฆัสโสปะฏิปันนะตัง,}\\
\indent ข้าพเจ้าผู้ไหว้อยู่จักประพฤติตาม, ซึ่งความปฏิบัติดีของพระสงฆ์\\
\textbf{นัตถิ เม สะระณัง อัญญัง, สังโฆ เม สะระณัง วะรัง,}\\
\indent สรณะอื่นของข้าพเจ้าไม่มี, พระสงฆ์เป็นสรณะอันประเสริฐของข้าพเจ้า\\
\textbf{เอเตนะ สัจจะวัชเชนะ, วัฑเฒยยัง สัตถุ สาสะเน,}\\
\indent ด้วยการกล่าวคำสัตย์นี้, ข้าพเจ้าพึงเจริญในพระศาสนาของพระศาสดา\\
\textbf{สังฆัง เม วันทะมาเนนะ (วันทะมานายะ), ยัง ปุญญัง ปะสุตัง อิธะ,}\\
\indent ข้าพเจ้าผู้ไหว้อยู่ซึ่งพระสงฆ์, ได้ขวนขวายบุญใด ในบัดนี้\\
\textbf{สัพเพปิ อันตะรายา เม, มาเหสุง ตัสสะ เตชะสา.}\\
\indent อันตรายทั้งปวงอย่าได้มีแก่ข้าพเจ้าด้วยเดชแห่งบุญนั้น.\\
\begin{center}
(หมอบกราบ)
\end{center}
\textbf{กาเยนะ วาจายะ วะ เจตะสา วา,}\\
\indent ด้วยกายก็ดี ด้วยวาจาก็ดี ด้วยใจก็ดี\\
\textbf{สังเฆ กุกัมมัง ปะกะตัง มะยา ยัง,}\\
\indent กรรมน่าติเตียนอันใดที่ข้าพเจ้าได้กระทำแล้วในพระสงฆ์\\
\textbf{สังโฆ ปะฏิคคัณ๎หะตุ อัจจะยันตัง,}\\
\indent ขอพระสงฆ์ จงงดซึ่งโทษล่วงเกินอันนั้น\\
\textbf{กาลันตะเร สังวะริตุง วะ สังเฆ.}\\
\indent เพื่อการสำรวมระวัง ในพระสงฆ์ในกาลต่อไป\\
\begin{center}
(จบคำทำวัตรเย็น)
\end{center}











\end{document}